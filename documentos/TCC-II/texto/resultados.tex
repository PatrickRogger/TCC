\chapter{APLICAÇÃO DO PAMPAMT}
    
    Nesse capítulo será demonstrado a aplicação do programa desenvolvido utilizando dados reais. Onde os dados foram processados utilizando o PampaMT e também a forma tradicional que vinha sendo realizada. Essa forma tradicional compreende processar utilizando os \en{scripts} e programas desenvolvidos pelo GEOMA, como comentado anteriormente.
    
    A realização dos processamentos utilizando as duas técnicas simultâneas, foi realizada visando comparar o tempo e os períodos escolhidas para cada técnica.  
    
    \section{Área de Estudo}
        
        A área de estudo escolhida para esse teste de comparação são 12 estações magnetotelúricas de banda larga, localizadas no nordeste brasileiro dentro da provícia borborema (Figura \ref{local-bor}). As estações MT fazem parte do projeto ``Estudos geofísicos e tectônicos na Província Borborema, Nordeste do Brasil” /CNPQ (Projeto Milênio) e “Estudo da estrutura da litosfera do Nordeste do Brasil” /CNPQ (INCT – Tectônica), levantadas nos anos de 2007 e 2009.
        
    \begin{figure}[H]
        \caption{Mapa de Localização}
            \begin{center}
                \includegraphics[width=12cm]{texto/figura/local-bor.png}
            \end{center}
        \legend{\Fonte{\oautor.}}
        \label{local-bor}
    \end{figure}        
        
    \section{Contexto Geológico}
    
    Segundo \cite{almeida}, a Província Borborema caracteriza-se como um complexo conjunto de blocos crustais reunidos por causa de processos geológicos que finalizaram na Orogenia Brasiliana/Pan-africana (700 a 450 Ma). Devido à complexidade tectônica, diferentes estudos vêm sendo apresentados por distintos pesquisadores para explicar as características dessa estrutura \cite{van}; \cite{teseandrea}; \cite{santos2014deep}; \cite{padilha}; \cite{barbosa}. A Província Borborema limita-se a sul com o Cráton São Francisco; a oeste com a Bacia do Parnaíba (sedimentos Fanerozóicos); a norte e a leste com as bacias sedimentares costeiras e interiores do Nordeste do Brasil (bacias Potiguar, Pernambuco-Paraíba e Sergipe-Alagoas, além da bacia Tucano-Jatobá que transpassa o limite da província com o Cráton São Francisco) -- \cite{medeiros}.    
    
    \begin{figure}[H]
        \caption{Mapa Geológico}
            \begin{center}
                \includegraphics[width=15cm]{texto/figura/mapa_geo.png}
            \end{center}
        \legend{\Fonte{O Autor -- Base de dados \cite{cprm}, 2018.}}
        \label{local-bor}
    \end{figure}
        
    \section{Processamento dos Dados}
        
    \section{Resultados e Interpretação Geofísica}
        
        
