\setcounter{page}{14}
\chapter{INTRODUÇÃO}
   
    %A interação do vento solar com o campo magnético terrestre, induzem correntes elétricas que propagam e penetram no interior da Terra \cite{cagniard1953basic}. Essas ondas são chamadas de correntes telúricas, as correntes telúricas quando interagem com as diversas litologias, induzem correntes eletromagnéticas. O método geofísico magnetotelúrico usa essas correntes para investigar e mapear o interior da Terra \cite{vozoff1991magnetotelluric}.  

    %O método geofísico magnetotelúrico, utiliza as baixas frequências do espectro eletromagnético, para investigar a subsuperfície do planeta Terra \citar{cagniard1953basic}. A interação do vento solar com o campo magnético terrestre, compõe a origem dessas ondas eletromagnéticas vozoff1991magnetotelluric.  

    %\citar{falar sobre o campo magneticio/mt}
    
    %A grande complexidade dos dados \mt desestimula o desenvolvimento de \en{softwares}. Atualmente os programas destinados a esse tipo de atividade são proprietários, com alto valor comercial, ou são livres operacionais exclusivamente por linhas de comando.  
    
    O método geofísico magnetotelúrico (MT) surge da interação de eventos naturais de origem espacial com a Terra. No caso, a interação do vento solar com a magnetosfera terrestre, e as altas ocorrências de tempestades equatoriais geram ondas eletromagnéticas, que por sua vez penetram e propagam-se por todo o interior da Terra. A indução gerada pela propagação das correntes (correntes telúricas) formam o sinal analítico do método geofísico magnetotelúrico -- MT \cite{parkinson93}. As aplicações do método giram em torno da prospecção de petróleo e estudos crustais. Essas aplicações justificam-se pela profundidade de investigação do método, que varia de 100 metros a 200 quilômetros.

    %No entanto, devido a complexidade do método, tem-se um desestimulo ao desenvolvimento de \softwares, o que produz um ambiente onde programas proprietários agregam um alto valor comercial, muitas vezes inviabilizando o uso acadêmico. E ainda, os programas de uso livre não são amigáveis com o usuário. De fato, o seu uso se dá, unicamente, por linhas de comando que não é familiar para a maioria dos potencias usuários do MT.

    No entanto, os programas desenvolvidos para as aplicações do MT e para o processamento do mesmo, agregam um alto valor comercial devido aos poucos núcleos de desenvolvimento. Já as aplicações desenvolvidas pelo meio acadêmico são em sua maioria manipuladas via linhas de comandos. Tais fatores criam um ambiente, onde novos usuários que desejam trabalhar com tal método, tenham que desprender horas de estudos adicionais de programação para que possam produzir resultados.       
    
    Para o usuário que deseja trabalhar com os programas livres e de uso acadêmico, pode encontrar na comunidade MTnet \cite{mtnet}, uma vasta biblioteca de programas e pesquisadores na área do MT, a comunidade reúne as aplicações destinadas aos processamentos de dados, tais como: \en{softwares} de pré-processamento, inversão, tratamento estatísticos, dentre outros. Os programas alocados no MTnet são de uso livre, como comentado anteriormente, e são destinados exclusivamente a comunidade acadêmica. Portando a proposta deste trabalho visa unir os programas livres alocados na comunidade em uma única plataforma gráfica, fazendo uso de uma GUI (\en{Graphical User Interface}) simples, agindo como intermediária entre o usuário e os programas já desenvolvidos por toda a comunidade.
    
    No Brasil, a metodologia adotada pelos grupos de pesquisas na área do MT, faz uso principalmente do programa EMTF para a obtenção das matrizes de impedância eletromagnética (seção \ref{sec-impedancia}), tal elemento é fundamental para o processamento de dados MT. O programa desenvolvido no âmbito do trabalho seguirá a metodologia adotada pelos grupos de pesquisas do Brasil e manterá todo o arcabolso de processamento adotados por eles.
    
    %Este trabalho foi idealizado com o objetivo de melhorar a interação com o usuário através do desenvolvimento de um software livre para tratar e manipular dados MT. A abordagem adotada no trabalho será a criação de uma GUI (Graphical Interface User) escrita e desenvolvida totalmente com programas livres. A GUI servirá como intermédio entre o usuário e os programas livres, já consagrados no trato com esse tipo de dado (EGBERT,1997a). Compreende junto à interface os seguintes objetivos: desenvolver novos algoritmos escritos em Python e otimizar códigos já existentes frente as novas tecnologias.

    
    %\citar{como esta sendo tratado os dados, citar os trabalhos(marcelo, marcos)}
    
     
    
