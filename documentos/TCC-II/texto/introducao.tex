\setcounter{page}{10}
\chapter{INTRODUÇÃO}
   
    %A interação do vento solar com o campo magnético terrestre, induzem correntes elétricas que propagam e penetram no interior da Terra \cite{cagniard1953basic}. Essas ondas são chamadas de correntes telúricas, as correntes telúricas quando interagem com as diversas litologias, induzem correntes eletromagnéticas. O método geofísico magnetotelúrico usa essas correntes para investigar e mapear o interior da Terra \cite{vozoff1991magnetotelluric}.  

    %O método geofísico magnetotelúrico, utiliza as baixas frequências do espectro eletromagnético, para investigar a subsuperfície do planeta Terra \citar{cagniard1953basic}. A interação do vento solar com o campo magnético terrestre, compõe a origem dessas ondas eletromagnéticas vozoff1991magnetotelluric.  

    %\citar{falar sobre o campo magneticio/mt}
    
    %A grande complexidade dos dados \mt desestimula o desenvolvimento de \en{softwares}. Atualmente os programas destinados a esse tipo de atividade são proprietários, com alto valor comercial, ou são livres operacionais exclusivamente por linhas de comando.  
    
    A interação do vento solar com a magnetosfera terrestre, gera correntes elétricas na alta atmosfera. Essas correntes são chamadas de correntes telúricas, visto que elas apresentam a propriedade de penetrar e se propagar por todo o interior da Terra. A indução gerada pela propagação das correntes telúricas forma o sinal analítico do método geofísico magnetotelúrico (MT) \cite{parkinson93}. As aplicações do MT giram em torno da prospecção de petróleo e estudos crustais. A aplicação justifica-se pela profundidade de investigação do método, que varia de 100 metros a 200 quilômetros.

    
    No entanto, devido a complexidade do método, tem-se um desestimulo ao desenvolvimento de \softwares, o que produz um ambiente onde programas proprietários agregam um alto valor comercial, muitas vezes inviabilizando o uso acadêmico. E ainda, os programas de uso livre não são amigáveis com o usuário. De fato, o seu uso se dá, unicamente, por linhas de comando que não é familiar para a maioria dos potencias usuários do MT.

    A comunidade MTnet \cite{mtnet}, mantém laços com diversos pesquisados na área do MT, e reúne as aplicações destinadas aos processamentos, tais como: \en{softwares} de pré-processamento, inversão, tratamento estatísticos, dentre outros. Os programas alocados no MTnet são de uso livre e destinados a comunidade acadêmica. %O programa desenvolvido faz uso de toda a base de programas do MTnet, como forma de dependência do código fonte.
    
    A proposta deste trabalho visa unir os programas livres em uma única plataforma. Essa será construída para ser amigável, fazendo uso de uma GUI (\en{Graphical User Interface}) simples, agindo como intermediaria entre o usuário e os programas disponíveis no MTnet.
    
    %Este trabalho foi idealizado com o objetivo de melhorar a interação com o usuário através do desenvolvimento de um software livre para tratar e manipular dados MT. A abordagem adotada no trabalho será a criação de uma GUI (Graphical Interface User) escrita e desenvolvida totalmente com programas livres. A GUI servirá como intermédio entre o usuário e os programas livres, já consagrados no trato com esse tipo de dado (EGBERT,1997a). Compreende junto à interface os seguintes objetivos: desenvolver novos algoritmos escritos em Python e otimizar códigos já existentes frente as novas tecnologias.

    
    %\citar{como esta sendo tratado os dados, citar os trabalhos(marcelo, marcos)}
    
     
    
