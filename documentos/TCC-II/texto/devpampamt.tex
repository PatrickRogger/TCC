\chapter{DESENVOLVIMENTO E RESULTADOS DO PampaMT}

    O \en{software} desenvolvido recebeu o nome de PampaMT, em homenagem a Universidade Federal do Pampa. O desenvolvimento iniciou-se em janeiro de 2018, com o primeiro protótipo de codinome RiMT, vizando testar os conceitos e a viabilidade das funções para o programa. Em junho de 2018 o código foi reestruturado e reescrito, atentando para os problemas e adicionando novas funções baseando no protótipo. A principal mudança foi a construção do \en{software} em módulos, facilitando a adição e manutenção de novas funções.
    
    No apêndice A pode ser encontrado o caminho para o código fonte do programa, junto com as informações para a instalação. O PampaMT foi desenvolvido para ambiente Linux rodando em base Debian. O código foi escrito em \Python, com alguns trechos escritos em \Shell{} para instalação e comunicação da interface com os executáveis dos programas \textbf{Dnff} e \textbf{TranMT}.
    
    O PampaMT foi dividido em duas etapas de processamento: a primeira destinada a criação do projeto; escolha dos arquivos a serem processados e a processamento EMTF. Já segunda parte foi destinada a escolha das melhores rodadas e períodos, esse processo destina a maior interação com o usuário e representa a principal justificativa para o desenvolvimento do \en{software}. 
    
    A Figura \ref{tela_inicial} mostra a primeira tela ao executar o PampaMT, nela pode-se escolher criar um novo projeto ou abrir um já existente.
    
    \begin{figure}[H]
        \caption{Tela Inicial PampaMT}
            \begin{center}
                \includegraphics[width=16cm]{texto/figura/tela1_com_fluxograma.eps}
            \end{center}
        \legend{\Fonte{\oautor.}}
        \label{tela_inicial}
    \end{figure}
    
     Após a escolha do diretório para um novo projeto, o usuário é direcionado a escolha dos arquivos \en{TS}, nele o usuário pode escolher entre três equipamentos: ADU-06, ADU-07 e LiMS, a seleção pode ser automática ou adicionada cada estação individualmente (Figura \ref{tela_sel_site}). 
    
    \begin{figure}[H]
        \caption{Seleção das Estações}
            \begin{center}
                \includegraphics[width=16cm]{texto/figura/tela2_com_fluxograma.eps}
            \end{center}
        \legend{\Fonte{\oautor.}}
        \label{tela_sel_site}
    \end{figure}
    
    Os dados após a seleção, são copiados para o diretório: \textbf{DADOS\_MT/projeto}, esse processo é realizado para prever eventuais perdas dos arquivos. Os dados então são convertidos e salvos no diretório: \textbf{PROC\_MT/projeto} 
    
    O usuário será levado a tela do processamento EMTF, esse processo já estabelece algumas configurações padrão, entretanto o usuário pode alterar qualquer configuração, tais como, escolher uma nova janela ou alterar o horário do relógio dos dados (Figura \ref{conf-procZ}).
    
    \begin{figure}[H]
        \caption{Tela de Configuração para o EMTF}
            \begin{center}
                \includegraphics[width=16cm]{texto/figura/tela3_com_fluxograma.eps}
            \end{center}
        \legend{\Fonte{\oautor.}}
        \label{conf-procZ}
    \end{figure}
    
    O processo EMTF tende a demorar um tempo considerável, cerca de 1 a 2 minutos para cada estação, para um levantamento típico de 30 estações por perfil esse processo pode demorar de 20 a 30 minutos, visto a grande quantidade de recursos do computador que ele consome. Finalizado o processo EMTF a janela é fechada e inicia-se o tela principal do PampaMT (figura \ref{tela-prin}).
    
    \begin{figure}[H]
        \caption{Tela Principal PampaMT}
            \begin{center}
                \includegraphics[width=16cm]{texto/figura/tela4.png}
            \end{center}
        \legend{\Fonte{\oautor.}}
        \label{tela-prin}
    \end{figure}
    
    
    Na tela principal contem todas as funcionalidades do PampaMT, incluindo a etapa de criação de um novo projeto. O carácter modular do PampaMT ajuda na adição de novas funcionalidades, como por exemplo, integração por programas SIG, integração com visualizados de dados, como o GMT, dentre outros. Um exemplo notável é a adição do programa RhoplusGUI em desenvolvido pelo autor, para o projeto PIBIC/INPE/CNPq ``Desenvolvimento de Interface Gráfica Amigável para Validação de Dados Magnetotelúrico a Partir do Processamento Rho+''. Esse programa auxilia na manipulação de dados para o processamento Rho+ \cite{parker1996optimal}, onde foi necessário adicionar poucas linhas de código para inclui-lo no PampaMT (Figura \ref{rhoplusGUI}).
    
    \begin{figure}[H]
        \caption{Integração com Outros Programas}
            \begin{center}
                \includegraphics[width=16cm]{texto/figura/rhoplusGUI.eps}
            \end{center}
        \legend{\Fonte{\oautor.}}
        \label{rhoplusGUI}
    \end{figure}
    
    A principal função que o usuário utilizará, será a escolha dos melhores períodos e rodadas. Esse processo vinha sendo executado, plotando cada arquivo \en{.zss} e contando manualmente a posição dos melhores períodos. Então, após a contagem, o usuário deve anotar os pontos que indicam os períodos, e finalmente executar o \en{script}: \textbf{ToJones}. Esse \en{script} mescla os arquivos \en{.zss} com os períodos escolhidos e converte-os para o formato \en{J} (\en{.dat}).
    
    Esse processo foi incorporado no PampaMT com a escolha dos períodos sendo realizada com o cursor. O usuário habilita a função de seleção, e o programa plota todos os pontos possíveis para a rodada escolhida, por fim o usuário arrasta uma janela de seleção e todos os períodos contidos nessa janela são selecionados (Figura \ref{selc-perio}).
    
    \begin{landscape}
    \begin{figure}[h]
        \caption{Seleção dos Períodos}
            \begin{center}
                \includegraphics[width=23cm]{texto/figura/tela3_selecionando.png}
            \end{center}
        \legend{\Fonte{\oautor.}}
        \label{selc-perio}
    \end{figure}
    \end{landscape}
    
    Após escolher os melhor períodos o usuário pode executar o \en{script}: \textbf{ToJones}, onde esse é realizado ao pressionar o botão no canto inferior esquerdo, o PampaMT abre uma caixa de diálogo para nomear o arquivo de saída, e executa o \en{script}, finalizando a ultima etapa do pré-processamento.
    
    A utilização do PampaMT para o pré-processamento elimina completamente o uso de linhas de comando, assim o tempo de processamento e aprendizagem é diminuído drasticamente. Para efeitos de comparação, ao utilizar o programa em fase alfa, usuários que nunca tiveram contato com o terminal \Shell, puderam executar e processar os dados com sucesso. A Figura \ref{teste1} mostra o tempo que três usuários geofísicos, onde não possuem familiaridade com o terminal \en{shell}, utilizaram para concluir o pré-processamento de duas estações teste: ufb104a e ufb105a.
    
    \begin{figure}[H]
        \caption{Tempo de Processamento -- Teste Alfa}
            \begin{center}
                \includegraphics[width=13cm]{texto/figura/tempoLWM.png}
            \end{center}
        \legend{\Fonte{\oautor.}}
        \label{teste1}
    \end{figure}
    
    
    
