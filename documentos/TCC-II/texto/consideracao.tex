\chapter{CONSIDERAÇÕES FINAIS}
    
    Expandindo os testes alfa do programa foi proposto o teste com usuários, que diferentes dos primeiros, não possem nenhuma familiaridade com o terminal \en{shell} implicando em não possuírem familiaridade com o processamento de dados MT. Os testes aplicados sobre esses usuários procedeu, partindo de uma explicação de como é a metodologia de processamento de dados. Concluído a explanação de como é realizado o processamento, os usuários foram instruídos a processar duas estações magnetotelúricas reais contando apenas com a intuitividade da interface do PampaMT, os mesmos usuários foram avaliados no quesito de tempo e dificuldade para realizar tais processos.
    
    Os resultados obtidos nesse segundo teste pode ser visto na figura \ref{teste1}, os três usuários testados obtiveram exito no processamento com um tempo hábio de em média 10 minutos para cada estação. No quesito de intuitividade os primeiros processos, como: conversão dos dados e o cálculo do tensor impedância, todos os usuários executaram sem qualquer dificuldade. Entretanto para a segunda parte, que consiste na escolha dos melhores períodos, os usuários necessitaram de uma segunda explicação de como procedia o processo. Esse fato pode ser resolvido com a implementação de tutoriais na seção de ajuda no programa, onde já está prevista tal implementação em versões futuras do programa.     
    
    \begin{figure}[H]
        \caption{Tempo de Processamento -- Teste Alfa}
            \begin{center}
                \includegraphics[width=10cm]{texto/figura/tempoLWM.png}
            \end{center}
        \legend{\Fonte{\oautor.}}
        \label{teste1}
    \end{figure}
    
    A partir dos resultados do segundo teste, provou-se que novos usuários que desejam trabalhar com dados MT consigam executar tal tarefa de forma fácil. Ao mesmo passo podem aprender como deve ser feito o processamento dos dados via terminal, habilitando a janela de visualização, onde são mostrados como são realizados os comandos internos no programa. Esse fato propicia que o PampaMT sejá aplicado para novos alunos de graduação, mestrado ou até mesmo doutorado, expandindo também para minicursos e pequenos treinamentos.
    
    
    \section{Desenvolvimentos Futuros para o PampaMT}
        
        Após o desenvolvimento do PampaMT para o presente trabalho e também após a utilização do mesmo para processamentos efetivos, propôs-se o desenvolvimento de novas funções e ferramentas, bem como, técnicas que possam melhorar a eficiência na utilização. Portanto o desenvolvimento do programa será continuado, a seguir são apresentadas algumas funções ou ideias que serão imprementadas com a continuidade do projeto.
        
        \begin{itemize}
         \item Ampliar a API de comunição com o EMTF, visando extrair ao máximo as funções dele;
         \item Implementar novas técnicas de analises temporais, alternativas ao EMTF;
         \item Adicionar novos processamentos, como: inversão 1D; inversão 2D e 3D;
         \item Adicionar construtores de grades para os referidos processos de inversão;
         \item Distribuir programa para outros sistemas operacionais.
        \end{itemize}
