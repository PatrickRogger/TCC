\chapter{CONSIDERAÇÕES FINAIS}
    
    Rodando a versão de teste interno (versão alfa), a utilização do PampaMT para processamentos sugere que o principal fator, o tempo, foi reduzido. O que tornou o processamento de dados magnetotelúricos mais dinâmico. Esse fator permite que o pré-processamento sejá realizado até mesmo em campo aumentando ainda mais a produtividade de grandes projetos. Outro fator que vale ressaltar é a diminuição do tempo de treinamento para novos usuários, onde pelos testes provou-se que mesmo usuário que não possuem familiaridade com o terminal realizaram o pré-processamento de dados MT  satisfatoriamente.     
    
    \section{Desenvolvimentos Futuros para o PampaMT}
        
        Após o desenvolvimento do PampaMT para o presente trabalho e também após a utilização do mesmo para processamentos efetivos, propôs-se o desenvolvimento de novas funções e ferramentas, bem como, técnicas que possam melhorar a eficiência na utilização. Portanto o desenvolvimento do programa será continuado, a seguir são apresentadas algumas funções ou ideias que serão imprementadas com a continuidade do projeto.
        
        \begin{itemize}
         \item Ampliar a API de comunição com o EMTF, visando extrair ao máximo as funções dele;
         \item Implementar novas técnicas de analises temporais, alternativas ao EMTF;
         \item Adicionar novos processamentos, como: inversão 1D; inversão 2D e 3D;
         \item Adicionar construtores de grades para os referidos processos de inversão;
         \item Distribuir programa para outros sistemas operacionais.
        \end{itemize}
