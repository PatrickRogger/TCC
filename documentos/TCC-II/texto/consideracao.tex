\chapter{CONSIDERAÇÕES FINAIS}
    
    Rodando a versão de teste interno (versão alfa), a utilização do PampaMT para processamentos sugere que o principal fator, o tempo, foi drasticamente reduzido. O que tornou o processamento de dados magnetotelúricos mais dinâmico.
    
    A continuidade desse trabalho será ampliar o uso do PampaMT, adicionando novas funções, tais como: processos de inversão, modelagem, referência remota, dentre outros.
    
    %\citar{escrever um pouco mais}
