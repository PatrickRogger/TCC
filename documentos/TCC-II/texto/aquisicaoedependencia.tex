\chapter{AQUISIÇÃO DE DADOS, DEPENDÊNCIAS E PROCESSAMENTO MT}
    
    Neste capítulo será discutido como é realizada a aquisição de dados MT e quais as técnicas atualmente utilizadas para o processamento dos dados. Também será mostrado quais as dependências que foram necessárias para a construção do \software, dentre elas estão: Kivy, EMTF (Dnff e TranMT) e conversores de dados.
        
    \section{Aquisição de Dados MT}        
    
        A aquisição de dados MT consiste na obtenção dos campos elétricos ($\campo{E}_x$ e $\campo{E}_y$) e magnéticos por meio dos campos magnetizantes ($\campo{H}_x$, $\campo{H}_y$ e $\campo{H}_z$), esses são os parâmetros essenciais para o cálculo da impedância ($Z$), Equações \ref{EZHvetor} e \ref{tensor-impe}.
        
        Devido a sensibilidade do sinal das sondagens MT, os sensores devem proporcionar uma alta relação sinal/ruído e alta capacidade de ampliar o sinal medido.
        
        O arranjo amplamente adotado para aquisição consiste de três magnetômetros distribuídos cada um paralelo a um eixo cartesiano, responsáveis pelas medidas dos campos magnéticos. Para os campos elétricos são distribuídos dois arranjos de eletrodos não polarizados, acoplados horizontalmente, no sentido $x$ e $y$. A Figura \ref{fig-arranjo-mt} mostra a disposição dos sensores na superfície.
        
        Vale ressaltar que o eixo $x$ da composição cartesiana deve estar paralelo a direção do fluxo magnético terrestre, ou seja, direcionado ao polo magnético terrestre\footnote{Ponto na superfícies terrestre, onde, a inclinação magnética do modelo matemático para o campo geomagnético é $+90^\circ$}. 
        
        \begin{figure}[H]
            \caption{Arranjo para Aquisição de dados MT.}
                \begin{center}
                    \includegraphics[width=10cm]{texto/figura/arranjo-mt.eps}
                \end{center}
            \legend{\Fonte{\oautor.}}
            \label{fig-arranjo-mt}
        \end{figure}
        
        Os sensores registram a variação da amplitude do sinal em função do tempo. Esses registros são chamados de series temporais e são considerados os dados brutos do método (Figura \ref{serie-temporal}).
        
        \begin{figure}[H]
            \caption{Exemplo de Série Temporal dos Dados MT.}
                \begin{center}
                    \includegraphics[width=16cm]{texto/figura/bor604a091_04D.pdf}
                \end{center}
            \legend{\Fonte{\oautor.}}
            \label{serie-temporal}
        \end{figure}
        
        Devido ao grande intervalo do espectro eletromagnético que é coletado nas sondagens MT (10\textsuperscript{-4} Hz a 10\textsuperscript{4} Hz), são configuradas varias taxas de aquisições diferentes. Para cada escolha de taxa de aquisição é considerada a representatividade do sinal respeitando a frequência de Nyquist \cite{nyquist28}. A representatividade é muito importante, pois, o sinal medido pelos sensores é a composição de várias ondas com frequências angulares diferentes. Se a taxa de aquisição for menor que duas vezes a frequência da onda, ela não pode ser representada fielmente.  
        
        A Figura \ref{fig-aquisicao} exemplifica o conceito apresentado no paragrafo anterior, No exemplo é mostrado a composição de um onda com 10 frequências diferentes ($f_{\omega(t)}$), variando de 1 a 10 Hz. Se atribuirmos à ela uma taxa de aquisição de 10 Hz pode-se perceber que a frequência de 6 Hz não é representada corretamente. Já para a frequência de 1 Hz ela é super representada, isso acaba aumentando o tamanho dos arquivos de aquisição. A escolha da taxa de aquisição deve conciliar da melhor forma possível esses dois fatores.
        
        \begin{figure}[H]
            \caption{Aquisição de Dados Discretos.}
                \begin{center}
                    \includegraphics[width=13cm]{texto/figura/fourier2.eps}
                \end{center}
            \legend{\Fonte{\oautor.}}
            \label{fig-aquisicao}
        \end{figure}
    
        As taxas de aquisições comumente utilizadas são estimadas por potências de 2, o que facilita na decomposição das frequências pela transformada de Fourier. Cada taxa de aquisição é chamada de \en{Banda} e varia de nome para cada equipamento utilizado. 
    
    \section{Processamento de Dados MT}
    
        Assim como outros métodos geofísicos, para relacionar o parâmetro físico estudado em função da profundidade, é realizado o processo de inversão. Porém para dados MT faz-se necessário antes das técnicas de inversão, uma primeira etapa de processamento.
        
        A primeira etapa do processamento consiste, sucintamente, em realizar processamentos de filtragem, tratamentos estatísticos, conversão de dados, mudança de domínios e mesclagem de arquivos. Inicia-se o processamento com a conversão dos arquivos de binários para ASCII. Embora seja um processo opcional para alguns equipamentos suportados nativamente pelo pacote EMTF, a boa prática é realizada para melhorar a legibilidade por parte dos usuários sobre os dados ou para tornar compatível com a entrada de dados do pacote EMTF.
        
        Após a conversão são utilizadas técnicas de filtragem, mudança do domínio dos dados de tempo para frequência angular, e cálculo do tensor impedância ($Z$). Esse processo será demonstrado mais detalhadamente no seção \ref{sec-robusto}, cuja a técnica recebe o nome de processamento Robusto, EMTF \cite{robusto-egbert}. Atualmente, esta é a técnica mais confiável e amplamente utilizadas no meio acadêmico para tratamento dos dados.
        
        Como comentado na seção anterior, devido as limitações dos equipamentos, os dados são coletados separadamente para cada banda. Utilizando o processamento Robusto, é gerado um arquivo com extensão \en{.zss}, \en{.zrr} ou \en{.zmm}. Nesses arquivos estão armazenados os tensores de impedância, ou seja, cada componente da matriz $Z$ para cada período.
        
        A etapa seguinte do processamento consiste em escolher, dentro de todos arquivos, a melhor composição dos períodos para todo o espectro de estudo. Esse processo é minucioso e depende da experiência do usuário. Nesta etapa, deve-se plotar cada arquivo e verificar a sua coerência dentro do conjunto total dos dados.
        
        O último estágio do processamento antecessor a inversão é mesclagem dos períodos escolhidos em um único arquivo, que irão conter todas as informações necessárias para realizar os processos de inversão.
        
        A Figura \ref{fluxo-preprocessamento} ilustra as etapas do pré-processamento.
        
        \begin{figure}[H]
            \caption{Fluxograma de Pré-processamento.}
                \begin{center}
                    \includegraphics[width=15cm]{texto/figura/fluxo_preprocessamento.eps}
                \end{center}
            \legend{\Fonte{\oautor.}}
            \label{fluxo-preprocessamento}
        \end{figure}
        
    \section{Formatos de Arquivos de Dados MT}   
    
        Parte das funções do \software{} desenvolvido será a simplificação no processo de conversão de dados. Atualmente os formatos mais utilizados, são\footnote{O formato EDI (\en{Electrical Data Interchange}) não será discutido neste trabalho, porém, o programa esta preparado para trabalhar com os mesmo. Mais detalhes podem serem encontrados em \cite{edi-format}.}:
        
        {\footnotesize \noindent
            \begin{table}[H]
                \begin{tabular*}{1cm}{p{2.05cm}p{0.5cm}p{10cm}}
                    \en{Z-file}          & {\footnotesize $\rightarrow$} & \en{Z (Impedance Tensor) File (.zss)} \\
                    \en{J-format}        & {\footnotesize $\rightarrow$} & \en{Jones Format (.dat)} \\
                    %\en{EDI-format}      & {\footnotesize $\rightarrow$} & \en{Eletrical Data Interchange Format (.edi)} \\
                \end{tabular*}
            \end{table}}
        
       
        
        Os arquivos gerados pelos equipamentos variam de formato dependendo do fabricante do equipamento, os arquivos são utilizados para registrar as series temporais, nos quais são armazenadas as amplitudes registradas pelos sensores em função do tempo. A grande parte dos equipamentos tem como saída padrão arquivos em formato ASCII, porém para a série de equipamentos ADU, desenvolvidas pela Metronix faz uso de um formato binário para tais arquivos. Esses arquivos podem ser convertidos para ASCII e em um formato compatível com o EMTF através do comando -- \textbf{ats2asc} -- \cite{geoma-proc}.
        
        Os arquivos contendo as séries temporais após realizar a transformada de Fourier e o cálculo do tensor impedância, não são mais necessário. Os arquivos gerados após esse processo contem todas as informações das series temporais, associadas aos tensores de impedância. A partir do tensor impedância é possível estimar todos os parâmetros associados a cada período, tais como: $\rho_a$, $\phi$ e as componentes do próprio tensor: $Z_{xx}$, $Z_{xy}$, $Z_{yx}$ e $Z_{yy}$, as componentes do tensor são importantes para a interpretação da adimensionalidade dos dados.
        
        Obtidos os tensores impedância, através do \textbf{TranMT} (Programa adotado para os cálculos do tensor impedância -- EMTF), são gerados os arquivos \en{.zss}, \en{.zrr} ou {.zmm} dependendo do tipo de processo escolhido, mas todos seguem a mesma estrutura principal. Esses arquivos são estruturados na forma de blocos, onde, cada bloco representa um período e nele contém o próprio tensor impedância, a matriz de covariância inversa e a matriz de covariância residual.
        
        Exemplo de arquivo \en{Z}:
        
\begin{footnotesize}        
\begin{verbatim}
      **** IMPEDANCE IN MEASUREMENT COORDINATES ****
      ********** WITH FULL ERROR COVARINCE**********
     Robust Single station                                                           
     station    :ufb104a091_03B      
     coordinate  -10.47560  -38.43750  declination   -23.00
     number of channels   5   number of frequencies   16
      orientations and tilts of each channel 
         1     0.00     0.00 ufb104a  Hx    
         2    90.00     0.00 ufb104a  Hy    
         3     0.00     0.00 ufb104a  Hz    
         4     0.00     0.00 ufb104a  Ex    
         5    90.00     0.00 ufb104a  Ey    
     period : 1.116071E-03    decimation level   1    freq. band from
     1600 to   1984
     number of data point  29853 sampling freq. 4.096000E+03 Hz
      Transfer Functions
       3.7279E-03  5.5604E-03 -1.2527E-03 -4.9609E-03
       5.7064E+01  1.7047E+01  3.1437E+02  2.1458E+02
      -2.6267E+02 -1.9591E+02 -3.2340E+01 -1.3460E+00
      Inverse Coherent Signal Power Matrix
       1.3513E+05  0.0000E+00
      -1.3265E+04  1.9036E+03  1.5955E+04  0.0000E+00
      Residual Covariance
       3.2219E-11  0.0000E+00
       6.8311E-10 -1.4571E-10  1.4028E-04  0.0000E+00
       1.2952E-10  2.3198E-10 -7.8350E-07 -1.5203E-07  1.3332E-04  0.0000E+00                                     
\end{verbatim}
\end{footnotesize}
        
        A partir dos arquivos \en{Z} (\en{.zss}), pode-se calcular a resitividade aparente ($\rho_a$) usando a seguinte equação \cite{z-files}:
        
        \begin{equation}
         \rho_a = \dfrac{T|Z_{ij}|^2}{5}
        \end{equation}

        {\footnotesize \noindent
            \begin{table}[H]
                \begin{tabular*}{1cm}{p{0.05cm}p{0.1cm}p{10cm}}
                    {\footnotesize $T$}          & {\footnotesize $\rightarrow$} & {\footnotesize Período [$s$] }\\
                    {\footnotesize $Z_{ij}$}  & {\footnotesize $\rightarrow$} & {\footnotesize Componente do tensor $Z$ [$\Omega$] }\\
                \end{tabular*}
            \end{table}}
        
        \noindent e para se obter a fase ($\phi$), é utilizada as seguinte relação:
        
        \begin{equation}
         \phi = \dfrac{180}{\pi} \textrm{arctan}\left [\dfrac{\Im(Z_{ij})}{\Re(Z_{ij})} \right ]
        \end{equation}

        {\footnotesize \noindent
            \begin{table}[H]
                \begin{tabular*}{5cm}{p{.9cm}p{0.1cm}p{10cm}}
                    {\footnotesize $\Im(Z_{ij})$}  & {\footnotesize $\rightarrow$} & {\footnotesize Componente imaginária de $Z$ [$\Omega$]}\\
                    {\footnotesize $\Re(Z_{ij})$}  & {\footnotesize $\rightarrow$} & {\footnotesize Componente real de  $Z$ [$\Omega$]}\\
                \end{tabular*}
            \end{table}}
        
        Os erros associados a cada componente podem ser obtidos a partir das matrizes de covariância, assim como os erros para $\rho_a$ e $\phi$.
        
        Os arquivos \en{J}, análogo aos arquivos \en{Z}, armazenam as informações do tensor, porém, a estrutura é mais sintetizada. Os arquivos são estruturados em dois blocos, um destinado as informações da estação MT, tais como: localização, elevação, nome da estação e azimute. O segundo bloco é destinado aos dados \cite{j-format}.
        
        Uma das versões de estrutura base utilizada para dados é a composição de seis sub-blocos. Quatro representam cada uma das componentes do tensor e dois sub-blocos destinados ao \en{Tipper} \cite{padua2004estudos}. O sub-bloco é divido em cinco colunas, onde assumem respectivamente a seguinte ordem: 
        
\begin{footnotesize}        
\begin{verbatim}
            Zij SI units (ohms)                 < componente
            n                                   < número de períodos
                periodo(n)  Real  Imaginario    Erro    Peso
\end{verbatim}
\end{footnotesize}
               
        Os arquivos \en{J} recebem a extensão \en{.dat}, e comumente utilizados para as etapas de inversão, podendo ser convertidos facilmente para o formato EDI. A adoção deste formato para esta etapa, se dá pela fácil leitura dos períodos e por armazenar toda a sondagem em um único arquivo. Ele armazena a mesclagem de todas as diferentes janelas resultantes do processamento EMTF.
        
        Exemplo de arquivo \en{J}:
        
\begin{footnotesize}        
\begin{verbatim}
        >STATION   =bor608b
        >AZIMUTH   =    -23.0000
        >LATITUDE  =    -8.72768
        >LONGITUDE =   -37.84493
        >ELEVATION =    664.0000
        bor608b -23.0
        ZXX SI units (ohms)
        2
            1.1161e-04   -6.6462e-02   -1.0728e-01    1.7715e-03   1
            1.5625e-04    7.7005e-04   -1.0442e-01    2.9007e-03   1
        ZXY SI units (ohms)
        2
            1.1161e-04    2.5648e+00    1.2953e+00    2.5613e-03   1
            1.5625e-04    2.4467e+00    1.4059e+00    5.9294e-03   1
        ZYX SI units (ohms)
        2
            1.1161e-04   -2.3499e+00   -1.1104e+00    1.6657e-03   1
            1.5625e-04   -2.3904e+00   -1.2251e+00    3.0355e-03   1
        ZYY SI units (ohms)
        2
            1.1161e-04    7.1532e-02    3.1711e-02    2.4082e-03   1
            1.5625e-04    5.9528e-02    3.7107e-02    6.2053e-03   1
        TZX
        2
            1.1161e-04   -2.5245e-02    2.0055e-03    5.9072e-04   1
            1.5625e-04   -4.7611e-02   -1.3914e-02    1.0110e-03   1
        TZY
        2
            1.1161e-04    1.3879e-02    1.7406e-02    9.1885e-04   1
            1.5625e-04    5.1640e-03    1.9010e-02    1.4921e-03   1            
\end{verbatim}
\end{footnotesize}
    
    \section{Processamento Robusto -- EMTF}
        \label{sec-robusto}
            
            %O processamento Robusto -- EMTF -- é responsável por obter os tensores impedância através das analises das séries temporais, oriundas das aquisições de campo.
            
            O pacote EMTF \cite{egbert-emtf} desenvolvido por Gary D. Egbert, é um conjunto de programas escrito em \en{Fortran 77} que realizam processamentos tais como: mudança de domínio, cálculo do tensor impedância, plotagem e alguns tipos de conversores de dados. 
            
            O processamento dos dados parte primeiramente da análise espectral das séries temporais, na qual primeiro faz-se necessário a mudança do domínio do tempo para a frequência angular e em seguida a filtragem, remoção de tendências e de dados ruins.
            
            A mudança do domínio do tempo para frequência angular é realizado pelo programa \textbf{Dnff}, contido no pacote. Esse programa realiza a troca do domínio través da FFT (\en{Fast Fourier Transform}) mesclando com processos de \en{Cascade Decimation} \apud{wight1980cascade}{padua2004estudos}. Antes de realizar os processos da transformada de Fourier discreta, são aplicados as series temporais janelas. Limitando as séries temporais, esse processo previne o vazamento da energia da serie durante o processo.
            
            Concluído o processo do \textbf{Dnff}, é iniciado o processo \textbf{TranMT} onde é responsável pelo cálculo do tensor impedância e o \en{Tipper}. Esse processo utiliza de uma forma mais aprimorada da técnica dos mínimos quadrados, incrementando à ela um termo que atribui o desvio de toda a série. Mais detalhes podem serem obtidos em \cite{robusto-egbert}
            
           
    \section{Pacotes de Processamento do Grupo Geoma -- INPE}
    
        O grupo GEOMA do INPE (Institudo Nacional de Pesquisas Espaciais) oferece treinamentos para alunos e colaboradores. O grupo dispõe de um série de \en{scripts} e programas para auxiliar na manipulação do processamento MT \cite{geomadge}.

        Os \en{scripts} oferecidos para o processamento MT foram desenvolvidos, em grande parte, pelo Dr. Marcelo Banik de Pádua e pelo Me. Marcos Banik de Pádua. Tais programas foram idealizados como a primeira interface de uso com os usuários, facilitando a utilização dos programas contidos no MTnet. 
        
        Os programas desenvolvidos pelo GEOMA vão desde conversores de dados até interfaces avançadas de processamento, como inversão geofísica por diversos tipos de programas. Visando a produtividade, pela alta demanda de processamento e a metodologia ja utilizada, os programas foram desenvolvidos para o ambiente Linux rodando em base Debian, e utilizados via linhas de comando. 
        
        O grupo Geoma é a maior referencia no Brasil nos trabalhos com magnetotelúrico, portanto a base de desconstrução do programa usará  as metodologias adotadas pelo grupo, bem como o apoio teórico para o desenvolvimento. 
        
        O programa desenvolvido faz uso dos \en{scripts} do grupo GEOMA, como ``ponte'' entre a interface e o programa EMTF, como também os \en{scripts} para os conversores de dados. Um exemplo de uso é a API \en{processamentoZ} que prepara os dados e extraí os parâmetros necessários para as rotinas \textbf{Dnff} e \textbf{TranMT}. 
        
        Para as plotagens dos dados o grupo usa o programa GMT \cite{gmt}, Figura \ref{plot-cmp-tf}. Embora o programa aqui desenvolvido tenha a sua própria saída gráfica, foi desenvolvida uma extensão que exporta as figuras utilizando o \en{Kernel} do GMT. Essa extensão visa aumentar a familiaridade de usuários já experientes no processamento MT e também ampliar as possibilidades de exportação de imagens. 
        
        Os programas do grupo GEOMA podem ser encontrados em \citeauthor{gitgeoma} (\citeyear{gitgeoma}) e são necessários como requisito para a instalação do programa desenvolvido.
    
    
        \begin{verbatim}
         
         
         
         
         
         
         
         
         
         
        \end{verbatim}

    
        \begin{figure}[H]
            \caption{Saída Gráfica Gerada pelo \en{script}: \en{plot-cmp-tf} \cite{geoma-proc}, utilizando o GMT.}
                \begin{center}
                    \includegraphics[width=15cm]{texto/figura/plot-cmp-tf.png}
                \end{center}
            \legend{\Fonte{\oautor.}}
            \label{plot-cmp-tf}
        \end{figure}

    
    \section{Construtor Gráfico -- Kivy}

        O Kivy é um \en{framework} desenvolvido em Python com trechos escritos em Cython\footnote{Linguagem que permite  a conversão de códigos escritos com a mesma sintaxe do Python para códigos nativos em C \cite{cython}}, utilizado para o desenvolvimento da interface gráfica do programa \cite{kivy}.
        
        O Kivy é um construtor gráfico que utiliza a API OpenGL \cite{opengl} para o processamento dos elementos visuais impressos na tela. Isso significa que todo o processamento dos elementos são executados nativamente no chip gráfico do computador.
        
        A escolha do Kivy também pode ser justificada pela simplicidade de implementação, visto que o mesmo permite o desenvolvimento através de uma linguagem própria, a \en{Kvlang} \cite{kvlang}. Essa linguagem permite a construção dos elementos, através de uma linguagem de marcação e indentada, ou seja, a hierarquia dos elementos estão sempre na identação mais a direita. A linguagem \en{Kvlang} é integrada ao código Python, isso permite o acesso do mesmo elemento nas duas estruturas de código.
