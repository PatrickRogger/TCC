\chapter{Introdução}
    
    O método geofísico magnetotelúrico, utiliza as baixas frequências do espectro eletromagnético, para investigar a subsuperfície do planeta Terra \cite{cagniard1953basic}. A interação do vento solar com o campo magnético terrestre, compõe a origem dessas ondas eletromagnéticas \cite{vozoff1991magnetotelluric}.  

    %\citar{falar sobre o campo magneticio/mt}
    
    A grande complexidade dos dados \mt desestimula o desenvolvimento de \en{softwares} para o processamento dos mesmos. Atualmente os programas destinados a esse tipo de atividade são proprietários, com alto valor comercial, ou são livres operacionais exclusivamente por linhas de comando.  
    
    A comunidade MTnet \cite{mtnet}, mantém laços com diversos pesquisados na área do MT, e reúne as aplicações destinadas aos processamentos, tais como: \en{softwares} de pré-processamento, inversão, tratamento estatísticos, dentre outros. Os programas alocados no MTnet são de uso livre e destinados a comunidade acadêmica.
    
    
    
    %\citar{como esta sendo tratado os dados, citar os trabalhos(marcelo, marcos)}
    
    A proposta deste trabalho visa unir os programas livres em uma única plataforma. Essa será \citar{(ou usa no futuro?)} construída para ser amigável, fazendo uso de uma GUI (\en{Graphical User Interface}\footnote{Interface Gráfica do Utilizador}) simples, agindo como intermediaria entre o usuário e os programas disponíveis no MTnet. 
    
