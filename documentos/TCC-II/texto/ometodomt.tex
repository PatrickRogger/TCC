\chapter{FUNDAMENTOS DO MÉTODO MAGNETOTELÚRICO}

    Proposto com \cite{tikhonov1950determining} e \cite{cagniard1953basic} o método magnetotelúrico usa as fontes passivas\footnote{São fontes de sinal que não dependem de instrumentos artificiais para gerá-la, ou seja, sinais naturais do planeta.} eletromagnéticas do planeta Terra para estudar e mapear a subsuperfície.
    
    %Nesta secção será mostrado sucintamente a origem do sinal MT. Antes de demonstrar as bases teóricas do método, será mostrado brevemente sobre a resistividade elétrica, sendo esse o elemento fundamental para as interpretações lito-geofísicas com base no magnetotelúrico.
        

    %\section{Resistividade Elétrica dos Materiais}
    
        
    %    O método MT usa a resistividade elétrica ($\rho$ [$\Omega m$]) ou o seu inverso, a condutividade elétrica ($\sigma$ [$S$]), para distinguir e estudar a distribuição dos elementos geológicos em subsuperfície. 
        
    %    A resistividade elétrica é um parâmetro físico intrínseco a cada material. Ela relaciona a dificuldade da passagem de corrente (independente do modelo nas dimenções do material estudado.) a cada material, sendo sempre contante\footnote{Para os materiais geológicos a resistividade é representada por um intervalo de valores, devido as complexidades químicas e físicas das diferentes litologias.}.  
        
    %    Georg Ohm em 1827 verificou de forma empírica que ao aplicar uma diferença de potencial ($\Delta U$ [$V$]) em um material, esse exerce uma resistência a passagem e corrente ($i$ [$A$]). Materiais que obedecem a seguinte Lei de Ohm (equação \ref{lei-ohm}) são chamados de materiais ôhmicos.
        
    %    \begin{equation}
    %        \label{lei-ohm}
    %        \Delta U = R i
    %    \end{equation}
        
    %    \citar{Escrever de novo, ta muito ruim kkkkkk}
    
    \section{Origem das Correntes Telúricas}
    
        O método MT utiliza-se de um amplo espectro do campo eletromagnético natural terrestre (10\textsuperscript{-4} a 10\textsuperscript{4} $Hz$) para as sondagens geofísicas. Essa característica permite que  a sondagem magnetotelúrica alcance centenas de quilômetros.
        
        O sinal MT tem sua origem nas ressonâncias de Schumann, nas micropulsações e nas variações diurnas \cite{padua2004estudos}. A figura \ref{sinalmt} mostra a contribuição de cada mecânismo no espectro MT.
        
        \begin{figure}[H]
            \caption{Campo magnético natural e as contribuições das fontes do sinal MT.}
                \begin{center}
                    \includegraphics[width=8.5cm]{texto/figura/espectro-contrinuicao-padua2004.eps}
                \end{center}
            \legend{\Fonte{\cite{padua2004estudos}.}}
            \label{sinalmt}
        \end{figure}
        
        As ressonâncias de Schumann tem sua origem principalmente nas tempestades equatoriais contribuindo para a fonte do sinal MT acima de 1 $Hz$, as frequências a baixo desse valor, tem origem na interação do vento solar com a magnetosfera, que geram ressonâncias Terra-ionosfera. A contribuição de parte do espectro MT, tambem pode ser explicada pela distorção do formato do campo magnético terrestre causada pelo Sol durante o dia, esse processo é chamado de variação diurna e contribui com a faixa de frequência de 10\textsuperscript{-5} a 10\textsuperscript{-4} $Hz$. 
        
        
        %\subsection{Ressonâncias de Schumann}
        
        %\subsection{Micropulsações}
        
        %\subsection{Variações Diurnas}   
        
    \section{Resposta do Método Magnetotelúrico}

        O \mt{} assim como outros métodos geofísicos eletromagnéticos, fundamentam-se nas Leis de Maxwell [\ref{lei-max-rotE} -- \ref{lei-max-divD}], pode-se partir das equações para estimar os parâmetros físicos para a investigação MT, assim:
        
        \begin{equation}
            \label{lei-max-rotE}
            \rot{E} = - \devpart{\vetor{B}}
        \end{equation}
        \begin{equation}
            \label{lei-max-rotH}
            \rot{H} = \vetor{J} + \devpart{\vetor{D}}
        \end{equation}

        \begin{equation}
            \label{lei-max-divB}
            \nabla \cdot \vetor{B} = 0 
        \end{equation}

        \begin{equation}
            \label{lei-max-divD}
            \nabla \cdot \vetor{D} = \varrho                
        \end{equation}
        
        \noindent Onde,
            
        {\footnotesize \noindent
            \begin{table}[H]
                \begin{tabular*}{1cm}{p{0.05cm}p{0.1cm}p{10cm}}
                    {\footnotesize $\vec{\textrm{E}}$}  & {\footnotesize $\rightarrow$} & {\footnotesize Campo Elétrico [$V/m$] }\\
                    {\footnotesize $\vec{\textrm{B}}$}  & {\footnotesize $\rightarrow$} & {\footnotesize Campo Magnético [$T$] }\\
                    {\footnotesize $\vec{\textrm{H}}$}  & {\footnotesize $\rightarrow$} & {\footnotesize Campo Magnetizante [$A/m$]} \\
                    {\footnotesize $\vec{\textrm{J}}$}  & {\footnotesize $\rightarrow$} & {\footnotesize Densidade de Corrente [$A/m^2$]} \\
                    {\footnotesize $\vec{\textrm{D}}$}  & {\footnotesize $\rightarrow$} & {\footnotesize Campo de Deslocamento Elétrico [$C/m^2$]} \\
                    {\footnotesize $\varrho$}           & {\footnotesize $\rightarrow$} & {\footnotesize Densidade de Carga [$C/m^3$]} \\
                    {\footnotesize $t$ }                & {\footnotesize $\rightarrow$} & {\footnotesize Tempo [$s$]}
                \end{tabular*}
            \end{table}}

        Para os estudos \mt s são feitas as seguintes afirmações, que auxiliam e simplificam o desenvolvimento:
       
        \begin{quote}
            A Terra comportasse como um condutor ôhmico e um semi-espaço isotrópico.
        \end{quote}
        
        Podemos utilizar, partindo dessas característica e atrelado a um campo eletromagnético pouco intenso as seguintes relações constitutivas:
        \begin{equation}
         \vetor{B} = \mu \vetor{H}
        \end{equation}
        
        \begin{equation}
         \vetor{D} = \varepsilon \vetor{E}
        \end{equation}
        
        \begin{equation}
         \vetor{J} = \sigma \vetor{E}
        \end{equation}

        {\footnotesize \noindent
            \begin{table}[H]
                \begin{tabular*}{1cm}{p{0.05cm}p{0.1cm}p{10cm}}
                    {\footnotesize $\mu$}          & {\footnotesize $\rightarrow$} & {\footnotesize Permeabilidade Magnética [$H/m$] }\\
                    {\footnotesize $\varepsilon$}  & {\footnotesize $\rightarrow$} & {\footnotesize Permissividade Elétrica [$F/m$] }\\
                    {\footnotesize $\sigma$}       & {\footnotesize $\rightarrow$} & {\footnotesize Condutividade Elétrica [$S/m$]} \\
                \end{tabular*}
            \end{table}}

        \noindent Cada coeficiente das relações constitutivas funcionam como tensores, variantes no tempo, para meios anisotrópicos. Para o estudo abordado e seguindo a afirmação, onde a Terra tornasse um meio isotrópico, isso implica que os tensores, $\mu$ e $\varepsilon$ são estáticos e assumem os seguintes valores:
        
        \begin{quote}
            \begin{enumerate}
                \item[$\mu$] $=$ \basedez{1{,}2566}{-6}$H/m$
                \item[$\varepsilon$] $=$ \basedez{8{,}85}{-12}$F/m$
            \end{enumerate}
        \end{quote}
        
        Utilizando as equações constitutivas podemos reescrever as equações \ref{lei-max-rotE} e \ref{lei-max-rotH}:
        
        {\setlength\arraycolsep{2pt}
        \begin{eqnarray}
            \label{rotEconstH}
            \rot{E} &=& - \devpart{\vetor{B}}; \qquad \vetor{B} = \mu \vetor{H} \nonumber \\
            \rot{E} &=& - \mu \devpart{\vetor{H}}
        \end{eqnarray}}

        {\setlength\arraycolsep{2pt}
        \begin{eqnarray}
            \label{rotHconstE}
            \rot{H} &=& \vetor{J} + \devpart{\vetor{D}}; \qquad \vetor{J} = \sigma \vetor{E} \quad \textrm{e} \quad \vetor{D} = \varepsilon \vetor{E}\nonumber \\
            & &\rot{H} =  \sigma \vetor{E} + \varepsilon \devpart{\vetor{E}}
        \end{eqnarray}}
        
        Na faixa da sondagem MT a Terra comporta-se como um condutor ôhmico, isso implica que o meio não possuem cargas livres, logo $\varrho \simeq 0$. 
        
        Para os campos pode ser assumida uma dependência temporal harmônica dada por $e^{- \imath \omega t}$, que pode ser decomposta em vários harmônicos pela transformada de Fourier, onde $t$ é o tempo e $\omega$ a frequência angular. 
    
        Portando as equações: \ref{rotEconstH}, \ref{rotHconstE}, \ref{lei-max-divB} e \ref{lei-max-divD}, podem ser reescritas como:
        
        \begin{equation}
            \label{rotEconstHreescrito}
            \rot{E} = \imath \omega \mu \vetor{H}           
        \end{equation}
        
        \begin{equation}
            \label{rotHconstEreescrito}
            \rot{H} = (\imath \omega \varepsilon + \sigma) \vetor{E}
        \end{equation}
        
        \begin{equation}
            \nabla \cdot \vetor{H}  = 0
        \end{equation}

        \begin{equation}
            \nabla \cdot \vetor{E} = 0
        \end{equation}

        Aplicando o rotacional na equação \ref{rotHconstEreescrito}, obtemos:
         
        \begin{equation}
            \label{rotrotH}
            \nabla \times \nabla \times \vetor{H} = (\imath \omega \varepsilon + \sigma) \rot{E}
        \end{equation}

        Comparando a equação \ref{rotrotH} com a equação \ref{rotEconstHreescrito}, pode-se reescreve-la como:
        
        {\setlength\arraycolsep{2pt}
        \begin{eqnarray}
            \label{preHk}
            \nabla \times \nabla \times \vetor{H} &=& (\imath \omega \varepsilon + \sigma) \rot{E}; \qquad \rot{E} = \imath \omega \mu \vetor{H} \nonumber \\
            & &\nabla \times \nabla \times \vetor{H} = \imath \omega \mu (\imath \omega \varepsilon + \sigma) \vetor{H}            
        \end{eqnarray}}
        
        Pode-se expressar a equação \ref{preHk} usando a seguinte identidade vetorial:
        
        \begin{equation}
            \nabla \times \nabla \times \vetor{A} = \nabla \nabla \cdot \vetor{A} - \nabla^2 \vetor{A} 
        \end{equation}

        \noindent Portanto:
        
        {\setlength\arraycolsep{2pt}
        \begin{eqnarray}
            \label{difH}
            \nabla \nabla \cdot \vetor{H} - \nabla^2 \vetor{H} & = & \imath \omega \mu (\imath \omega \varepsilon + \sigma) \vetor{H} \nonumber \\
            \nabla (\nabla \cdot \vetor{H}) - \nabla^2 \vetor{H} & = & \vetor{H}[\cancelto{\kappa^2}{\imath \omega \mu (\imath \omega \varepsilon + \sigma)}] \nonumber \\
            \nabla (\cancelto{0}{\nabla \cdot \vetor{H}}) - \nabla^2 \vetor{H} & = & \kappa^2 \vetor{H} \nonumber \\
            \nabla^2 \vetor{H} + \kappa^2 \vetor{H} & = &  0; \qquad \kappa^2 = \imath \omega \mu (\imath \omega \varepsilon + \sigma)
        \end{eqnarray}}        
        
        Considerando um condutor ôhmico ($\sigma \gg \omega \varepsilon$), assim:
        \begin{equation}
            \label{kappaquad}
            \kappa^2 = \imath \omega \mu \sigma
        \end{equation}
        
        \noindent Onde, $\kappa^2$ é o módulo do vetor de onda ($\vetor{k}$).
        
        A equação \ref{kappaquad} pode ser expressa seguindo a definição, como:
        
        {\setlength\arraycolsep{2pt}
        \begin{eqnarray}
            \kappa & = & \sqrt{\imath \omega \mu \sigma}; \quad \imath = e^{\imath \frac{\pi}{2}} \nonumber \\
            \kappa & = & \sqrt{\omega \mu \sigma} \sqrt{e^{\imath \frac{\pi}{2}}} \nonumber \\
            \kappa & = & \sqrt{\omega \mu \sigma} e^{\imath \frac{\pi}{4}}; \quad e^{\imath \frac{\pi}{4}} = \sqrt{1/2} (1 + \imath) \nonumber \\
            \kappa & = & \sqrt{\dfrac{\omega \mu \sigma}{2}} (1 + \imath) \nonumber \\
            \kappa & = & \dfrac{(1 + \imath)}{\delta}
        \end{eqnarray}} 
        
        \noindent Onde,
        
        \begin{equation}
            \label{shin-depth}
            \delta_\omega = \sqrt{\dfrac{2 \rho}{\omega \mu}} \longrightarrow \delta_f \approx 500  \sqrt{\frac{\rho_a}{f}}
        \end{equation}
        
        A equação \ref{shin-depth} é chamada de \en{skin-depth}\footnote{Espessura pelicular}, ela representa a profundidade de penetração da onda eletromagnética em um meio condutor.
        A partir da equação são mapeadas as litologias em subsuperfície, relacionando-as com a $\rho_a$.        

        O meio geológico influência diretamente a profundidade de investigação, a figura \ref{fig-skin-depth} mostra que para uma mesma frequência, ela pode representar valores diferentes de profundidade, variando o meio em subsuperfície, isso é representado por $\rho$. Os meios mais resistivos geram profundidade maiores, já meios condutivos diminuem a profundidade. Esse fenômeno é importante por que, ao interpretar as seções lito-geofísicas, é comum estudar contextos de bacias sedimentares (meio condutivo) em contados com contextos cristalinos (meio resistivo), a atenção deve-se voltar para o fato de que um mesmo período em função de $\rho$ pode representar duas profundidades diferentes, estando a estação em cima do contexto sedimentar ou em cima do contexto cristalino. 
        
        \begin{figure}[H]
            \caption[Gráfico do \en{skin-depth}]{Gráfico do \en{skin-depth} em função da frequência [$Hz$], variando a resistividade do meio}
            \begin{center}
                \includegraphics[width=12cm]{texto/figura/skin-depth.png}
            \end{center}
            \legend{\Fonte{\oautor.}}
            \label{fig-skin-depth}
        \end{figure}
        
    \section{Impedância Eletromagnética}
        
        Baseado na fundamentação teórica apresentada a seção anterior, o MT busca obter a resistividade aparente em função da profundidade. 
        
        A solução da equação \ref{difH} e da sua análoga para o campo $\vetor{E}$ são dadas por:
        
        \begin{equation}
            \label{soluH}
            \vetor{H}_{(\vetor{r})} = \vetor{H} e^{-\vetor{k} \cdot \vetor{r}}
        \end{equation}

        \begin{equation}
            \label{soluE}
            \vetor{E}_{(\vetor{r})} = \vetor{E} e^{-\vetor{k} \cdot \vetor{r}}
        \end{equation}

        %\noindent Onde $\vetor{k}$ é o vetor de onda, cujo o módulo é definido por $\kappa$ na equação \ref{kappaquad}.
        
        Substituindo a equação \ref{soluH} e \ref{soluE} em \ref{rotHconstEreescrito}, temos:
        
        {\setlength\arraycolsep{2pt}
        \begin{eqnarray}
            \label{HEpossolu}
            \nabla \times \vetor{H}e^{-\vetor{k} \cdot \vetor{r}} & = & (\imath \omega \varepsilon + \sigma) \vetor{E}e^{-\vetor{k} \cdot \vetor{r}}; \quad \sigma \gg \imath \omega \varepsilon \ \nonumber \\
            \nabla \times \vetor{H}e^{-\vetor{k} \cdot \vetor{r}} & = & \sigma \vetor{E}e^{-\vetor{k} \cdot \vetor{r}}; \quad \sigma = \dfrac{\kappa^2}{\imath \omega \mu} \nonumber \\
            \nabla \times \vetor{H}e^{-\vetor{k} \cdot \vetor{r}} & = &\dfrac{\kappa^2}{\imath \omega \mu} \vetor{E}e^{-\vetor{k} \cdot \vetor{r}}
        \end{eqnarray}} 
        
        \noindent Usando as identidades:
        
        \begin{equation}
            \nabla (e^{-\vetor{k} \cdot \vetor{r}}) = - e^{-\vetor{k} \cdot \vetor{r}} \vetor{k}
        \end{equation}

        \begin{equation}
            \nabla \times \vetor{C}(f_{(\vetor{r})}) = - \vetor{C} \times \nabla f_{(\vetor{r})} 
        \end{equation}
        
        Pode-se reescreve a equação \ref{HEpossolu}:
        
        {\setlength\arraycolsep{2pt}
        \begin{eqnarray}
            \label{EZHvetor}
            - \vetor{H} \times (- e^{-\vetor{k} \cdot \vetor{r}} \vetor{k}) = \dfrac{\kappa^2}{\imath \omega \mu} \vetor{E}e^{-\vetor{k} \cdot \vetor{r}} \nonumber \\
            \cancelto{}{e^{-\vetor{k} \cdot \vetor{r}}} (\vetor{H} \times \vetor{k}) = \cancelto{}{e^{-\vetor{k} \cdot \vetor{r}}} \dfrac{\kappa^2}{\imath \omega \mu} \vetor{E}\nonumber \\
            \vetor{E} = \dfrac{\imath \omega \mu}{\kappa^2} \vetor{H} \times \vetor{k} \nonumber \\
            \vetor{E} = \dfrac{\imath \omega \mu}{\kappa} \vetor{H} \times \dfrac{\vetor{k}}{\kappa}
        \end{eqnarray}} 
        
        \noindent A relação $\vetor{k}/\kappa$ é o versor de $\vetor{k}$ ou  $\hat{\unitario{k}}$, representando a ortogonalidade entre $\vetor{H}$ e $\vetor{E}$.
        
        A partir da equação anterior pode ser definido que $Z = \imath \omega \mu / \kappa$, esta definição é conhecida como impedância intrínseca do meio ou impedância eletromagnética, também pode ser representada da seguinte forma:
        
        \begin{equation}
            Z = \dfrac{|\vetor{E}|}{|\vetor{H}|} = \dfrac{\imath \omega \mu}{\kappa} = \sqrt{\omega \mu \rho} e^{\imath \frac{\pi}{4}}
        \end{equation}
        
        A impedância eletromagnética ($Z$) pode ser decomposta em função das componentes de $\vetor{E}$ e $\vetor{H}$, representada na forma matricial:
        
        \begin{equation}
            \label{tensor-impe}
            \left (\begin{array}{c}
                \textrm{E}_x\\
                \textrm{E}_y
                    \end{array}\right)
                =
            \left (\begin{array}{cc}
                \textrm{Z}_{xx} & \textrm{Z}_{xy}\\
                \textrm{Z}_{yx} & \textrm{Z}_{yy}
                    \end{array}\right)
            \left (\begin{array}{c}
                \textrm{H}_x\\
                \textrm{H}_y
                    \end{array}\right)
	    \end{equation}
	    
	    O método MT, então, obtém a resistividade aparente a partir da impedância eletromagnética, e atribui a ela uma profundidade, onde pode ser definida pela função de \en{skin-depth} (equação \ref{shin-depth}).          
