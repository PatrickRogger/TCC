\chapter{Código Fonte PampaMT}
    
    \vspace*{2cm} 
    
    O programa PampaMT esta armazenado no servidor GitHub, atualmente é a maior comunidade de códigos fontes, nela podemos encontrar o \en{Kernel} Linux, a plataforma SU (\en{Seismic Unix}), o pacote abn\TeX2, dentre um vasto catálogo de outros projetos.   
    
    \vspace*{1cm}
    
    \begin{figure*}[h]
        \begin{center}
            \includegraphics[width=5cm]{texto/figura/qr-code-pampamt.eps}
        \end{center}
    \end{figure*}
    
    \begin{center}
        \url{https://github.com/PatrickRogger/PampaMT}
    \end{center}

    \vfill
