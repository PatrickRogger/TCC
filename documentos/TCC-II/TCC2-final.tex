\documentclass[12pt,twoside,oneright,a4paper,chapter=TITLE,english,brazil]{unipampa}
\usepackage[utf8]{inputenc}                 % O Latex deve ser escrito em utf-8, pois utiliza o pacote abntex2
\usepackage[alf,abnt-emphasize=bf]{abntex2cite}               % Pacote para utilizar a citação da abnt (Nome,ANO)
\usepackage{times}                          % Usa a fonte padrao Times new roman
%\usepackage{pdfpages}                      % Usar quando for adicionar a folha de aprovação assinada
\usepackage{graphicx}                       % Usar para incluir figuras
\usepackage{float}                          % Permite travar as figuras usando o parametro [H]
\usepackage{makeidx}\makeindex              % Usar para criar indexação no texto
\usepackage{xcolor}
\usepackage{amsmath}
\usepackage{cancel}
\usepackage{totcount}
\usepackage{multicol}
\usepackage{pdflscape}
\usepackage{lscape}
% ===============================================================================================================


% MEUS COMANDOS
% =============================================================
\newcommand{\en}[1]{\textit{#1}}

\newcommand{\mt}{magnetotelúrico}
\newcommand{\citar}[1]{\textcolor{red}{#1}}

\newcommand{\Python}{\en{Python}}
\newcommand{\software}{\en{software}}
\newcommand{\Software}{\en{Software}}
\newcommand{\softwares}{\en{softwares}}
\newcommand{\Shell}{\en{Shell}}
% =============================================================

% =============================================================
\newcommand{\rot}[1]{\nabla \times \vec{\textrm{#1}}}
\newcommand{\devpart}[1]{\dfrac{\partial #1}{\partial t}}
\newcommand{\vetor}[1]{\vec{\textrm{#1}}}
\newcommand{\unitario}[1]{\textrm{#1}}
\newcommand{\basedez}[2]{$#1 \times 10^{#2}$}
\newcommand{\campo}[1]{\textrm{#1}}
% =============================================================

% =============================================================
\hyphenation{ASCII}
\hyphenation{ProcessingZ}
% =============================================================



% =============================================================
% Novos contadores - FLUXOGRAMAS

%\newcounter{fluxo}
%\newenvironment{fluxograma}{%
%\addtocounter{fluxo}{1}%
%\newcommand{\titulofluxo}[1]{Fluxograma \thefluxo -- ##1}
%}{}



% Create new file
\newwrite\flu
\immediate\openout\flu=\jobname.flu
\newcommand{\flux}[1]{\write\flu{#1}}



% Configuração do documento (autor, titulo, orientação ....)

% Autor, para varios autores use: \and entre eles
\autortrabalho{Garcia}{Patrick Rogger}           
\autor{Patrick Rogger Garcia}

% Titulo do trabalho
\titulo{Desenvolvimento de Software livre para processamentos de dados magnetotelúricos}                

% Orientador e Coorientador
\orientador[Orientador]{Vinicius de Abreu Oliveira}
\coorientador[Co-orientadora]{Andréa Cristina Lima dos Santos Matos}

% Curso
\curso{geofísica}                           

% LOCAL
\local{Caçapava do Sul}
\data{\the\year}            % Pode ser passado um ano especifico. Ex. \data{2018}

% Tipo do Trabalho
\tipotrabalho{Trabalho de Conclusão de Curso (Graduação)}

% Preambulo do Trabalho 
\preambulo{\imprimirtipotrabalho{} apresentado ao curso
            de Bacharelado em \imprimircurso{} da Universidade
            Federal do Pampa como requisito
            parcial para obtenção do grau de Bacharel em \imprimircurso.}

% Codigo cutter            
\cutter{G216d} % Se não tiver, deixe em branco

% Area de concentração
\areadeconcentracao{Geofísica Espacial, Geofísica de \en{Software}}

% Texto da defesa
\defesa{Trabalho de Conclusão de Curso defendido e aprovado em: 30 de novembro de 2018.}

% Membros da Banca
% ======================================================================================
\titulacaoorientador{Prof. Dr.}             % Titulação do orientador
\membroA{Prof. Dr.}{Éverton Frigo}{UNIPAMPA}  % Utilize o formato \membroA{Titulo}{NOME}{INSTITUIÇÃO}
\membroB{Dr.}{Marcelo Banik de Pádua}{INPE}   %
% ======================================================================================

% Palavras Chaves (Portugues): A-J
\chaveA{Magnetotelúrico}
\chaveB{Python3}
\chaveC{Software Livre}
\chaveD{Processamento Robusto}
%=================================

% Palavras Chaves (Ingles): A-J
\keywordA{Magnetotelluric}
\keywordB{Python3}
\keywordC{Free Software}
\keywordD{Robust Processing}
%=================================


% INICIO DO DOCUMENTOS
\begin{document}

\imprimircapa                   % Imprime a Capa

\imprimirfolhaderosto*          % Imprime a Folha de Rosto, 
                                % Para gerar o pdf sem a ficha Catalografica no verso, retire o '*'

\imprimirfichacatalografica     % Imprime a Ficha Catalografica

\imprimirfolhadeaprovacao       % Imprime a Folha de Aprovação
%\includepdf{folhadeaprovacao_digitalizada.pdf}    % Use para adicionar a versão digitalizada com as assinaturas


% DEDICATORIA 
% ===================================
\begin{dedicatoria}
    \textit{Dedico este trabalho, bem como todas as demais conquistas da minha vida, a minha esposa -- Fernanda --, minha mãe -- Cláudia --, minha irmã -- Thairine -- e minha avó -- Suzana --.} 
\end{dedicatoria}
% ===================================


% AGRADECIMENTOS
% ===================================
\begin{agradecimentos}
    %Agradeço primeiramente a minha irmã -- Thairine Garcia -- por este trabalho, pois sem ela os últimos quatro anos não seriam possíveis. Também a minha mãe, Claudia Valéria, pelo exemplo e pelo amor que me deu. 
    Gostaria de agradecer antes de tudo a minha irmã -- Thairine Garcia -- por tornar possível os últimos 4 anos. A minha mãe Cláudia Valéria, pelo exemplo e amor que me deu.
    
    \SingleSpacing
    \OnehalfSpacing
    
    \noindent Obrigado minha amada e mais do que tudo minha grande amiga -- Fernanda Melo -- por todas as horas ao meu lado, por todo o apoio e por todos os momentos desde a chuva de meteoros de novembro de 2014.
    
    %Á minha amada e mais do que tudo grande amiga -- Fernanda Melo -- por todas as horas ao meu lado. Pelo apoio em todos os momentos desde a chuva de meteoros de novembro de 2014.
    
    \SingleSpacing
    \OnehalfSpacing
    
    \noindent A meu grande amigo, Paulo Roberto, que conquistei ao longo destes 4 anos. Agradeço por todas as horas de conversas, principalmente sobre furos de sondagem e prospecção de ouro.  
    
    \SingleSpacing
    \OnehalfSpacing
    
    %\noindent Agradeço meus professores pelos conhecimentos passados, principalmente ao professor Éverton Frigo, que me apresentou ao mundo da programação e pelos seu carinho ao curso da geofísica, onde sempre esteve pronto a nos ajudar.
    
    \noindent Aos meus professores pelos conhecimentos passados, principalmente ao professor Éverton Frigo, que me apresentou o mundo da programação e pelo seu carinho ao curso da geofísica, onde sempre esteve pronto para nos ajudar. 
    
    \SingleSpacing
    \OnehalfSpacing
    
    %\noindent Obrigado a professora Andréa Matos por me apresentar o Magnetotelúrico e a todo o Grupo GEOMA do INPE. Em especial agradeço ao Marcelo de Pádua que concedeu os programas e todo o auxilio para o desenvolvimento do PampaMT.   
    
    \noindent Também gostaria de agradecer a minha coorientadora, Andréa Matos por me apresentar o Magnetotelúrico e a todo o Grupo GEOMA do INPE. Em especial agradeço a Marcelo Banik e a Marcos Banik que concederam os programas e todo o auxilio para o desenvolvimento do PampaMT.
    
    \SingleSpacing
    \OnehalfSpacing

    \noindent Ao meu orientador e amigo, Vinicius Oliveira, agradeço por sempre me ajudar e pelas ótimas sugestões nesse último ano.
    
    \SingleSpacing
    \OnehalfSpacing
    
    \noindent Á Lenon Ilha, Matheus Cruz e Welynton Ramos, agradeço pela amizade e a disposição para os testes alfa do PampaMT.
    
    \SingleSpacing
    \OnehalfSpacing
    
    \noindent A todos meus amigos, Fátima Nunes, Sr. Ney, a Banda Municipal Dr. Cyro Carlos de Melo, Godinho (Filipi), meu irmão (Vinicius Geriolli) e meu pai (Rogério Santos) que direta ou indiretamente me ajudaram nos 4 anos da minha graduação. 
   
    \SingleSpacing
    \OnehalfSpacing
   
    \noindent E a minha avó -- Suzana Simões -- pelo seu amor e pelas horas de sabedoria ao telefone.
    
\end{agradecimentos}
% ===================================


% EPÍGRAFE
% ===================================
\begin{epigrafe}
    \textit{``Most good programmers do programming not because they expect to get paid or get adulation by the public, but because it is fun to program''.
    \DoubleSpacing \\
    -- (Linus Torvalds)}
\end{epigrafe}
% ===================================


% RESUMO (PORTUGUES)
% ===================================
\begin{resumo}
 %Este trabalho trata do desenvolvimento um Software para o processamento de dados magnetotelúrico (MT), esse projeto foi idealizado visto a carência de programas para o trabalho com estes tipos de dados. O programa desenvolvido realiza as etapas de processamento dos dados desde a coleta, realizada pelos equipamentos do tipo Metronix ADU06 ou ADU07, até as etapas de visualização, passando por processamentos estatísticos, conversão de formatos de arquivos e processamento robusto. Todo o programa foi desenvolvido utilizando a linguagem de programação Python sob a licença de software Livre, o programa une inúmeros \en{scripts} e rotinas consagrados no processamento de dados magnetotelúrico, onde esses processamentos serão feitos através de uma interface gráfica. Facilitando, assim, as etapas de processamentos para os novos usuários. O que se contra põe aos \en{scripts} e rotinas disponíveis atualmente para o processamento de dados que utilizam apenas linhas de comando e procedimentos excessivamente manuais. Tais fatores, muitas vezes, afasta novos pesquisadores o que restringe este processamento a pequenos núcleos de pesquisadores. O intuito final do trabalho é tornar o processamento MT mais dinâmico através deste novo programa.
 Este trabalho trata do desenvolvimento do \Software{} PampaMT. Este programa foi desenvolvido para auxiliar e otimizar o pré-processamento de dados do método geofísico Magnetotelúrico (MT). O método MT utiliza fontes eletromagnéticas naturais do planeta Terra para investigar a distribuição da condutividade elétrica em subsuperfície. A faixa de frequência utilizada pelo MT compreende de 10\textsuperscript{-4} Hz a 10\textsuperscript{4} Hz, possibilitando a investigação geofísica de 100 a 200 quilômetros de profundidade. O programa foi desenvolvido utilizando a linguagem de programação Python sob a licença de software Livre, o programa une inúmeros programas e rotinas, de uso livre, consagrados no pré-processamento de dados magnetotelúrico. A utilização do programa será realizada através de uma interface gráfica amigável (GUI), facilitando assim, as etapas de pré-processamentos para novos usuários. O que se contra põe aos programas e rotinas disponíveis atualmente para o processamento de dados que utilizam apenas linhas de comando e procedimentos excessivamente manuais. Tais fatores, muitas vezes torna-se exaustivo o treinamento de novos usuários e adiam a produção de resultados trabalhando com dados MT. A validação da eficiência do programa, em termos do tempo e usabilidade, foi realizada através do processamento simultâneo de dados reais, localizados na região nordeste do Brasil, tais dados, já foram processados por trabalhos anteriores, e permitem a comparação efetiva do processamento utilizando as técnicas já consolidadas e a nova forma de processamento através do PampaMT.  
\end{resumo}
% ===================================


% RESUMO (INGLES)
% ===================================
\begin{resumoingles}
%This work deals with the development of PampaMT software. This program was developed to aid and optimize the pre-processing of the Magnetotelluric (MT) geophysical method. The MT method uses natural electromagnetic sources from planet Earth to investigate the subsurface of the planet. The frequency range used by the MT comprises from 10\textsuperscript{-4} Hz a 10\textsuperscript{4} Hz, allowing the geophysical investigation of 100 to 200 km depth. The program was developed using the Python programming language under the Free software license, the program joins numerous programs and routines, free use, consecrated in the pre-processing of Magnetotelluric data. The use of the program will be done through a friendly graphical interface, thus facilitating the pre-processing steps for new users. What is against the programs and routines currently available for data processing that use only command lines and procedures excessively manual. Such factors often alienate new researchers which constrains this processing to small nuclei of researchers. The validation of the program's efficiency in terms of time and usability was performed through the simultaneous processing of real data, located in the northeastern region of Brazil. These data have already been processed by previous works and allow for the effective comparison of the processing using the techniques already consolidated and the new form of processing through PampaMT.
This work deals with the development of PampaMT Software. This program was developed to aid and optimize the pre-processing of the Magnetotelluric (MT) geophysical method. The MT method uses natural electromagnetic sources from planet Earth to investigate the distribution of electrical conductivity in subsurface. The frequency range used by the MT comprises 10 \textsuperscript{-4} Hz to 10 \textsuperscript{4} Hz, allowing the geophysical investigation of 100 to 200 kilometers deep. The program was developed using the Python programming language under the Free software license, the program joins numerous programs and routines, free use, consecrated in the pre-processing of magnetotelluric data. The use of the program will be performed through a graphical user interface (GUI), thus facilitating the pre-processing steps for new users. What is against the programs and routines currently available for data processing that use only command lines and procedures excessively manual. Such factors often become exhausting training for new users and delay the production of results working with MT data. The validation of the program's efficiency in terms of time and usability was performed through the simultaneous processing of real data, located in the northeastern region of Brazil, which has already been processed by previous works and allows the effective comparison of the processing using the techniques already in place and the new form of processing through PampaMT.
\end{resumoingles}
% ===================================


% LISTA DE FIGURAS E TABELAS
% ================================================
\listoffigures      % Imprime a lista de figuras
%\listoftables       % Imprime a lista de tabelas
% ================================================


% LISTA DE SIGLAS E ABREVIATURAS
% ==============================================================
% Utilize o formato:
%   \item[SIGLA --]         Nome da sigla
\begin{siglas}
    \sigla{ASCII}{  \en{American Standard Code for Information Interchange}}
    \siglae{            -- Código Padrão Norte-americano para Intercâmbio de Informações -- (Tradução Nossa)}
    \\
    \sigla{API}{    \en{Application Programming Interface}}
    \siglae{            -- Interface de Programação de Aplicações -- (Tradução Nossa)}
    \\
    \sigla{CNPq}{   Conselho Nacional de Desenvolvimento Científico e Tecnológico}
    \\
    \sigla{EMTF}{   \en{Electromagnetic Transfer Function}}
    \siglae{            -- Função de Transferência Eletromagnética -- (Tradução Nossa)}
    \\
    \sigla{FFT}{    \en{Fast Fourier Transform}}
    \siglae{            -- Transformada Rápida de Fourier -- (Tradução Nossa)}
    \\
    \sigla{GEOMA}{  Grupo de Geomagnetismo}
    \\
    \sigla{GMT}{    \en{Generic Mapping Tools}}
    \siglae{            -- Ferramentas Genéricas para Mapeamento -- (Tradução Nossa)}
    \\
    \sigla{GUI}{    \en{Graphical User Interface}}
    \siglae{            -- Interface Gráfica do Utilizador -- (Tradução Nossa)}
    \\
    \sigla{INPE}{   Instituto Nacional de Pesquisa Espaciais}
    \\
    \sigla{MT}{     Magnetotelúrico}
    \\
    \sigla{PIBIC}{  Programa Institucional de Bolsas de Iniciação Científica}
    \\
    \sigla{SIG}{    Sistema de Informação Geográfica}
    \\
    \sigla{TS}{     \en{Time Series}}
    \siglae{            -- Séries Temporais -- (Tradução Nossa)}
\end{siglas}
% ==============================================================


% LISTA DE SÍMBOLOS
% ==============================================================
% Utilize o formato:
%   \item[Simbolo]          Nome do símbolo
\begin{simbolos}
    \begin{multicols}{2}
    \item[$\delta$]                 \en{Skin-depth}
    \item[$\varepsilon$]             Permissividade Elétrica
    \item[$\mu$]                     Permeabilidade Magnética
    \item[$\kappa$]                  Modulo do Vetor de Onda
    \item[$\rho$]                    Resistividade Elétrica
    \item[$\rho_a$]                  Resistividade Elétrica Aparente
    \item[$\varrho$]                 Densidade de Carga
    \item[$\sigma$]                  Condutividade Elétrica
    \item[$\phi$]                    Fase de Onda
    \item[$\omega$]                  Frequência Angular
    \item[$A$]                       Área
    \item[$\vetor{B}$]               Vetor Campo Magnético
    \item[$\vetor{D}$]               Vetor Campo de Deslocamento Elétrico
    \item[$\vetor{E}$]               Vetor Campo Elétrico
    \item[$f$]                       Frequência
    \item[$\vetor{H}$]               Vetor Campo Magnetizante
    \item[$i$]                       Corrente Elétrica
    \item[$\vetor{J}$]               Vetor Densidade de Corrente
    \item[$\vetor{k}$]               Vetor de Onda
    \item[$R$]                       Resistência Elétrica
    \item[$t$]                       Tempo
    \item[$T$]                       Período
    \item[$Z$]                       Impedância Eletromagnética  
    \item[$\imath$]                  Unidade Imaginária
    \item[$\Delta l$]                Comprimento
    \item[$\Delta U$]                Diferença de Potencial
    \item[$\nabla$]                  Gradiente
    \item[$\nabla \times$]           Rotacional
    \item[$\nabla \cdot$]            Divergente
    \item[$\nabla^2$]                Laplaciano
    \item[$\Im$]                     Componente Imaginária 
    \item[$\Re$]                     Componente Real
    \end{multicols}                                                   
\end{simbolos}
% ==============================================================


% SUMÁRIO
% ==============================================================
\tableofcontents       % Imprime o sumario
% ==============================================================


% CORPO DO TEXTO
% =============================================================================
%
% Hierarquia:
% \chapter{NOME DO CAPITULO}                 --- secção primária   | 1.
%    \section{Nome da secção}                --- secção secundária | 1.1
%       \subsection{Nome da Secção primeria} --- secção terciária  | 1.1.1
%          \subsubsection{Nome}              --- secção quaternária| 1.1.1.1  
%              \subsubsubsection{Nome}       --- secção quinária   | 1.1.1.1.1
%
\textual                    % Inicia os elementos Textuais
\pagestyle{simple}          % Retira a linha e titulo no verso das paginas
\OnehalfSpacing             % Ajusta o texto para 1.5 entre linhas
% =============================================================================

\setcounter{page}{14}
\chapter{INTRODUÇÃO}
   
    %A interação do vento solar com o campo magnético terrestre, induzem correntes elétricas que propagam e penetram no interior da Terra \cite{cagniard1953basic}. Essas ondas são chamadas de correntes telúricas, as correntes telúricas quando interagem com as diversas litologias, induzem correntes eletromagnéticas. O método geofísico magnetotelúrico usa essas correntes para investigar e mapear o interior da Terra \cite{vozoff1991magnetotelluric}.  

    %O método geofísico magnetotelúrico, utiliza as baixas frequências do espectro eletromagnético, para investigar a subsuperfície do planeta Terra \citar{cagniard1953basic}. A interação do vento solar com o campo magnético terrestre, compõe a origem dessas ondas eletromagnéticas vozoff1991magnetotelluric.  

    %\citar{falar sobre o campo magneticio/mt}
    
    %A grande complexidade dos dados \mt desestimula o desenvolvimento de \en{softwares}. Atualmente os programas destinados a esse tipo de atividade são proprietários, com alto valor comercial, ou são livres operacionais exclusivamente por linhas de comando.  
    
    O método geofísico magnetotelúrico (MT) surge da interação de eventos naturais de origem espacial com a Terra. No caso, a interação do vento solar com a magnetosfera terrestre, e as altas ocorrências de tempestades equatoriais geram ondas eletromagnéticas, que por sua vez penetram e propagam-se por todo o interior da Terra. A indução gerada pela propagação das correntes (correntes telúricas) formam o sinal analítico do método geofísico magnetotelúrico -- MT \cite{parkinson93}. As aplicações do método giram em torno da prospecção de petróleo e estudos crustais. Essas aplicações justificam-se pela profundidade de investigação do método, que varia de 100 metros a 200 quilômetros.

    %No entanto, devido a complexidade do método, tem-se um desestimulo ao desenvolvimento de \softwares, o que produz um ambiente onde programas proprietários agregam um alto valor comercial, muitas vezes inviabilizando o uso acadêmico. E ainda, os programas de uso livre não são amigáveis com o usuário. De fato, o seu uso se dá, unicamente, por linhas de comando que não é familiar para a maioria dos potencias usuários do MT.

    No entanto, os programas desenvolvidos para as aplicações do MT e para o processamento do mesmo, agregam um alto valor comercial devido aos poucos núcleos de desenvolvimento. Já as aplicações desenvolvidas pelo meio acadêmico são em sua maioria manipuladas via linhas de comandos. Tais fatores criam um ambiente, onde novos usuários que desejam trabalhar com tal método, tenham que desprender horas de estudos adicionais de programação para que possam produzir resultados.       
    
    Para o usuário que deseja trabalhar com os programas livres e de uso acadêmico, pode encontrar na comunidade MTnet \cite{mtnet}, uma vasta biblioteca de programas e pesquisadores na área do MT, a comunidade reúne as aplicações destinadas aos processamentos de dados, tais como: \en{softwares} de pré-processamento, inversão, tratamento estatísticos, dentre outros. Os programas alocados no MTnet são de uso livre, como comentado anteriormente, e são destinados exclusivamente a comunidade acadêmica. Portando a proposta deste trabalho visa unir os programas livres alocados na comunidade em uma única plataforma gráfica, fazendo uso de uma GUI (\en{Graphical User Interface}) simples, agindo como intermediária entre o usuário e os programas já desenvolvidos por toda a comunidade.
    
    No Brasil, a metodologia adotada pelos grupos de pesquisas na área do MT, faz uso principalmente do programa EMTF para a obtenção das matrizes de impedância eletromagnética (seção \ref{sec-impedancia}), tal elemento é fundamental para o processamento de dados MT. O programa desenvolvido no âmbito do trabalho seguirá a metodologia adotada pelos grupos de pesquisas do Brasil e manterá todo o arcabolso de processamento adotados por eles.
    
    %Este trabalho foi idealizado com o objetivo de melhorar a interação com o usuário através do desenvolvimento de um software livre para tratar e manipular dados MT. A abordagem adotada no trabalho será a criação de uma GUI (Graphical Interface User) escrita e desenvolvida totalmente com programas livres. A GUI servirá como intermédio entre o usuário e os programas livres, já consagrados no trato com esse tipo de dado (EGBERT,1997a). Compreende junto à interface os seguintes objetivos: desenvolver novos algoritmos escritos em Python e otimizar códigos já existentes frente as novas tecnologias.

    
    %\citar{como esta sendo tratado os dados, citar os trabalhos(marcelo, marcos)}
    
     
    

\chapter{OBJETIVOS}

	\section{Objetivo Geral}
        O objetivo deste trabalho é desenvolver um \en{software} livre, visando integrar e facilitar o processamento de dados MT.
    
    \section{Objetivos Específicos}
        
        Os objetivos específicos compreendem os seguintes itens:
        
        \begin{itemize}
            \item Criar novos algoritmos escritos em \Python, tanto para a GUI quanto para otimizar o tempo de processamento dos dados;
            \item Atualizar algoritmos já existentes usando novas tecnologias;
            \item Testar e comparar a eficiência do programa desenvolvido em relação a técnicas consolidadas no processamento, através de dados reais processados por diferentes perfis de usuários.
        \end{itemize}

%\chapter{JUSTIFICATIVA}
    
    \citar{vamos colocar a justificativa aqui}

\chapter{FUNDAMENTOS DO MÉTODO MAGNETOTELÚRICO}

    Proposto por \citeauthor{tikhonov1950determining} (\citeyear{tikhonov1950determining}) e \citeauthor{cagniard1953basic} (\citeyear{cagniard1953basic}) o método magnetotelúrico usa fontes passivas\footnote{São fontes de sinal que não dependem de instrumentos artificiais para gerá-la, ou seja, sinais naturais do planeta.} eletromagnéticas do planeta Terra para estudar e mapear a distribuição de condutividade elétrica em subsuperfície.
    
    Nesta secção será mostrado sucintamente a origem do sinal MT. Entretando, antes de demonstrar as bases teóricas do método, será descrito brevemente os conceitos de resistividade elétrica aplicada para o meio geológico. O estudo da resstividade elétrica no meio geológico é essencial para as interpretações dos dados resultantes do método MT.
    
    
    %Nesta secção será mostrado sucintamente a origem do sinal MT. Entretanto, antes de demonstrar as bases teóricas do método, será descrito brevemente sobre a resistividade elétrica, sendo esse o elemento fundamental para as interpretações lito-geofísicas com base no magnetotelúrico.
        

    \section{Resistividade Elétrica dos Materiais}
    
        
        O método MT usa a resistividade elétrica ($\rho$ [$\Omega m$]) ou o seu inverso, a condutividade elétrica ($\sigma$ [$S$]), para distinguir e estudar a distribuição dos elementos geológicos em subsuperfície. 
        
        A resistividade elétrica é um parâmetro físico intrínseco a cada material. Ela pode ser definida pela oposição do fluxo de corrente elétrica em um material, ou seja, a resistividade elétrica é igual ao campo elétrico sobre a densidade de corrente \cite{eletromag8hayt}:
        
        \begin{equation}
            \label{resis-antes}
            \rho = \dfrac{E}{J}
        \end{equation}

        \noindent onde, $E$ é a magnitude do campo elétrico em [V/m] e $J$ é a magnetude da densidade de corrente em [A/m\textsuperscript{2}].
        
        Se o campo elétrico e a densidade de corrente elétrica forem constantes, eles podem assumir os seguintes valores:
        
        \begin{equation}
            \label{campo-eletrico}
            E = \frac{\Delta U}{\Delta l}
        \end{equation}
        
        {\footnotesize \noindent
            \begin{table}[H]
                \begin{tabular*}{5cm}{p{0.4cm}p{0.1cm}p{10cm}}
                    {\footnotesize $\Delta U$}  & {\footnotesize $\rightarrow$} & {\footnotesize Diferença de potêncial [V] }\\
                    {\footnotesize $\Delta l$}         & {\footnotesize $\rightarrow$} & {\footnotesize Comprimento do condutor [m]}\\
                \end{tabular*}
            \end{table}}
            
        \noindent e ainda,
        
        \begin{equation}
            \label{densidade-de-corrente}
            J = \frac{i}{A}
        \end{equation}
        
        {\footnotesize \noindent
            \begin{table}[H]
                \begin{tabular*}{1cm}{p{0.05cm}p{0.1cm}p{10cm}}
                    {\footnotesize $i$}  & {\footnotesize $\rightarrow$} & {\footnotesize Corrente Elétrica [A] }\\
                    {\footnotesize $A$}  & {\footnotesize $\rightarrow$} & {\footnotesize Seção transversal do condutor [m\textsuperscript{2}]}\\
                \end{tabular*}
            \end{table}}

        \noindent Substituindo as equações [\ref{campo-eletrico}] e [\ref{densidade-de-corrente}] na equação [\ref{resis-antes}], obtém-se:
        
        \begin{equation}
            \label{resistividade}
            %\rho = \dfrac{\Delta U \, A}{\Delta l \, i}
            \rho = \dfrac{\Delta U}{i} \dfrac{A}{\Delta l}
        \end{equation}

        Utilizando a equação [\ref{resistividade}], pode-se estimar a resistividade elétrica de um material geológico experimentalmente, a partir do arranjo mostrado na Figura \ref{arranjo-resistividade}.

        \begin{figure}[H]
            \caption[Arranjo para obter a resistividade elétrica]{Arranjo para obter experimentalmente a resistividade elétrica de um material geológico.}
                \begin{center}
                    \includegraphics[width=3cm]{texto/figura/resisti_telford.png}
                \end{center}
            \legend{\Fonte{Adaptado \cite{telford}.}}
            \label{arranjo-resistividade}
        \end{figure}
        
        Devido a complexidade físico-química dos materiais geológicos, a resistividade elétrica é representada por um intervalo de valores. A Figura \ref{tabela-resis} mostra esses intervalos de valores para alguns tipos de materiais geológicos. Entretanto vale ressaltar que os valores podem serem alterados dependendo do contexto geológico. 
        
        \begin{figure}[H]
            \caption{Tabela de resistividade elétrica para materiais geológicos.}
                \begin{center}
                    \includegraphics[width=10cm]{texto/figura/resistividade_tabela.png}
                \end{center}
            \legend{\Fonte{Adaptado \cite{palacky}.}}
            \label{tabela-resis}
        \end{figure}
    
    \section{Origem das Correntes Telúricas}
    
        O método MT utiliza-se de um amplo espectro do campo eletromagnético natural terrestre (10\textsuperscript{-4} a 10\textsuperscript{4} Hz) para as sondagens geofísicas. Essa característica permite que  a sondagem magnetotelúrica alcance centenas de quilômetros.
        
        O sinal MT tem sua origem nas ressonâncias de Schumann, nas micropulsações e nas variações diurnas \cite{padua2004estudos}. A figura \ref{sinalmt} mostra a contribuição de cada mecânismo no espectro MT.
        
        \begin{figure}[H]
            \caption{Campo magnético natural e as contribuições das fontes do sinal MT.}
                \begin{center}
                    \includegraphics[width=12cm]{texto/figura/padua04.eps}
                \end{center}
            \legend{\Fonte{Adaptado \cite{padua2004estudos}.}}
            \label{sinalmt}
        \end{figure}
        
        As ressonâncias de Schumann tem sua origem principalmente nas tempestades equatoriais contribuindo para a fonte do sinal MT acima de 1 Hz. Frequências a baixo desse valor, tem origem na interação do vento solar com a magnetosfera, que geram ressonâncias Terra-Ionosfera. A contribuição de parte do espectro MT, tambem pode ser explicada pela distorção do formato do campo magnético terrestre causado pelo Sol durante o dia, no processo denominado variação diurna que contribui com a faixa de frequência de 10\textsuperscript{-5} a 10\textsuperscript{-4} Hz. 
        
        
        %\subsection{Ressonâncias de Schumann}
        
        %\subsection{Micropulsações}
        
        %\subsection{Variações Diurnas}   
        
    \section{Resposta do Método Magnetotelúrico}

        No caso do \mt{} assim como nos casos de outros métodos geofísicos eletromagnéticos, que fundamentam-se nas Leis de Maxwell [Equações \ref{lei-max-rotE} -- \ref{lei-max-divD}], pode-se partir das equações: 
        
        \begin{equation}
            \label{lei-max-rotE}
            \rot{E} = - \devpart{\vetor{B}}
        \end{equation}
        \begin{equation}
            \label{lei-max-rotH}
            \rot{H} = \vetor{J} + \devpart{\vetor{D}}
        \end{equation}

        \begin{equation}
            \label{lei-max-divB}
            \nabla \cdot \vetor{B} = 0 
        \end{equation}

        \begin{equation}
            \label{lei-max-divD}
            \nabla \cdot \vetor{D} = \varrho                
        \end{equation}
        
        \noindent Onde,
            
        {\footnotesize \noindent
            \begin{table}[H]
                \begin{tabular*}{1cm}{p{0.05cm}p{0.1cm}p{10cm}}
                    {\footnotesize $\vec{\textrm{E}}$}  & {\footnotesize $\rightarrow$} & {\footnotesize Vetor Campo Elétrico [V/m] }\\
                    {\footnotesize $\vec{\textrm{B}}$}  & {\footnotesize $\rightarrow$} & {\footnotesize Vetor Campo Magnético [T] }\\
                    {\footnotesize $\vec{\textrm{H}}$}  & {\footnotesize $\rightarrow$} & {\footnotesize Vetor Campo Magnetizante [A/m]} \\
                    {\footnotesize $\vec{\textrm{J}}$}  & {\footnotesize $\rightarrow$} & {\footnotesize Vetor Densidade de Corrente [A/m\textsuperscript{2}]} \\
                    {\footnotesize $\vec{\textrm{D}}$}  & {\footnotesize $\rightarrow$} & {\footnotesize Vetor Campo de Deslocamento Elétrico [C/m\textsuperscript{2}]} \\
                    {\footnotesize $\varrho$}           & {\footnotesize $\rightarrow$} & {\footnotesize Densidade de Carga [C/m\textsuperscript{3}]} \\
                    {\footnotesize $t$ }                & {\footnotesize $\rightarrow$} & {\footnotesize Tempo [s]}
                \end{tabular*}
            \end{table}}
        \noindent estimar os parâmetros físicos para a investigação MT.
        
        Para os estudos \mt s são feitas as seguintes afirmações, que auxiliam e simplificam o desenvolvimento:
       
        \begin{quote}
            I) A Terra comporta-se como um condutor ôhmico e um semi-espaço isotrópico.
        \end{quote}
        
        Podemos utilizar, partindo dessas características e atrelado a um campo eletromagnético pouco intenso, as seguintes relações constitutivas:
        \begin{equation}
         \vetor{B} = \mu \vetor{H}
        \end{equation}
        
        \begin{equation}
         \vetor{D} = \varepsilon \vetor{E}
        \end{equation}
        
        \begin{equation}
         \vetor{J} = \sigma \vetor{E}
        \end{equation}

        {\footnotesize \noindent
            \begin{table}[H]
                \begin{tabular*}{1cm}{p{0.05cm}p{0.1cm}p{10cm}}
                    {\footnotesize $\mu$}          & {\footnotesize $\rightarrow$} & {\footnotesize Permeabilidade Magnética [H/m] }\\
                    {\footnotesize $\varepsilon$}  & {\footnotesize $\rightarrow$} & {\footnotesize Permissividade Elétrica [F/m] }\\
                    {\footnotesize $\sigma$}       & {\footnotesize $\rightarrow$} & {\footnotesize Condutividade Elétrica [S/m]} \\
                \end{tabular*}
            \end{table}}

        Cada coeficiente das relações constitutivas funcionam como tensores, variantes no tempo, para meios anisotrópicos. Para o estudo abordado e seguindo a afirmação, onde a Terra torna-se um meio isotrópico, isso implica que os tensores, $\mu$ e $\varepsilon$ são estáticos e assumem os valores referência para o vácuo:
        
        \begin{quote}
            \begin{enumerate}
                \item[$\mu$] $=$ \basedez{1{,}2566}{-6} H/m
                \item[$\varepsilon$] $=$ \basedez{8{,}85}{-12} F/m
            \end{enumerate}
        \end{quote}
        
        Utilizando as equações constitutivas podemos reescrever as equações \ref{lei-max-rotE} e \ref{lei-max-rotH}:
        
        {\setlength\arraycolsep{2pt}
        \begin{eqnarray}
            \label{rotEconstH}
            \rot{E} &=& - \devpart{\vetor{B}}; \qquad \vetor{B} = \mu \vetor{H} \nonumber \\
            \rot{E} &=& - \mu \devpart{\vetor{H}}
        \end{eqnarray}}

        {\setlength\arraycolsep{2pt}
        \begin{eqnarray}
            \label{rotHconstE}
            \rot{H} &=& \vetor{J} + \devpart{\vetor{D}}; \qquad \vetor{J} = \sigma \vetor{E} \quad \textrm{e} \quad \vetor{D} = \varepsilon \vetor{E}\nonumber \\
            & &\rot{H} =  \sigma \vetor{E} + \varepsilon \devpart{\vetor{E}}
        \end{eqnarray}}
        
        Na faixa da sondagem MT a Terra comporta-se como um condutor ôhmico, isso implica que o meio não possuem cargas livres, logo $\varrho \simeq 0$. 
        
        Para os campos pode ser assumida uma dependência temporal harmônica dada por $e^{- \imath \omega t}$, que pode ser decomposta em vários harmônicos pela transformada de Fourier, onde $t$ é o tempo e $\omega$ a frequência angular. 
    
        Portando as equações: \ref{rotEconstH}, \ref{rotHconstE}, \ref{lei-max-divB} e \ref{lei-max-divD}, podem ser reescritas como:
        
        \begin{equation}
            \label{rotEconstHreescrito}
            \rot{E} = \imath \omega \mu \vetor{H}           
        \end{equation}
        
        \begin{equation}
            \label{rotHconstEreescrito}
            \rot{H} = (\imath \omega \varepsilon + \sigma) \vetor{E}
        \end{equation}
        
        \begin{equation}
            \nabla \cdot \vetor{H}  = 0
        \end{equation}

        \begin{equation}
            \nabla \cdot \vetor{E} = 0
        \end{equation}

        Aplicando o rotacional na equação \ref{rotHconstEreescrito}, obtemos:
         
        \begin{equation}
            \label{rotrotH}
            \nabla \times \nabla \times \vetor{H} = (\imath \omega \varepsilon + \sigma) \rot{E}
        \end{equation}

        Comparando a equação \ref{rotrotH} com a equação \ref{rotEconstHreescrito}, pode-se reescreve-la como:
        
        {\setlength\arraycolsep{2pt}
        \begin{eqnarray}
            \label{preHk}
            \nabla \times \nabla \times \vetor{H} &=& (\imath \omega \varepsilon + \sigma) \rot{E}; \qquad \rot{E} = \imath \omega \mu \vetor{H} \nonumber \\
            & &\nabla \times \nabla \times \vetor{H} = \imath \omega \mu (\imath \omega \varepsilon + \sigma) \vetor{H}            
        \end{eqnarray}}
        
        Pode-se expressar a equação \ref{preHk} usando a seguinte identidade vetorial:
        
        \begin{equation}
            \nabla \times \nabla \times \vetor{A} = \nabla \nabla \cdot \vetor{A} - \nabla^2 \vetor{A} 
        \end{equation}

        \noindent Portanto:
        
        {\setlength\arraycolsep{2pt}
        \begin{eqnarray}
            \label{difH}
            \nabla \nabla \cdot \vetor{H} - \nabla^2 \vetor{H} & = & \imath \omega \mu (\imath \omega \varepsilon + \sigma) \vetor{H} \nonumber \\
            \nabla (\nabla \cdot \vetor{H}) - \nabla^2 \vetor{H} & = & \vetor{H}[\cancelto{\kappa^2}{\imath \omega \mu (\imath \omega \varepsilon + \sigma)}] \nonumber \\
            \nabla (\cancelto{0}{\nabla \cdot \vetor{H}}) - \nabla^2 \vetor{H} & = & \kappa^2 \vetor{H} \nonumber \\
            \nabla^2 \vetor{H} + \kappa^2 \vetor{H} & = &  0; \qquad \kappa^2 = \imath \omega \mu (\imath \omega \varepsilon + \sigma)
        \end{eqnarray}}        
        
        Considerando um condutor ôhmico ($\sigma \gg \omega \varepsilon$), assim:
        \begin{equation}
            \label{kappaquad}
            \kappa^2 = \imath \omega \mu \sigma
        \end{equation}
        
        \noindent Onde, $\kappa^2$ é o módulo do vetor de onda ($\vetor{k}$).
        
        A equação \ref{kappaquad} pode ser expressa seguindo a definição, como:
        
        {\setlength\arraycolsep{2pt}
        \begin{eqnarray}
            \kappa & = & \sqrt{\imath \omega \mu \sigma}; \quad \imath = e^{\imath \frac{\pi}{2}} \nonumber \\
            \kappa & = & \sqrt{\omega \mu \sigma} \sqrt{e^{\imath \frac{\pi}{2}}} \nonumber \\
            \kappa & = & \sqrt{\omega \mu \sigma} e^{\imath \frac{\pi}{4}}; \quad e^{\imath \frac{\pi}{4}} = \sqrt{1/2} (1 + \imath) \nonumber \\
            \kappa & = & \sqrt{\dfrac{\omega \mu \sigma}{2}} (1 + \imath) \nonumber \\
            \kappa & = & \dfrac{(1 + \imath)}{\delta}
        \end{eqnarray}} 
        
        \noindent Onde,
        
        \begin{equation}
            \label{shin-depth}
            \delta_\omega = \sqrt{\dfrac{2 \rho}{\omega \mu}} \longrightarrow \delta_f \approx 500  \sqrt{\frac{\rho_a}{f}}
        \end{equation}
        
        A equação \ref{shin-depth} é chamada de \en{skin-depth} (espessura pelicular), e representa a profundidade de penetração da onda eletromagnética em um meio condutor.
        A partir da equação são mapeadas as litologias em subsuperfície. Porém a resistividade ($\rho$) representa todo o pacote de rochas em subsuperfície, portanto, adota-se o termo resistividade aparente ($\rho_a$) para a mesma. A resistividade efetiva ($\rho$) pode ser obtida a partir do processo de inversão geofísica (não tratado neste trabalho).        

        O meio geológico influencia diretamente a profundidade de investigação. A Figura \ref{fig-skin-depth} mostra que para uma mesma frequência, pode representar valores diferentes de profundidade, variando o meio em subsuperfície, isso é representado por $\rho$. Os meios mais resistivos geram profundidade maiores, já meios condutivos diminuem a profundidade. Esse fenômeno é importante porque, ao interpretar as seções lito-geofísicas, é comum estudar contextos de bacias sedimentares (meio condutivo) em contado com contextos cristalinos (meio resistivo), a atenção deve-se voltar para o fato de que um mesmo período em função de $\rho$ pode representar duas profundidades diferentes, estando a estação em cima do contexto sedimentar ou em cima do contexto cristalino. 
        
        \begin{figure}[H]
            \caption[Gráfico do \en{skin-depth}]{Gráfico do \en{skin-depth} em função da frequência [Hz], variando a resistividade do meio}
            \begin{center}
                \includegraphics[width=10cm]{texto/figura/skin-depth.png}
            \end{center}
            \legend{\Fonte{\oautor.}}
            \label{fig-skin-depth}
        \end{figure}
        
    \section{Impedância Eletromagnética}
        \label{sec-impedancia}
        Baseado na fundamentação teórica apresentada na seção anterior, o MT busca obter a resistividade aparente em função da profundidade. A partir da solução da equação \ref{difH} e da sua análoga para o campo $\vetor{E}$, onde são dadas por:
        
        \begin{equation}
            \label{soluH}
            \vetor{H}_{(\vetor{r})} = \vetor{H} e^{-\vetor{k} \cdot \vetor{r}}
        \end{equation}

        \begin{equation}
            \label{soluE}
            \vetor{E}_{(\vetor{r})} = \vetor{E} e^{-\vetor{k} \cdot \vetor{r}}
        \end{equation}

        %\noindent Onde $\vetor{k}$ é o vetor de onda, cujo o módulo é definido por $\kappa$ na equação \ref{kappaquad}.
        
        Substituindo a equação \ref{soluH} e \ref{soluE} em \ref{rotHconstEreescrito}, temos:
        
        {\setlength\arraycolsep{2pt}
        \begin{eqnarray}
            \label{HEpossolu}
            \nabla \times \vetor{H}e^{-\vetor{k} \cdot \vetor{r}} & = & (\imath \omega \varepsilon + \sigma) \vetor{E}e^{-\vetor{k} \cdot \vetor{r}}; \quad \sigma \gg \imath \omega \varepsilon \ \nonumber \\
            \nabla \times \vetor{H}e^{-\vetor{k} \cdot \vetor{r}} & = & \sigma \vetor{E}e^{-\vetor{k} \cdot \vetor{r}}; \quad \sigma = \dfrac{\kappa^2}{\imath \omega \mu} \nonumber \\
            \nabla \times \vetor{H}e^{-\vetor{k} \cdot \vetor{r}} & = &\dfrac{\kappa^2}{\imath \omega \mu} \vetor{E}e^{-\vetor{k} \cdot \vetor{r}}
        \end{eqnarray}} 
        
        \noindent Usando as identidades:
        
        \begin{equation}
            \nabla (e^{-\vetor{k} \cdot \vetor{r}}) = - e^{-\vetor{k} \cdot \vetor{r}} \vetor{k}
        \end{equation}

        \begin{equation}
            \nabla \times \vetor{C}(f_{(\vetor{r})}) = - \vetor{C} \times \nabla f_{(\vetor{r})} 
        \end{equation}
        
        Pode-se reescrever a equação \ref{HEpossolu}:
        
        {\setlength\arraycolsep{2pt}
        \begin{eqnarray}
            \label{EZHvetor}
            - \vetor{H} \times (- e^{-\vetor{k} \cdot \vetor{r}} \vetor{k}) = \dfrac{\kappa^2}{\imath \omega \mu} \vetor{E}e^{-\vetor{k} \cdot \vetor{r}} \nonumber \\
            \cancelto{}{e^{-\vetor{k} \cdot \vetor{r}}} (\vetor{H} \times \vetor{k}) = \cancelto{}{e^{-\vetor{k} \cdot \vetor{r}}} \dfrac{\kappa^2}{\imath \omega \mu} \vetor{E}\nonumber \\
            \vetor{E} = \dfrac{\imath \omega \mu}{\kappa^2} \vetor{H} \times \vetor{k} \nonumber \\
            \vetor{E} = \dfrac{\imath \omega \mu}{\kappa} \vetor{H} \times \dfrac{\vetor{k}}{\kappa}
        \end{eqnarray}} 
        
        \noindent A relação $\vetor{k}/\kappa$ é o versor de $\vetor{k}$ ou  $\hat{\unitario{k}}$, representando a ortogonalidade entre $\vetor{H}$ e $\vetor{E}$.
        
        A partir da equação anterior pode ser definido que $Z = \imath \omega \mu / \kappa$, esta definição é conhecida como impedância intrínseca do meio ou impedância eletromagnética, também pode ser representada da seguinte forma:
        
        \begin{equation}
            Z = \dfrac{|\vetor{E}|}{|\vetor{H}|} = \dfrac{\imath \omega \mu}{\kappa} = \sqrt{\omega \mu \rho} e^{\imath \frac{\pi}{4}}
        \end{equation}
        
        A impedância eletromagnética ($Z$) pode ser decomposta em função das componentes de $\vetor{E}$ e $\vetor{H}$, representada na forma matricial:
        
        \begin{equation}
            \label{tensor-impe}
            \left (\begin{array}{c}
                \textrm{E}_x\\
                \textrm{E}_y
                    \end{array}\right)
                =
            \left (\begin{array}{cc}
                \textrm{Z}_{xx} & \textrm{Z}_{xy}\\
                \textrm{Z}_{yx} & \textrm{Z}_{yy}
                    \end{array}\right)
            \left (\begin{array}{c}
                \textrm{H}_x\\
                \textrm{H}_y
                    \end{array}\right)
	    \end{equation}
	    
	    O método MT, então, obtém a resistividade aparente a partir da impedância eletromagnética e atribui a ela uma profundidade, que pode ser definida pela função de \en{skin-depth} (Equação \ref{shin-depth}).          

\chapter{AQUISIÇÃO DE DADOS E DEPENDÊNCIAS}
    
    Neste capítulo será discutido como é realizada a aquisição de dados MT, e quais as técnicas atualmente utilizadas para o processamento de dados. Também será mostrado quais as dependências que foram necessárias para a construção do \software, dentre elas estão: Kivy, EMTF (Dnff e TranMT) e conversores de dados.
        
    \section{Aquisição de Dados MT}        
    
        A aquisição de dados MT consiste na obtenção dos campos elétricos ($\campo{E}_x$ e $\campo{E}_y$) e magnéticos ($\campo{H}_x$, $\campo{H}_y$ e $\campo{H}_z$), onde são os parâmetros essenciais para o cálculo da impedância ($Z$).
        
        Devido a sensibilidade do sinal das sondagens MT, os sensores devem proporcionar uma alta relação sinal/ruído e alta capacidade de ampliar o sinal medido.
        
        O arranjo amplamente adota para aquisição, são três magnetômetros distribuídos cada um paralelo a um eixo cartesiano, responsáveis pelos campos magnéticos. Para os campos elétricos, são distribuídos dois arranjos de eletrodos não polarizados, onde são acoplados horizontalmente, no sentido $x$ e $y$. A figura \ref{fig-arranjo-mt} mostra a disposição dos sensores na superfícies.
        
        Vale ressaltar que o eixo $x$ da composição cartesiana deve estar paralelo a direção do fluxo magnético terrestre, ou seja, direcionado ao polo magnético terrestre\footnote{Ponto na superfícies terrestre onde a inclinação magnética é $+90^\circ$}. 
        
        \begin{figure}[H]
            \caption{Arranjo para Aquisição de dados MT.}
                \begin{center}
                    \includegraphics[width=13cm]{texto/figura/arranjo-mt.eps}
                \end{center}
            \legend{\Fonte{\oautor.}}
            \label{fig-arranjo-mt}
        \end{figure}
        
        Os sensores registram a variação da amplitude do sinal em função do tempo, esses registros são chamados de series temporais e são considerados os dados brutos do método. 
        
        Devido ao grande intervalo do espectro eletromagnético que abrange as sondagens MT ($10^{-3}\, Hz$ a $10^{4}\, Hz$), são configuradas varias taxas de aquisições diferentes. Para cada escolha de taxa de aquisição é considerada a representatividade do sinal respeitando a frequência de Nyquist \cite{nyquist28}. A representatividade é muito importante, pois, o sinal medido pelos sensores é a composição de várias ondas com frequências angulares diferentes, se a taxa de aquisição for menor que duas vezes a frequência da onda, ela não pode ser representada fielmente.  
        
        A figura \ref{fig-aquisicao} exemplifica o conceito apresentado no paragrafo anterior, no exemplo é mostrado a composição de um onda com 10 frequências diferentes ($f_{\omega(t)}$), variando de $1$ a $10\, Hz$, se atribuirmos à ela uma taxa de aquisição de $10\,Hz$ pode-se perceber que a frequência de $6\,Hz$ não é representada corretamento, já percebe-se que para a frequência de $1\,Hz$ é super representada, isso acaba aumentando o tamanho dos arquivos de aquisição. A escolha da taxa de aquisição deve conciliar na melhor forma possível esses dois fatos.
        
        \begin{figure}[H]
            \caption{Aquisição de Dados Discretos.}
                \begin{center}
                    \includegraphics[width=13cm]{texto/figura/fourier2.eps}
                \end{center}
            \legend{\Fonte{\oautor.}}
            \label{fig-aquisicao}
        \end{figure}
    
        As taxas de aquisições comumente utilizadas são valores estimados por potências de 2, isso facilita na decomposição das frequências pela transforma de Fourier. Cada taxa de aquisição é chamada de \en{Banda} e varia de nome para cada equipamento utilizado. 
    
    \section{Formatos de Arquivos de Dados MT}   
    
        Parte das funções do \software{} será a simplificação no processo de conversão de dados, atualmente os formatos mais utilizados, são:
        
        {\footnotesize \noindent
            \begin{table}[H]
                \begin{tabular*}{1cm}{p{2.05cm}p{0.5cm}p{10cm}}
                    \en{TS-format}       & {\footnotesize $\rightarrow$} & \en{Time Series Format} \\
                    \en{Z-file}          & {\footnotesize $\rightarrow$} & \en{Z (Impedance Tensor) File} \\
                    \en{J-format}        & {\footnotesize $\rightarrow$} & \en{Jones Format} \\
                    \en{EDI-format}      & {\footnotesize $\rightarrow$} & \en{Eletrical Data Interchange Format} \\
                \end{tabular*}
            \end{table}}
        
        Os arquivos \en{TS} são utilizados para registrar as series temporais, onde são armazenadas as amplitudes registradas pelos sensores em função do tempo. A grande parte dos equipamentos tem como saída padrão a forma binária dos arquivos \en{TS}. Os arquivos binários podem  posteriormente serem convertidos para o formato ASCII.
        
        O arquivo \en{TS} é composto por dois blocos, o primeiro é destinado a comentários e configurações da aquisição, já  o segundo compõe o bloco de dados, distribuídos em cinco colunas, cada uma registra a amplitude do sinal dos sensores $H_x$, $H_y$, $H_z$, $E_x$ e $E_y$. O tempo associado a cada registro pode ser estimado pela hora de inicio e a taxa de aquisição da rodada.   
        
        \SingleSpacing
        
        Exemplo de arquivo \en{TS} (ASCII):

        \begin{footnotesize}        
\begin{verbatim}
        # time series file from mp2ts 
        # date: Mon May 12 10:15:57 1997
        # input file: sno101/sno101as.1mp
        # site description: KM 222.5
        # Latitude   :062:39:47 N
        # Longitude  :116:12:32 W
        # LiMS         number :           52
        # Magnetometer number :           52
        # Ex line length (m):     100.0000000
        # Ey line length (m):     100.0000000
        # Azimuths relative to: MAGNETIC NORTH 
        # Ex azimuth;          -17
        # Ey azimuth;           73 
        # Hx azimuth;          -17 
        # Hy azimuth;           73 
        1.98250   0.878400    3.64780    1.10889    2.02644                                  
        1.93980   0.976000    3.65390    1.15682    2.01610                                     
\end{verbatim}
\end{footnotesize}
            \begin{flushright}
                \cite{ts-format}
            \end{flushright}
        
        Como comentado no capítulo anterior, 
        
        Os arquivos \en{Z} são a saída padrão do processamento Robusto (Seção \ref{sec-robusto}), diferentes dos arquivo

            
        
    
    \section{Processamento de Dados MT}
    
    \subsection{Processamento Robusto -- EMTF}
        \label{sec-robusto}
        \cite{robusto-egbert}
        \subsubsection{Mudança Tempo/Frequência Ângular}
            
        \subsubsection{Função de Transferências}
           
    \section{Pacotes de Processamento do Grupo Geoma -- INPE}
    
    \section{Construtor Gráfico -- Kivy}

    

\chapter{DESENVOLVIMENTO E ARQUITETURA DO PampaMT}

\chapter{APLICAÇÃO DO PAMPAMT}
    
    Nesse capítulo será demonstrado a aplicação do programa desenvolvido utilizando dados reais. Onde os dados foram processados utilizando o PampaMT e também a forma tradicional que vinha sendo realizada. Essa forma tradicional compreende processar utilizando os \en{scripts} e programas desenvolvidos pelo GEOMA, como comentado anteriormente.
    
    A realização dos processamentos utilizando as duas técnicas simultâneas, foi realizada visando comparar o tempo e os períodos escolhidas para cada técnica.  
    
    \section{Área de Estudo}
        
        A área de estudo escolhida para esse teste de comparação são 12 estações magnetotelúricas de banda larga, localizadas no nordeste brasileiro dentro da provícia borborema (Figura \ref{local-bor}). As estações MT fazem parte do projeto ``Estudos geofísicos e tectônicos na Província Borborema, Nordeste do Brasil” /CNPQ (Projeto Milênio) e “Estudo da estrutura da litosfera do Nordeste do Brasil” /CNPQ (INCT – Tectônica), levantadas nos anos de 2007 e 2009.
        
    \begin{figure}[H]
        \caption{Mapa de Localização}
            \begin{center}
                \includegraphics[width=12cm]{texto/figura/local-bor.png}
            \end{center}
        \legend{\Fonte{\oautor.}}
        \label{local-bor}
    \end{figure}        
        
    \section{Contexto Geológico}
    
    Segundo \cite{almeida}, a Província Borborema caracteriza-se como um complexo conjunto de blocos crustais reunidos por causa de processos geológicos que finalizaram na Orogenia Brasiliana/Pan-africana (700 a 450 Ma). Devido à complexidade tectônica, diferentes estudos vêm sendo apresentados por distintos pesquisadores para explicar as características dessa estrutura \cite{van}; \cite{teseandrea}; \cite{santos2014deep}; \cite{padilha}; \cite{barbosa}. A Província Borborema limita-se a sul com o Cráton São Francisco; a oeste com a Bacia do Parnaíba (sedimentos Fanerozóicos); a norte e a leste com as bacias sedimentares costeiras e interiores do Nordeste do Brasil (bacias Potiguar, Pernambuco-Paraíba e Sergipe-Alagoas, além da bacia Tucano-Jatobá que transpassa o limite da província com o Cráton São Francisco) -- \cite{medeiros}.    
    
    \begin{figure}[H]
        \caption{Mapa Geológico}
            \begin{center}
                \includegraphics[width=15cm]{texto/figura/mapa_geo.png}
            \end{center}
        \legend{\Fonte{O Autor -- Base de dados \cite{cprm}, 2018.}}
        \label{local-bor}
    \end{figure}
        
    \section{Processamento dos Dados}
        
    \section{Resultados e Interpretação Geofísica}
        
        

\chapter{CONSIDERAÇÕES FINAIS}
    
    Rodando a versão de teste interno (versão alfa), a utilização do PampaMT para processamentos sugere que o principal fator, o tempo, foi drasticamente reduzido. O que tornou o processamento de dados magnetotelúricos mais dinâmico.
    
    A continuidade desse trabalho será ampliar o uso do PampaMT, adicionando novas funções, tais como: processos de inversão, modelagem, referência remota, dentre outros.
    
    %\citar{escrever um pouco mais}


% ==============================================================================

        
% PÓS-TEXTUAL
% ==============================================================================
\postextual     % Inicia os elementos pós-textuais
% ==============================================================================


% REFERENCIAS
% ==============================================================================
\bibliography{bibliografia}         % Imprime a referencia bibliografica
% ============================================================================== 
 
 
% APÊNDICE
% ==============================================================================
\apendices                      % Inicia os apêndices

\chapter{Código Fonte PampaMT}
    
    \vspace*{1cm} 
    
    O programa PampaMT esta armazenado no servidor GitHub, atualmente o mesmo é a maior comunidade de códigos fontes, nela podemos encontrar o \en{Kernel} Linux, a plataforma SU (\en{Seismic Unix}), o pacote abn\TeX2, dentre um vasto catálogo de outros projetos.   
    
    \vspace*{1cm}
    
    \begin{center}
    Repositório com o código fonte na data de entrega do trabalho de conclusão de curso.
    \end{center}
    
    \begin{figure*}[h]
        \begin{center}
            \includegraphics[width=5cm]{texto/figura/qr-code-pampamt.eps}
        \end{center}
    \end{figure*}
    \begin{center}
        \url{https://github.com/PatrickRogger/PampaMT}
    \end{center}
    
    \begin{center}
    Repositório com o código fonte para desenvolvedores.
    \end{center}
    
    \begin{figure*}[h]
        \begin{center}
            \includegraphics[width=5cm]{texto/figura/qr-code-git-pampamt.eps}
        \end{center}
    \end{figure*}
    \begin{center}
        \url{https://github.com/PampaMT/PampaMT}
    \end{center}

    \vfill
\chapter{Pré-processamento Usando o Método Tradicional}

    %\begin{landscape}
    \begin{figure}[H]
        \caption{Manual -- bor603b}
            \begin{center}
                \includegraphics[width=15cm]{texto/figura/sites/M-bor603b.png}
            \end{center}
        \legend{\Fonte{\oautor.}}
    \end{figure}
    %\end{landscape}
    \begin{figure}[H]
        \caption{Manual -- bor604a}
            \begin{center}
                \includegraphics[width=15cm]{texto/figura/sites/M-bor604a.png}
            \end{center}
        \legend{\Fonte{\oautor.}}
    \end{figure}
    
    \begin{figure}[H]
        \caption{Manual -- bor604b}
            \begin{center}
                \includegraphics[width=16cm]{texto/figura/sites/M-bor604b.png}
            \end{center}
        \legend{\Fonte{\oautor.}}
    \end{figure}
    
    \begin{figure}[H]
        \caption{Manual -- bor605a}
            \begin{center}
                \includegraphics[width=16cm]{texto/figura/sites/M-bor605a.png}
            \end{center}
        \legend{\Fonte{\oautor.}}
    \end{figure}
    
    \begin{figure}[H]
        \caption{Manual -- bor605b}
            \begin{center}
                \includegraphics[width=15cm]{texto/figura/sites/M-bor605b.png}
            \end{center}
        \legend{\Fonte{\oautor.}}
    \end{figure}
    
    \begin{figure}[H]
        \caption{Manual -- bor606a}
            \begin{center}
                \includegraphics[width=15cm]{texto/figura/sites/M-bor606a.png}
            \end{center}
        \legend{\Fonte{\oautor.}}
    \end{figure}
    
    \begin{figure}[H]
        \caption{Manual -- bor606b}
            \begin{center}
                \includegraphics[width=16cm]{texto/figura/sites/M-bor606b.png}
            \end{center}
        \legend{\Fonte{\oautor.}}
    \end{figure}
    
    \begin{figure}[H]
        \caption{Manual -- bor607a}
            \begin{center}
                \includegraphics[width=16cm]{texto/figura/sites/M-bor607a.png}
            \end{center}
        \legend{\Fonte{\oautor.}}
    \end{figure}
    
    \begin{figure}[H]
        \caption{Manual -- bor607b}
            \begin{center}
                \includegraphics[width=16cm]{texto/figura/sites/M-bor607b.png}
            \end{center}
        \legend{\Fonte{\oautor.}}
    \end{figure}
    
    \begin{figure}[H]
        \caption{Manual -- bor608a}
            \begin{center}
                \includegraphics[width=16cm]{texto/figura/sites/M-bor608a.png}
            \end{center}
        \legend{\Fonte{\oautor.}}
    \end{figure}
    
    \begin{figure}[H]
        \caption{Manual -- bor608b}
            \begin{center}
                \includegraphics[width=16cm]{texto/figura/sites/M-bor608b.png}
            \end{center}
        \legend{\Fonte{\oautor.}}
    \end{figure}
    
    \begin{figure}[H]
        \caption{Manual -- bor608c}
            \begin{center}
                \includegraphics[width=16cm]{texto/figura/sites/M-bor608c.png}
            \end{center}
        \legend{\Fonte{\oautor.}}
    \end{figure}
    
\chapter{Pré-processamento Usando o PampaMT}

\begin{figure}[H]
        \caption{PampaMT -- bor603b}
            \begin{center}
                \includegraphics[width=15cm]{texto/figura/sites/P-bor603b.png}
            \end{center}
        \legend{\Fonte{\oautor.}}
    \end{figure}
    %\end{landscape}
    \begin{figure}[H]
        \caption{PampaMT -- bor604a}
            \begin{center}
                \includegraphics[width=15cm]{texto/figura/sites/P-bor604a.png}
            \end{center}
        \legend{\Fonte{\oautor.}}
    \end{figure}
    
    \begin{figure}[H]
        \caption{PampaMT -- bor604b}
            \begin{center}
                \includegraphics[width=16.5cm]{texto/figura/sites/P-bor604b.png}
            \end{center}
        \legend{\Fonte{\oautor.}}
    \end{figure}
    
    \begin{figure}[H]
        \caption{PampaMT -- bor605a}
            \begin{center}
                \includegraphics[width=16.5cm]{texto/figura/sites/P-bor605a.png}
            \end{center}
        \legend{\Fonte{\oautor.}}
    \end{figure}
    
    \begin{figure}[H]
        \caption{PampaMT -- bor605b}
            \begin{center}
                \includegraphics[width=16.5cm]{texto/figura/sites/P-bor605b.png}
            \end{center}
        \legend{\Fonte{\oautor.}}
    \end{figure}
    
    \begin{figure}[H]
        \caption{PampaMT -- bor606a}
            \begin{center}
                \includegraphics[width=16.5cm]{texto/figura/sites/P-bor606a.png}
            \end{center}
        \legend{\Fonte{\oautor.}}
    \end{figure}
    
    \begin{figure}[H]
        \caption{PampaMT -- bor606b}
            \begin{center}
                \includegraphics[width=16.5cm]{texto/figura/sites/P-bor606b.png}
            \end{center}
        \legend{\Fonte{\oautor.}}
    \end{figure}
    
    \begin{figure}[H]
        \caption{PampaMT -- bor607a}
            \begin{center}
                \includegraphics[width=16.5cm]{texto/figura/sites/P-bor607a.png}
            \end{center}
        \legend{\Fonte{\oautor.}}
    \end{figure}
    
    \begin{figure}[H]
        \caption{PampaMT -- bor607b}
            \begin{center}
                \includegraphics[width=16.5cm]{texto/figura/sites/P-bor607b.png}
            \end{center}
        \legend{\Fonte{\oautor.}}
    \end{figure}
    
    \begin{figure}[H]
        \caption{PampaMT -- bor608a}
            \begin{center}
                \includegraphics[width=16.5cm]{texto/figura/sites/P-bor608a.png}
            \end{center}
        \legend{\Fonte{\oautor.}}
    \end{figure}
    
    \begin{figure}[H]
        \caption{PampaMT -- bor608b}
            \begin{center}
                \includegraphics[width=16.5cm]{texto/figura/sites/P-bor608b.png}
            \end{center}
        \legend{\Fonte{\oautor.}}
    \end{figure}
    
    \begin{figure}[H]
        \caption{PampaMT -- bor608c}
            \begin{center}
                \includegraphics[width=16.5cm]{texto/figura/sites/P-bor608c.png}
            \end{center}
        \legend{\Fonte{\oautor.}}
    \end{figure}

%\chapter{Codigo fonte}
%    Conteudo do apendice A
% ==============================================================================


% ANEXO
% ==============================================================================
%\anexos                         % Inicia os anexos


% ==============================================================================

    
% INDICE
% ==============================================================================
% Para utilizar o indexamento automatico use no corpo do texto:
%   \index{palavra} 
%
%\printindex             % Imprime o indice
% ==============================================================================

\end{document}
