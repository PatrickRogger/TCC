
\chapter{Objetivos \, \,\,\,\,\, OK}
\label{cap-objetivos}

\section{Objetivos Gerais}
\label{cap-objetivos gerais}

    
    O objeto de estudo desse trabalho é o desenvolvimento de um \en{Software} para tratamentos e processamentos de dados \MT, também a aplicação do programa no processamento e comparação de dados reais coletados no nordeste brasileiro, conhecida como província Borborema, região que vem sendo grandemente estudada com o \MT e proporciona a este trabalho boa validação dos algoritmos de processamento.   

\section{Objetivos Específicos}
\label{cap-objetivos especificos}

    O MT é um método geofísico que trabalha com propriedades eletromagnéticas, essa característica torna o processamento dos dados extremamente trabalhoso e com alto custo computacional, o trabalho então propõe a criação de novos algoritmos e otimização dos que já existem para tornar o processamento mais fácil e barato.
    
    Tem objetivo também de obter pseudosecções de resistividades e modelos lito-geofísicos de resistividade para a região de estudo com as novas rotinas comparando com trabalhos anteriores \cite{tese_andrea} e \cite{alane}.
