
\chapter{Objetivos}
\label{cap-objetivos}

\section{Objetivos Gerais}
\label{cap-objetivos gerais}

    
    O objeto de estudo desse trabalho é o desenvolvimento de um \en{Software} para tratamentos e processamentos de dados \MT, também a aplicação do programa no processamento e comparação de dados reais coletados em uma região do nordeste brasileiro, conhecida como província Borborema, região que vem sendo grandemente estudada com o \MT e proporciona a este trabalho boa validação dos algoritmos de processamento.

\section{Objetivos Específicos}
\label{cap-objetivos especificos}

    O MT é um método geofísico que trabalha com propriedades eletromagnéticas, essa característica torna o processamento dos dados extremamente trabalhoso e com alto custo computacional, o trabalho então propõe:
    
    \begin{itemize}
     \item Criação de novos algoritmos escritos em Python.
     \item Otimização dos códigos frente as novas tecnologias.
     \item Obter secções lito-geofísicas de resistividade para a região de estudo.
     \item Comparar os resultados obtidos com trabalhos já consolidados\footnote{\cite{tese_andrea} e \cite{alane}}.
    \end{itemize}
