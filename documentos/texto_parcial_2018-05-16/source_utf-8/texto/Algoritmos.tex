\chapter{Algoritmos e Processamentos}
\label{cap-algoritmos}
    
    O \MT segue algumas etapas de processamento, dentre elas: a conversão dos dados de binário para ASCII, processamentos estatísticos para remoção de ruídos, separação de melhores bandas adquiridas em diferentes faixas de frequência e a geração de um arquivo padrão já com diversas informações processadas. Essas etapas são chamadas de pré-processamento e são as etapas que consomem a maior parte do tempo de processamento.
    
    O programa então desenvolvido nesse trabalho de conclusão de curso foca nessas etapas de processamento. As secções a seguir mostram como cada epata será abordada dentro do fluxo de processamento do programa.
    
    \section{Arquitetura do Programa e Dependências}
    \section{Conversão dos Dados}
    \section{Processamento Estatísticos}
    \section{Plotagem das Bandas de Frequências}
    \section{Criação do Arquivo de Saída Padrão}
    
    
    
