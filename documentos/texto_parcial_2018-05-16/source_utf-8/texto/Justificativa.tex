\chapter{Justificativa \,\,\,\,\, OK}
\label{cap-justificativa}

    Esse trabalho foi pensado para auxiliar a expansão do \MT, hoje existem vários programas proprietários para processamento de dados \MT, mas, são programas caros e que não permitem ver como são feitos os fluxos de processamentos dentro do programa. O processamento hoje usado na maioria dos trabalhos acadêmicos ou não usam como base os algoritmos propostos por \citeauthor{egbert97} para a primeira fase de processamento e a rotina To Jones \cite{egbert97} para compilação dos dados em arquivos manipuláveis.
    
    Esses processos hoje são feitos através de linhas de comando unindo diversos programas separados, esse processo acaba sendo muito instável provocando diversos tipos de erros e não prevendo outros, também força a instalação de vários pacotes separados que muitas vezes são compilados e instalados manualmente, afastam uma pessoa leiga no assunto a utilizar.
    
    O programa então pretende ser escrito em apenas uma linguagem e utilizar pacotes nativamente compatíveis isso preve erros de compatibilidades e torna a manutenção do código mais fácil uma vez que os códigos não serão compilados, outra forma de beneficiar a utilização do programa será a construção de uma interface gráfica para todos os processos, tornando então mais fácil e prevenindo erros provocados pelo usuário.    
    
    
