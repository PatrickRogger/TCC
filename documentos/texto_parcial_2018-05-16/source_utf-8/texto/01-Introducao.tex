% INTRODUÇÃO


\chapter{Introdução}
    \label{cap-introducao}
    
	Apoiado nas leis de Maxwell o método MT (Magnetotelúrico) usa a Terra como 
	um condutor ôhmico e as variações do seu campo 
	magnético promovido por ventos solares \cite{parkinson93} e tempestades equatoriais 
	que interagem com a ionosfera para investigar as 
	estruturas internas da Terra e litologias rasas. 
	
	
	No Brasil o uso do método MT é insipiente, restrito ao meio acadêmico e pouco
	utilizado na indústria, porém, pode ser bem aplicado na prospecção de 
	hidrocarbonetos, tendo a sua resolução melhor que a magnetometria
	e gravimetria, também em estudos crustais
	apoiando a sismologia devido sua grande profundidade de investigação, mas o 
	alto custo de processamento, a falta de \textit{softwares} para trabalhar com os 
	dados tem sido algumas das causas do fraco uso.
	
	
	Esse trabalho foi pensado primeiramente para tornar o MT mais difundido, 
	construindo um \textit{software} com interface gráfica amigável e 
	distribuição livre. O RiMT (Roplus inversion Magnetotelluric) nasceu 
	então com esse propósito, compreendendo o processamento de dados
	MT desde a coleta até a primeira visualização dos dados, como: escolha 
	de bandas, plotagem de pseudo-secções em função de resistividade e fase 
	também fazendo tratamentos estátisticos e processamento robusto 
	proposto por \citeauthor{egbert97} \citeyearpar{egbert97}.
	
	
	O programa será construido usando a linguagem Python \cite{python36} 
	e a construção da interface gráfica será desenvolvida usando a API
	Kivy \cite{kivy110} dentre outros pacotes. A escolha por essa linguagem foi a vasta quantidade de pacotes,  o crescente 
	número de pessoas implementando e a facilidade com que é a construção de seu
	código.



