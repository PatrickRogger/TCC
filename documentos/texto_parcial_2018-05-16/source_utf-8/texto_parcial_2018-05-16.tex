\documentclass[csgeo,tcc]{unipampa}                                          
\usepackage[utf8]{inputenc} 
\usepackage[T1]{fontenc}                                    
\usepackage{graphicx}                                       % pacote para importar figuras
\usepackage{chngcntr}                                       % pacote que permite a mudança da contagem das figuras e tabelas
\counterwithout{figure}{chapter}                            % contagem contínua das figuras [Não especificado nas normas, porém de preferência do autor]
\usepackage{times}                                          % pacote para usar fonte Adobe Times
\usepackage{mathptmx}                                       % pacote usar fonte Adobe Times nas fórmulas
\usepackage[alf,abnt-emphasize=bf]{abntex2cite}	           % pacote para usar citações abnt
\usepackage{ragged2e}                                       % pacote para prevenir hifenização e melhorar disposição de objetos
\hyphenpenalty=10000                                        % penalização de hifenização, força texto justificado
\usepackage{booktabs}                                       % pacote que melhora a qualidade das tabelas geradas pelo LaTeX
\usepackage[table]{xcolor}                                  % pacote para utlização de cores personalizadas
\usepackage[labelsep=endash]{caption}                       % pacote para customização de títulos ou legendas de figuras 
\usepackage{amsmath}

% ATALHOS =================================================================================================================

\newcommand{\gC}{\,$^{\rm o}{\rm C}$}                       %% macro para o símbolo de grau

% =========================================================================================================================





% TITULO, AUTOR, ORIENTADOR =============================================================================================== 

% TITULO
	\title{Desenvolvimento de Software livre para processamentos de Dados Magnetotelúricos}                              

% AUTOR
	\author{Garcia}{Patrick Rogger}                                    

% ORIENTAÇÃO
	\advisor[Prof.~Dr.]{Oliveira}{Vinicius Abreu de}                        
	\coadvisor[Prof\textsuperscript{a}.~Dr.\textsuperscript{a}.]{Matos}{Andréa Cristina Lima dos Santos}   
	\renewcommand{\coadvisorname}{Co-orientadora}
% =========================================================================================================================






% BANCA ===========================================================================================================

% BIBLIOTECA
	\cutter{} 
% MEMBRO 1
	\banca[Prof.~Dr.]{SOBRENOME}{NOME}			                
	\inst{Universidade Federal do Pampa}					    

% MEMBRO 2
	\banca[Prof.~Dr.]{SOBRENOME}{NOME}                           
	\inst{Universidade Federal do Pampa}	                    

% DATA DA DEFERSA
	\defesa{DIA}{MÊS}{ANO}                                    
% ====================================================================================================================








% DATA E LOCAL =======================================================================================================
% SE PRECISAO MUDAR A DATA DO DOCUMENTO
	%\date{MÊS}{ANO}
% LOCAL
	\location{Caçapava do Sul}{RS}                              
% =====================================================================================================================








% PALAVRAS CHAVES =====================================================================================================
	\keyword{Kivy}	
	\keyword{Magnetotelúrico}
	\keyword{Python3}
	\keyword{RiMT}
	\keyword{Software Livre}
% ======================================================================================================================








% CORRIGIR TEXTO =======================================================================================================
	\sloppy                                                       
% ======================================================================================================================








% DOCUMENTO ============================================================================================================
\begin{document}
\maketitle






% DEDICATORIA ==========================================================================================================

	%\begin{dedicatoria} 
		%Escreva aqui sua dedicatória
	%\end{dedicatoria}
% =======================================================================================================================








% AGRADECIMENTOS ========================================================================================================
	
	%\chapter*{Agradecimento}
	%Escreva aqui seus agradecimentos.
% ========================================================================================================================







% EPIGRADE ================================================================================================================
	%\begin{epigrafe}
		%``Escreva aqui o epígrafe desejado.''\\
		%--- autor da frase
	%\end{epigrafe}
% ========================================================================================================================








% RESUMO ===============================================================================================================

	\begin{abstract} 
		Escreva aqui seu resumo em português. 		
	\end{abstract}
% =========================================================================================================================







% RESUMO EM INGLES ========================================================================================================
	\begin{englishabstract}
		Escreva aqui seu resumo em língua estrangeira.
	\end{englishabstract}

% ==========================================================================================================================







% LISTA DE FIGURA E TABELA =================================================================================================

\listoffigures
\listoftables
% ==========================================================================================================================






%  LISTA DE TEOREMAS ==========================================================================================================================
% Listas de definições e teoremas, para quem usar o pacote formais, para trabalhos que possuam definições formais e teoremas

%\listofdefinitions
%\listoftheorems
% ==========================================================================================================================






% LISTA DE ABREVIATURAS ====================================================================================================

\begin{listofabbrv}{UNIPAMPA}
        
        \item[SIGLA]{NOMECOMPLETO}        
        \item[SIGLA]{NOMECOMPLETO}         
      
\end{listofabbrv}
% =============================================================================================================================







% LISTA DE SIMBOLOS ===========================================================================================================

\begin{listofsymbols}{$\alpha\beta\pi\omega$}
       \item[$\rho$]          			  Resistividade
       \item[$\nabla$]    				  Nabla
       
\end{listofsymbols}
% =============================================================================================================================







% SUMARIO ==================================================================================================================
\tableofcontents
% ==========================================================================================================================







% CORPO DO TEXTO ===============================================================================================================
\inputencoding{utf8}                                %% Arquivos em outra codificação: UTF-8

% CAPITULO 1 ===================================================================================================================
	% INTRODUÇÃO


\chapter{Introdução}
    \label{cap-introducao}
    
	Apoiado nas leis de Maxwell o método MT (Magnetotelúrico) usa a Terra como 
	um condutor ôhmico e as variações do seu campo 
	magnético promovido por ventos solares \cite{parkinson93} e tempestades equatoriais 
	que interagem com a ionosfera para investigar as 
	estruturas internas da Terra e litologias rasas. 
	
	
	No Brasil o uso do método MT é insipiente, restrito ao meio acadêmico e pouco
	utilizado na indústria, porém, pode ser bem aplicado na prospecção de 
	hidrocarbonetos, sendo a sua resolução melhor que a magnetometria
	e gravimetria, também em estudos crustais
	apoiando a sismologia devido sua grande profundidade de investigação, mas o 
	alto custo de processamento e a falta de \textit{softwares} para trabalhar com os 
	dados tem sido algumas das causas do fraco uso.
	
	
	Esse trabalho foi inicialmente pensado para tornar o MT mais difundido, 
	desenvolvento um \textit{software} com interface gráfica e 
	distribuição livre. Assim o projeto nasceu 
	com esse propósito, compreendendo o processamento de dados
	MT desde a coleta até a primeira visualização dos dados, como: escolha 
	de bandas, plotagem de pseudo-secções em função de resistividade e fase, além de fazer tratamentos estatísticos e processamento robusto 
	proposto por \citeauthor{egbert97} \citeyearpar{egbert97}.
	
	
	O programa será construído usando a linguagem Python \cite{python36} 
	e a construção da interface gráfica será desenvolvida usando a API
	Kivy \cite{kivy110} dentre outros pacotes. A escolha por essa linguagem foi a vasta quantidade de pacotes,  o crescente 
	número de pessoas implementando e a facilidade da construção do código.






% CAPITULO 2 ===================================================================================================================
	\input{texto/02-Materiais}


% CAPITULO 3 ===================================================================================================================
	% RESULTADOR ESPERADOS
\chapter{Resultados Esperados}
    \label{cap-resultados}
    Espera-se ao final desse trabalho de conclusão de curso criar um programa para processamento do método \MT, escrito em Python e de fácil usabilidade.
    
    Também melhorar a compatibilidade com os diversos sistemas operacionais e distribuir sobre a licença de \en{software livre} para a comunidade geofísica. O que deve possibilitar a expansão no \MT na academia visto que qualquer pessoa terá acesso ao programa.
    
    Ao final comparar os resultados obtidos com o programa em relação a forma que vinha sendo trabalhada até então. As principais comparações serão: tempo de processamento, visualização, tempo de aprendizagem para uso da plataforma, coerência entre resultados e manipulação da forma de visualização. 



% CAPITULO 4 ===================================================================================================================
	% CRONOGRAMA

\chapter{Cronograma de Atividades}
    \label{cap-cronograma}
    %% CRONOGRAMA

\chapter{Cronograma de Atividades}
    \label{cap-cronograma}
    %% CRONOGRAMA

\chapter{Cronograma de Atividades}
    \label{cap-cronograma}
    %\input{cap/Cronograma}
    
    
    \section{1º Semestre}
    
    \begin{table}[h]
    \caption{Cronograma - 1º Semestre 2018}
    
    \begin{center}
    \centering
    \begin{tabular}{|l|c|c|c|c|c|c|}
    
    \hline
    {\bf Tarefa}                   		& {\bf Jan}& {\bf Fev}& {\bf Mar}& {\bf Abr}& {\bf Mai}& {\bf Jun}\\  
    \hline
    {\bf 1.} Revisão Bibliográfica 		& 	X  & 	      & 	 & 	    & 	       & 	  \\ 
    \hline
    {\bf 1.1} Magnetotelúrico      		& 	X  & 	X     & X        & 	    & 	       & 	  \\ 
    \hline
    {\bf 1.2} Python 3.5           		& 	   &          & X	 &X 	    & 	       & 	  \\ 
    \hline
    {\bf 1.2.1} Linguagem           		& 	   & 	      & X	 &X 	    & 	       & 	  \\ 
    \hline
    {\bf 1.2.2} Kivy 1.10.0           		& 	   & 	      & X	 & X	    & 	       & 	  \\ 
    \hline
    {\bf 1.2.3} Numpy, Scipy, MatplotLib        & 	   & 	      & X	 & X	    & 	       & 	  \\ 
    \hline
    {\bf 1.3} Pacote PROC-MT (INPE)           	& 	   & 	      & 	 & 	    & X	       & X	  \\ 
    \hline
    {\bf 1.3.1} Ats2asc           		& 	   & 	      & 	 & 	    &X 	       & 	  \\ 
    \hline
    {\bf 1.3.2} ProcessamentoZ           	& 	   & 	      & 	 & 	    & 	       & X	  \\ 
    \hline
    {\bf 1.3.3} Tojones           		& 	   & 	      & 	 & 	    & 	       & X	  \\ 
    \hline
   
    
    
    \end{tabular}
 
   
    \end{center}

    \fonte{O autor}  
    \end{table}

    \section{2º Semestre}
    
    \begin{table}[h]
    \caption{Cronograma - 2º Semestre 2018}
    
    \begin{center}
    \centering
    \begin{tabular}{|l|c|c|c|c|c|c|}
    
    \hline
    {\bf Tarefa}                    		 & {\bf Jul}& {\bf Ago}& {\bf Set}& {\bf Out}& {\bf Nov}& {\bf Dez}\\  
    \hline
    {\bf 1.} Construção da Interface Gráfica 	 &X 	    & X	       & 	  & 	     & 	        & 	   \\ 
    \hline
    {\bf 2.} Desenvolvimento dos Scripts         & 	    & 	       & X	  & 	     & 	        & 	   \\ 
    \hline
    {\bf 3.} Fase de testes com Dados Sintéticos & 	    & 	       & 	  & X	     & 	        & 	   \\ 
    \hline
    {\bf 4.} Fase de testes com Dados Reais      & 	    & 	       & 	  & 	     &X	        & 	   \\ 
    \hline
    {\bf 5.} Liberação do Código                 & 	    & 	       & 	  & 	     & 	        & X	   \\ 
    \hline
    
   
    
    
    \end{tabular}
 
   
    \end{center}

    \fonte{O autor}  
    \end{table}

    
    
    \section{1º Semestre}
    
    \begin{table}[h]
    \caption{Cronograma - 1º Semestre 2018}
    
    \begin{center}
    \centering
    \begin{tabular}{|l|c|c|c|c|c|c|}
    
    \hline
    {\bf Tarefa}                   		& {\bf Jan}& {\bf Fev}& {\bf Mar}& {\bf Abr}& {\bf Mai}& {\bf Jun}\\  
    \hline
    {\bf 1.} Revisão Bibliográfica 		& 	X  & 	      & 	 & 	    & 	       & 	  \\ 
    \hline
    {\bf 1.1} Magnetotelúrico      		& 	X  & 	X     & X        & 	    & 	       & 	  \\ 
    \hline
    {\bf 1.2} Python 3.5           		& 	   &          & X	 &X 	    & 	       & 	  \\ 
    \hline
    {\bf 1.2.1} Linguagem           		& 	   & 	      & X	 &X 	    & 	       & 	  \\ 
    \hline
    {\bf 1.2.2} Kivy 1.10.0           		& 	   & 	      & X	 & X	    & 	       & 	  \\ 
    \hline
    {\bf 1.2.3} Numpy, Scipy, MatplotLib        & 	   & 	      & X	 & X	    & 	       & 	  \\ 
    \hline
    {\bf 1.3} Pacote PROC-MT (INPE)           	& 	   & 	      & 	 & 	    & X	       & X	  \\ 
    \hline
    {\bf 1.3.1} Ats2asc           		& 	   & 	      & 	 & 	    &X 	       & 	  \\ 
    \hline
    {\bf 1.3.2} ProcessamentoZ           	& 	   & 	      & 	 & 	    & 	       & X	  \\ 
    \hline
    {\bf 1.3.3} Tojones           		& 	   & 	      & 	 & 	    & 	       & X	  \\ 
    \hline
   
    
    
    \end{tabular}
 
   
    \end{center}

    \fonte{O autor}  
    \end{table}

    \section{2º Semestre}
    
    \begin{table}[h]
    \caption{Cronograma - 2º Semestre 2018}
    
    \begin{center}
    \centering
    \begin{tabular}{|l|c|c|c|c|c|c|}
    
    \hline
    {\bf Tarefa}                    		 & {\bf Jul}& {\bf Ago}& {\bf Set}& {\bf Out}& {\bf Nov}& {\bf Dez}\\  
    \hline
    {\bf 1.} Construção da Interface Gráfica 	 &X 	    & X	       & 	  & 	     & 	        & 	   \\ 
    \hline
    {\bf 2.} Desenvolvimento dos Scripts         & 	    & 	       & X	  & 	     & 	        & 	   \\ 
    \hline
    {\bf 3.} Fase de testes com Dados Sintéticos & 	    & 	       & 	  & X	     & 	        & 	   \\ 
    \hline
    {\bf 4.} Fase de testes com Dados Reais      & 	    & 	       & 	  & 	     &X	        & 	   \\ 
    \hline
    {\bf 5.} Liberação do Código                 & 	    & 	       & 	  & 	     & 	        & X	   \\ 
    \hline
    
   
    
    
    \end{tabular}
 
   
    \end{center}

    \fonte{O autor}  
    \end{table}

    
    
    \section{1º Semestre}
    
    \begin{table}[h]
    \caption{Cronograma - 1º Semestre 2018}
    
    \begin{center}
    \centering
    \begin{tabular}{|l|c|c|c|c|c|c|}
    
    \hline
    {\bf Tarefa}                   		& {\bf Jan}& {\bf Fev}& {\bf Mar}& {\bf Abr}& {\bf Mai}& {\bf Jun}\\  
    \hline
    {\bf 1.} Revisão Bibliográfica 		& 	X  & 	      & 	 & 	    & 	       & 	  \\ 
    \hline
    {\bf 1.1} Magnetotelúrico      		& 	X  & 	X     & X        & 	    & 	       & 	  \\ 
    \hline
    {\bf 1.2} Python 3.5           		& 	   &          & X	 &X 	    & 	       & 	  \\ 
    \hline
    {\bf 1.2.1} Linguagem           		& 	   & 	      & X	 &X 	    & 	       & 	  \\ 
    \hline
    {\bf 1.2.2} Kivy 1.10.0           		& 	   & 	      & X	 & X	    & 	       & 	  \\ 
    \hline
    {\bf 1.2.3} Numpy, Scipy, MatplotLib        & 	   & 	      & X	 & X	    & 	       & 	  \\ 
    \hline
    {\bf 1.3} Pacote PROC-MT (INPE)           	& 	   & 	      & 	 & 	    & X	       & X	  \\ 
    \hline
    {\bf 1.3.1} Ats2asc           		& 	   & 	      & 	 & 	    &X 	       & 	  \\ 
    \hline
    {\bf 1.3.2} ProcessamentoZ           	& 	   & 	      & 	 & 	    & 	       & X	  \\ 
    \hline
    {\bf 1.3.3} Tojones           		& 	   & 	      & 	 & 	    & 	       & X	  \\ 
    \hline
   
    
    
    \end{tabular}
 
   
    \end{center}

    \fonte{O autor}  
    \end{table}

    \section{2º Semestre}
    
    \begin{table}[h]
    \caption{Cronograma - 2º Semestre 2018}
    
    \begin{center}
    \centering
    \begin{tabular}{|l|c|c|c|c|c|c|}
    
    \hline
    {\bf Tarefa}                    		 & {\bf Jul}& {\bf Ago}& {\bf Set}& {\bf Out}& {\bf Nov}& {\bf Dez}\\  
    \hline
    {\bf 1.} Construção da Interface Gráfica 	 &X 	    & X	       & 	  & 	     & 	        & 	   \\ 
    \hline
    {\bf 2.} Desenvolvimento dos Scripts         & 	    & 	       & X	  & 	     & 	        & 	   \\ 
    \hline
    {\bf 3.} Fase de testes com Dados Sintéticos & 	    & 	       & 	  & X	     & 	        & 	   \\ 
    \hline
    {\bf 4.} Fase de testes com Dados Reais      & 	    & 	       & 	  & 	     &X	        & 	   \\ 
    \hline
    {\bf 5.} Liberação do Código                 & 	    & 	       & 	  & 	     & 	        & X	   \\ 
    \hline
    
   
    
    
    \end{tabular}
 
   
    \end{center}

    \fonte{O autor}  
    \end{table}



% CAPITULO 5 ===================================================================================================================
	%\chapter{Revisão Bibliográfica}
\label{cap-revisaobibliografica}

\section{Teoria dos métodos e técnicas}
\label{cap-teometetec}


\section{Trabalhos anteriores aplicados}
\label{cap-trabant}




% CAPITULO 6 ===================================================================================================================
	%\chapter{Materiais e métodos}
\label{cap-matemet}






% CAPITULO 7 ===================================================================================================================
	%\chapter{Planejamento}
\label{cap-planejamento}


\section{Fluxograma}
\label{cap-fluxograma}



\section{Cronograma}
\label{cap-cronograma}






% CAPITULO 8 ===================================================================================================================
	%\chapter{Resultados esperados}
\label{cap-resultadosesperados}




% CAPITULO 9 ===================================================================================================================
	%\chapter{Exemplo de Inserção de Figuras}
\label{cap-Figuras}

% % % % % % % % % % % % % % % % % % % % % % % % % % % % % % % % % % % % % % % % % % % % % %
% Incluir figuras no LaTeX não se dá por apenas copiar e colar, porém o processo é        %
% tão simples quanto. Use o ambiente figure demonstrado abaixo sempre que for necessário  %
% incluir uma imagem. Trocando apenas a localização/nome da imagem. O comando [h] na      %
% frente do ambiente é para que a imagem apareça o mais rápido possível no texto          %
% % % % % % % % % % % % % % % % % % % % % % % % % % % % % % % % % % % % % % % % % % % % % %

\begin{figure}[h]
\centering
\includegraphics[scale=0.5]{TEXTO/IMAGENS/nomedafigura.png}
\caption{}
\fonte{}
\end{figure}


% CAPITULO 10 ===================================================================================================================
	%\chapter{Exemplo de Tabela}
\label{cap-tabs}

% % % % % % % % % % % % % % % % % % % % % % % % % % % % % % % % % % % % % % % % % % % % 
% Para gerar tabelas mais facilmente utilize o site https://www.tablesgenerator.com/  %
% Porém para manter a configuração centralizada da tabela, copie do site apenas a     %
% partir de "\begin{tabular} até o final da tabela, NÃO SUBSTITUINDO o \end{tabular}%}%
% ou substitua porém lembre-se de incluir um %} ao final do comando.                  %
% % % % % % % % % % % % % % % % % % % % % % % % % % % % % % % % % % % % % % % % % % % % 

\begin{table}[h]
    \centering
    \caption{Exemplo de Tabela usando o Latex}
    \label{my-label}
    \resizebox{\textwidth}{!}{%
    \begin{tabular}{@{}ccccl@{}}
        \toprule
        \textbf{Um}  & \textbf{Exemplo}   & \textbf{de}   & \textbf{Tabela} &  \\
        \midrule
        \begin{tabular}[c]{@{}c@{}}
            coluna1     \\
            linha1
        \end{tabular} & 
        
        \begin{tabular}[c]{@{}c@{}}
            coluna2     \\
            linha1
        \end{tabular} & 
        
        \begin{tabular}[c]{@{}c@{}}
            coluna3     \\
            linha1
        \end{tabular} &         x               &  \\
        
        \begin{tabular}[c]{@{}c@{}}
        coluna1     \\
        linha2
        \end{tabular} & {\color[HTML]{000000} \begin{tabular}[c]{@{}c@{}}coluna2\\ linha2\end{tabular}} & \begin{tabular}[c]{@{}c@{}}coluna3\\ linha2\end{tabular}                         & x               &  \\
\begin{tabular}[c]{@{}c@{}}coluna1\\ linha3\end{tabular} & \begin{tabular}[c]{@{}c@{}}coluna2\\ linha3\end{tabular}                        & {\color[HTML]{000000} \begin{tabular}[c]{@{}c@{}}coluna3\\ linha3\end{tabular}}  & x               &  \\ 
\bottomrule
\end{tabular}%
}
\fonte{autor.}
\end{table}



% CAPITULO 11 ===================================================================================================================

% ===================================================================================================================








% BIBLIOGRAFIA ===================================================================================================================
\bibliographystyle{abntex2-alf}
\bibliography{texto/bibliografia}
% ===================================================================================================================







% GLOSSARIO ===================================================================================================================

	%\chapter*{Glossário}
% =============================================================================================================================







% APENDICE ===================================================================================================================

%\appendix

	%\chapter{Nome do Apêndice}
		%Depois do termo ``appendix'', qualquer capítulo aparecerá na forma correta, com o termo ``Apêndice''. Use apêndices quando houver material produzido pelo autor que ajuda no entendimento do trabalho mas que não faz parte do texto principal. Modelos de questionários utilizados, código fonte de programas, partituras completas, provas de teoremas acessórias, etc.
% ===================================================================================================================








% ANEXO ======================================================================================================================
%\annex

	%\chapter{Nome do Anexo}
		%Depois do termo ``annex'', qualquer capítulo aparecerá na forma correta, com o termo ``Anexo'' no título. Use anexos quando se tratar de material não produzido pelo autor, mas necessário no entendimento do trabalho. Por exemplo, definições matemáticas, sintaxe formal de linguagens de programação, trechos de manuais, etc.
% ===============================================================================================================================






\end{document}
% =============================================================================================================================
