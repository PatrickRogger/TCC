\documentclass[csgeo,tcc]{unipampa}                                          
\usepackage[utf8]{inputenc} 
\usepackage[T1]{fontenc}                                    
\usepackage{graphicx}                                       % pacote para importar figuras
\usepackage{chngcntr}                                       % pacote que permite a mudança da contagem das figuras e tabelas
\counterwithout{figure}{chapter}                            % contagem contínua das figuras [Não especificado nas normas, porém de preferência do autor]
\usepackage{times}                                          % pacote para usar fonte Adobe Times
\usepackage{mathptmx}                                       % pacote usar fonte Adobe Times nas fórmulas
\usepackage[alf,abnt-emphasize=bf]{abntex2cite}	           % pacote para usar citações abnt
\usepackage{ragged2e}                                       % pacote para prevenir hifenização e melhorar disposição de objetos
%\hyphenpenalty=10000                                        % penalização de hifenização, força texto justificado
\usepackage{booktabs}                                       % pacote que melhora a qualidade das tabelas geradas pelo LaTeX
\usepackage[table]{xcolor}                                  % pacote para utlização de cores personalizadas
\usepackage[labelsep=endash]{caption}                       % pacote para customização de títulos ou legendas de figuras 
\usepackage{amsmath}
\usepackage{xcolor}
\usepackage{verbatim}

% ATALHOS =================================================================================================================

\newcommand{\rot}[1]{\nabla \times \vec{\textrm{#1}}}
\newcommand{\MT}{magnetotelúrico }                                              % mt -> magnetotelúrico
\newcommand{\citar}[1]{\textcolor{red}{#1}}                                     % marca em vermelho locais para correção
\newcommand{\en}[1]{\textit{#1}}                                                % marca em italico termos em ingles
\newcommand{\ven}[1]{\vec{\textrm{#1}}}                                         % Coloca o vetor em cima da incognita 
\newcommand{\parct}[1]{\dfrac{\partial \vec{\textrm{#1}}}{\partial t}}          % derivada Parcial de t
\newcommand{\parctto}[1]{\dfrac{\partial^2 \vec{\textrm{#1}}}{\partial t^2}}    % derivada Parcial de t




% ATALHOS PARA CODIGO FONTE
\newcommand{\codnum}[1]{\begin{flushright} \normalsize (#1) \end{flushright}}   %numera os scripts
\newcommand{\codbox}[1]{{\fontfamily{pcr}\selectfont #1}}                       %ambiente para escrever codigo fonte. OBS deve estar dentro do ambiente quote
\newcommand{\cc}[1]{\textcolor{gray}{\textit{\# #1}}}                           %cor para comentarios (cinza)
\newcommand{\f}[1]{\textbf{\textcolor{OliveGreen}{#1}}}                         %cor para funcoes (verde)
\newcommand{\cl}[1]{\textbf{\textcolor{BurntOrange}{#1}}}                       %cor para classes (laranja)
\newcommand{\ob}[1]{\textcolor{Blue}{#1}}
\newcommand{\st}[1]{\textcolor{OliveGreen}{#1}}
\newcommand{\ini}{>{}>{}> \, }                                                  % produz >>>
\newcommand{\init}{>{}>{}> \, \, \, \, \,}                                      % produz >>> tabulado 1vez
\newcommand{\initt}{>{}>{}> \, \, \, \, \, \, \, \, \,}                        % produz >>> tabulado 2vez
\newcommand{\inittt}{>{}>{}> \, \, \, \, \, \, \, \, \, \, \, \, \,}            % produz >>> tabulado 3vez






% =========================================================================================================================





% TITULO, AUTOR, ORIENTADOR =============================================================================================== 

% TITULO
	\title{Desenvolvimento de Software livre para processamentos de Dados Magnetotelúricos}                              

% AUTOR
	\author{Garcia}{Patrick Rogger}                                    

% ORIENTAÇÃO
	\advisor[Prof.~Dr.]{Oliveira}{Vinicius Abreu de}                        
	\coadvisor[Prof\textsuperscript{a}.~Dr.\textsuperscript{a}.]{Matos}{Andréa Cristina Lima dos Santos}   
	\renewcommand{\coadvisorname}{Co-orientadora}
% =========================================================================================================================






% BANCA ===========================================================================================================

% BIBLIOTECA
	\cutter{} 
% MEMBRO 1
	\banca[Prof.~Dr.]{SOBRENOME}{NOME}			                
	\inst{Universidade Federal do Pampa}					    

% MEMBRO 2
	\banca[Prof.~Dr.]{SOBRENOME}{NOME}                           
	\inst{Universidade Federal do Pampa}	                    

% DATA DA DEFERSA
	\defesa{DIA}{MÊS}{ANO}                                    
% ====================================================================================================================








% DATA E LOCAL =======================================================================================================
% SE PRECISAO MUDAR A DATA DO DOCUMENTO
	%\date{MÊS}{ANO}
% LOCAL
	\location{Caçapava do Sul}{RS}                              
% =====================================================================================================================








% PALAVRAS CHAVES =====================================================================================================
	\keyword{Kivy}	
	\keyword{Magnetotelúrico}
	\keyword{Python3}
	\keyword{Software Livre}
% ======================================================================================================================








% CORRIGIR TEXTO =======================================================================================================
	\sloppy                                                       
% ======================================================================================================================








% DOCUMENTO ============================================================================================================
\begin{document}
\maketitle


% % capa
% CAPA

% E aqui vai a parte principal:
\newpage
    \begin{center}
      {\bf
	  \large UNIVERSIDADE FEDERAL DO PAMPA

	      \vspace{150pt}
	  
	      PATRICK ROGGER GARCIA
	  
	  
	  \vfill
	      \large DESENVOLVIMENTO DE SOFTWARE LIVRE PARA PROCESSAMENTOS DE DADOS MAGNETOTELÚRICOS
	  \vfill}
      

	  {\bf
	      ORIENTADOR:}
	      VINICIUS ABREU DE OLIVEIRA
	   
	  {\bf
	      CO-ORIENTADORA:}
	      ANDRÉA CRISTINA LIMA DOS SANTOS MATOS 
	      \vspace{200pt}
	  
      \bf{
	  Caçapava do Sul
 
	  2018}
\end{center}


\newpage
    \begin{center}
      {\bf
	  \large UNIVERSIDADE FEDERAL DO PAMPA

	      \vspace{150pt}
	  
	      PATRICK ROGGER GARCIA
	  
	  
	  \vfill
	      \large DESENVOLVIMENTO DE SOFTWARE LIVRE PARA PROCESSAMENTOS DE DADOS MAGNETOTELÚRICOS
	  \vfill}
      

	  
	  
      \bf{
	  Caçapava do Sul
 
	  2018}
\end{center}




% DEDICATORIA ==========================================================================================================

	%\begin{dedicatoria} 
	%	
	%\end{dedicatoria}
% =======================================================================================================================








% AGRADECIMENTOS ========================================================================================================
	
	%\chapter*{Agradecimento}
	% 
% ========================================================================================================================







% EPIGRADE ================================================================================================================
	%\begin{epigrafe}
	%	``Moça bonita, moça bem feita.''\\
	%	--- Sr. Madruga
	%\end{epigrafe}
% ========================================================================================================================








% RESUMO ===============================================================================================================

%	\begin{abstract} 
%		Escreva aqui seu resumo em português. 		
%	\end{abstract}
% =========================================================================================================================







% RESUMO EM INGLES ========================================================================================================
%	\begin{englishabstract}
%		Escreva aqui seu resumo em língua estrangeira.
%	\end{englishabstract}

% ==========================================================================================================================







% LISTA DE FIGURA E TABELA =================================================================================================

\listoffigures
\listoftables
% ==========================================================================================================================






%  LISTA DE TEOREMAS ==========================================================================================================================
% Listas de definições e teoremas, para quem usar o pacote formais, para trabalhos que possuam definições formais e teoremas

%\listofdefinitions
%\listoftheorems
% ==========================================================================================================================






% LISTA DE ABREVIATURAS ====================================================================================================

\begin{listofabbrv}{UNIPAMPA}
        
        \item[API]          {Application Programming Interface}
        \item[ASCII]        {American Standard Code for Information Interchange}
        \item[CPU]          {Central Processing Unit}
        \item[CWI]          {Centrum Wishunde \& Informatica}
        \item[GEOMA]        {Grupo de Geomagnetismo}
        \item[INPE]         {Instituto Nacional de Pesquisas Espaciais}
        \item[MT]           {Magnetotelúrico}
        \item[1D]           {Uma Dimensão}
        \item[2D]           {Duas Dimensões}
        \item[3D]           {Três Dimensões}
        \item[TE]           {Transversal Elétrico}
        \item[TM]           {Transversal Magnético}
        
\end{listofabbrv}
% =============================================================================================================================







% LISTA DE SIMBOLOS ===========================================================================================================

\begin{listofsymbols}{$\alpha\beta\pi\omega$}
       \item[$\sigma$]                  Condutividade Elétrica
       \item[$\rho$]                    Resistividade Elétrica
       \item[$V$]                       Diferença de Potencial
       \item[$i$]                       Corrente Elétrica
       \item[$R$]                       Resistência Elétrica
       \item[$A$]                       Área
       \item[$L$]                       Comprimento
       \item[$\nabla \times$]           Rotacional
       \item[$\nabla \cdot$]            Divergente
       \item[$\ven{E}$]                 Campo Elétrico
       \item[$\ven{H}$]                 Campo Magnetizante
       \item[$\ven{B}$]                 Campo Magnético
       \item[$\ven{J}$]                 Densidade de Corrente
       \item[$\ven{D}$]                 Campo de Deslocamento Elétrico
       \item[$\rho_f$]                  Densidade de Carga
       \item[$t$]                       Tempo
       \item[$\mu$]                     Permissividade Magnética
       \item[$\varepsilon$]             Permissividade Elétrica
       
       
\end{listofsymbols}
% =============================================================================================================================







% SUMARIO ==================================================================================================================
\tableofcontents
% ==========================================================================================================================







% CORPO DO TEXTO ===============================================================================================================
\inputencoding{utf8}                                %% Arquivos em outra codificação: UTF-8

% CAPITULO 1 ===================================================================================================================
	% INTRODUÇÃO


\chapter{Introdução}
    \label{cap-introducao}
    
	Apoiado nas leis de Maxwell o método MT (Magnetotelúrico) usa a Terra como 
	um condutor ôhmico e as variações do seu campo 
	magnético promovido por ventos solares \cite{parkinson93} e tempestades equatoriais 
	que interagem com a ionosfera para investigar as 
	estruturas internas da Terra e litologias rasas. 
	
	
	No Brasil o uso do método MT é insipiente, restrito ao meio acadêmico e pouco
	utilizado na indústria, porém, pode ser bem aplicado na prospecção de 
	hidrocarbonetos, sendo a sua resolução melhor que a magnetometria
	e gravimetria, também em estudos crustais
	apoiando a sismologia devido sua grande profundidade de investigação, mas o 
	alto custo de processamento e a falta de \textit{softwares} para trabalhar com os 
	dados tem sido algumas das causas do fraco uso.
	
	
	Esse trabalho foi inicialmente pensado para tornar o MT mais difundido, 
	desenvolvento um \textit{software} com interface gráfica e 
	distribuição livre. Assim o projeto nasceu 
	com esse propósito, compreendendo o processamento de dados
	MT desde a coleta até a primeira visualização dos dados, como: escolha 
	de bandas, plotagem de pseudo-secções em função de resistividade e fase, além de fazer tratamentos estatísticos e processamento robusto 
	proposto por \citeauthor{egbert97} \citeyearpar{egbert97}.
	
	
	O programa será construído usando a linguagem Python \cite{python36} 
	e a construção da interface gráfica será desenvolvida usando a API
	Kivy \cite{kivy110} dentre outros pacotes. A escolha por essa linguagem foi a vasta quantidade de pacotes,  o crescente 
	número de pessoas implementando e a facilidade da construção do código.






% CAPITULO 2 ===================================================================================================================
	
\chapter{Objetivos}
\label{cap-objetivos}

\section{Objetivos Gerais}
\label{cap-objetivos gerais}

    
    O objeto de estudo desse trabalho é o desenvolvimento de um \en{Software} Livre, em \en{Python} para tratamentos e processamentos de dados \MT.
    
\section{Objetivos Específicos}
\label{cap-objetivos especificos}

    O MT é um método geofísico que trabalha com propriedades eletromagnéticas, essa característica torna o processamento dos dados extremamente trabalhoso e com alto custo computacional, o trabalho então propõe:
    
    \begin{itemize}
     \item Criação de novos algoritmos escritos em Python.
     \item Otimização dos códigos frente as novas tecnologias.
     \item Obter secções lito-geofísicas de resistividade para a região de estudo.
     \item Comparar os resultados obtidos com trabalhos já consolidados\footnote{\cite{tese_andrea} e \cite{alane}}.
    \end{itemize}

	\chapter{Justificativa \,\,\,\,\, OK}
\label{cap-justificativa}

    Esse trabalho foi pensado para auxiliar a expansão do \MT, hoje existem vários programas proprietários para processamento de dados \MT, mas, são programas caros e que não permitem ver como são feitos os fluxos de processamentos dentro do programa. O processamento hoje usado na maioria dos trabalhos acadêmicos ou não usam como base os algoritmos propostos por \citeauthor{egbert97} para a primeira fase de processamento e a rotina To Jones \cite{egbert97} para compilação dos dados em arquivos manipuláveis.
    
    Esses processos hoje são feitos através de linhas de comando unindo diversos programas separados, esse processo acaba sendo muito instável provocando diversos tipos de erros e não prevendo outros, também força a instalação de vários pacotes separados que muitas vezes são compilados e instalados manualmente, afastam uma pessoa leiga no assunto a utilizar.
    
    O programa então pretende ser escrito em apenas uma linguagem e utilizar pacotes nativamente compatíveis isso preve erros de compatibilidades e torna a manutenção do código mais fácil uma vez que os códigos não serão compilados, outra forma de beneficiar a utilização do programa será a construção de uma interface gráfica para todos os processos, tornando então mais fácil e prevenindo erros provocados pelo usuário.    
    
    

	\chapter{Fundamentos do Método \MT}
    \label{cap-fundamentosMT}    
    
    O método \MT proposto por \citeauthor{tikhonov50} \citeyearpar{tikhonov50} e
    \citeauthor{cagniard53} \citeyearpar{cagniard53}, usa as propriedades
    eletromagnéticas para estudar a distribuição de resistividade elétrica na crosta, 
    podendo variar a sua investigação de dezenas de metros a dezenas de 
    quilômetros.
    
    \section{Origem das Correntes Telúricas}
    
    As flutuações no campo magnético terrestre geram campos elétricos na alta atmosfera que induzem correntes magnéticas.
    
    As ondas eletromagnéticas penetram no interior da Terra na forma de ondas planas ortogonais que induzem novas correntes chamadas de corrente telúricas que trazem informações das características físicas das litologias \cite{tikhonov50} e \cite{cagniard53}.
    
    Uma das características é a modulação da frequência, causada por diferentes tipos de rochas e estruturas, esse fenômeno é diretamente 
	relacionado a resistividade elétrica do meio. As frequências das ondas são baixas variando de 1 $mHz$ à 10 $kHz$.
	
	Ondas com frequências menores que 1 $Hz$ tem origem nos ventos solares que interagem como o campo magnético terrestre, já ondas com frequências maiores de 1 $Hz$ são provocadas por tempestades equatoriais. Para o estudo do \MT são feitas as seguintes suposições:
    
    \begin{enumerate}
	    \item Ondas geradas na ionosfera, distantes o suficientes, penetram ortogonais à superfície da Terra.	    
	    \item A Terra se comporta como um condutor ôhmico.
	    \item A Terra é considerada um semi-espaço isotrópico.
	\end{enumerate}   
    
    
%RESISTIVIDADE ==============================================================================================================================================
    \section{Resistividade dos Materiais}
    
    Para o \MT a propriedade investigação e contraste é a condutividade elétrica ($\sigma$) ou resistividade elétrica ($\rho$) sendo essa o inverso da primeira.
    A resistividade elétrica é uma propriedade particular de um determinado material, ou seja, a partir de uma resistividade elétrica podemos estimar a qual material ela pertence\footnote{Para os meios geológicos essa propriedade é representada por um intervalo de valores, devido as complexidades químicas e físicas das diferentes litologias.}.
    
    Em 1827, Georg Ohm verificou de forma empírica que aplicando uma diferença de potencial em um material esse gera uma resistência a passagem de corrente, essa relação é chamada de lei de Ohm (equação \ref{lei_de_ohm})\cite{eletromag8hayt}.
    
    \begin{equation}
        \label{lei_de_ohm}
        V = R i
    \end{equation}
    
    Onde $V$ é a diferença de potencial (Volts - V), $i$ é a corrente (Ampère - A) e $R$ é a resistência (Ohms - $\Omega$), materiais que obedecem essa lei são chamados de materiais ômicos.
    
    A Terra pode ser considerada como um material ôhmico. No entanto para a investigação geofísica a resistência não é uma propriedade viável, visto  que depende muito da geometria do problema, assim foi proposto a resistividade elétrica, onde, um mesmo material terá a sua resistividade elétrica igual independente da geometria.
    
    A figura \ref{fig_resistividade} mostra um circuito para se obter a resistividade elétrica, sendo A a área [$m^2$], R a resistência [$\Omega$], L o comprimento [$m$] e $\rho$ a resistividade elétrica dada em $\Omega m$. 
    
    \begin{equation}
        \label{resistividade}
        \rho = \dfrac{R A}{L}\, \, \, ;\, \, \, \, \, \, \, \, \, \, \, \, \, \,  R = \dfrac{V}{i}
    \end{equation}
    
    
    \begin{figure}[h]
        \centering
        \caption{Arranjo para medir a resistividade elétrica ($\rho$) de um material}
        \centerline{\includegraphics[width=4cm]{texto/fig/resisti_telford.png}}
        \fonte{Adaptado \citeauthor{telford}, \citeyearpar{telford}}
        \label{fig_resistividade}
    \end{figure}
    
    A figura \ref{tabela_resistividade} mostra a distribuição de resistividade elétrica para diversos materiais geológicos. 
    Portanto podemos identificar a partir de um contexto geológico quais litologias pertence a cada resistividade elétrica encontrada.
    
    Por exemplo, uma litologia que tenha resistividade elétrica em torno de  $100 \, \Omega m$ e outra com $3000 \, \Omega m $ pode ser caracterizada como um arenito e uma rocha ígnea respectivamente. 
    
    \begin{figure}[h]
        \centering
        \caption{Resistividade Elétrica dos Materiais Geológicos}
        \centerline{\includegraphics[width=14cm]{texto/fig/resistividade_tabela.png}}
        \fonte{Adaptado \citeauthor{eletromag_met}, \citeyearpar{eletromag_met}}
        \label{tabela_resistividade}
    \end{figure}
% FIM RESISTIVIDADE
%===========================================================================================================================================================    
    
    
    
% Eletromagneticos
%===================================================================================================================================================
    \section{Fundamentos Teóricos dos Métodos Eletromagnéticos}
        Usando as leis de Maxwell \cite{eletromag8hayt} podemos medir os campos elétricos e magnéticos e a partir deles estimar a resistividade elétrica dos meios litológicos em sub-superfície.
	
        Os campos podem ser descritos pelas seguintes equações\footnote{Para cargas e correntes livres
        (macroscópica)}:
            \begin{equation}
                \label{rot_elet_max}
                \nabla \times \vec{\textrm{E}}=-\frac{\partial \vec{\textrm{B}}}{\partial t} 
            \end{equation}
            \begin{equation}
                \label{rot_mag_max}
                \nabla \times \vec{\textrm{H}} = \vec{\textrm{J}} + \frac{\partial \vec{\textrm{D}}}{\partial t}
            \end{equation}
            \begin{equation}
                \nabla \cdot \vec{\textrm{B}} = 0
            \end{equation}
            \begin{equation}
                \label{div_d}
                \nabla \cdot \vec{\textrm{D}} = \rho_f
            \end{equation}
            
            \noindent Onde,
            
            {\footnotesize \noindent $\vec{\textrm{E}}$ $\rightarrow$ Campo Elétrico [$V/m$]
	    
            \noindent $\vec{\textrm{B}}$ $\rightarrow$ Campo Magnético [$T$]
	    
            \noindent $\vec{\textrm{H}}$ $\rightarrow$ Campo Magnetizante [$A/m$]
	    
            \noindent $\vec{\textrm{J}}$ $\rightarrow$ Densidade de Corrente [$A/m^2$]
	    
            \noindent $\vec{\textrm{D}}$ $\rightarrow$ Campo de Deslocamento Elétrico [$C/m^2$]
	    
            \noindent $\rho_f$ $\rightarrow$ Densidade de Carga [$C/m^3$]
	    
            \noindent $t$ $\rightarrow$ Tempo [$s$]}

            Obedecendo as relações de contorno para um meio isotrópico temos as seguintes
            relações (equações constitutivas):
            \begin{equation}
                \label{con_B}
                \vec{\textrm{B}} = \mu \vec{\textrm{H}}
            \end{equation}
            \begin{equation}
                \label{con_D}
                \vec{\textrm{D}} = \varepsilon  \vec{\textrm{E}}
            \end{equation}
            \begin{equation}
                \label{con_J}
                \vec{\textrm{J}} = \sigma \vec{\textrm{E}}
            \end{equation}
	    
            {\footnotesize \noindent $\mu$ $\rightarrow$ Permeabilidade Magnética [$H/m$]
	    
            \noindent $\varepsilon$ $\rightarrow$ Permissividade Elétrica [$F/m$]
	    
            \noindent $\sigma$ $\rightarrow$ Condutividade Elétrica [$S/m$]}
	    
            Cada escalar das equações anteriores são características que dependem do meio em que a onda se propaga.
	    
            Para a crosta $\mu = 1,2566\textrm{x}10^{-6} H/m$ e $\varepsilon = 8,85
            \textrm{x}10^{-12} F/m$; esses parâmetros funcionam como tensores em um meio
            anisotrópico que variam em função do tempo.
            
            Considerando para os 
            trabalhos de investigação o meio supõe-se ser isotrópico, assim, 
            tornando estáticos os tensores.
	
            Através das propriedades dos meios isotrópicos podemos reescrever as equações \ref{rot_elet_max} e \ref{rot_mag_max} usando as equações constitutivas \ref{con_B}, \ref{con_D} e \ref{con_J}.
            
            \begin{equation}
                \label{rot_elet_con}
                \rot{E} = - \mu \parct{H}
            \end{equation}
            
            \begin{equation}
                \label{rot_mag_con}
                \rot{H} = \sigma \ven{E} + \varepsilon \parct{E}
            \end{equation}
            
            Derivando a equação \ref{rot_mag_con} no tempo, multiplicando por $\mu$ e usando a equação \ref{rot_elet_con} temos:
            
            
\begin{align*}
\dfrac{\partial (\nabla \times \vec{\textrm{H}})}{\partial t} & = \dfrac{\partial (\sigma \ven{E})}{\partial t} + \dfrac{\partial}{\partial t} \bigg(\varepsilon \parct{E}\bigg)\\[10pt]
             \dfrac{\nabla \times \partial \ven{H}}{\partial t} & = \sigma \parct{E} + \varepsilon \parctto{E} \\[10pt]
             \mu \dfrac{\nabla \times \partial \ven{H}}{\partial t} & = \mu \sigma \parct{E} + \mu \varepsilon \parctto{E} \\[10pt]
             -\dfrac{\mu}{\mu} \nabla \times \rot{E} & = \mu \sigma \parct{E} + \mu \varepsilon \parctto{E}
             \end{align*}
             \begin{equation}
                \label{menos_rot_e}
              - \nabla \times \rot{E} = \mu \sigma \parct{E} + \mu \varepsilon \parctto{E}
             \end{equation}
                
            Usando a identidade vetorial:
            
            \begin{equation}
             \nabla \times \rot{A} = -\nabla^2 \ven{A} + \nabla(\nabla \cdot \ven{A})
            \end{equation}
            
            Podemos reescrever a equação \ref{div_d} considerando, que para meios homogêneos e isotrópicos não há troca de carga entre ele e a densidade de carga, $\rho_f$, é zero assim:
            
            \begin{equation}
             \nabla \cdot \ven{E} = 0
            \end{equation}
            
            Portanto:
            
            \begin{equation}
             \label{rot_rot_e}
             \nabla \times \rot{E} = -\nabla^2 \ven{E} + \nabla(\nabla \cdot \ven{E});\,\,\,\,\,\,\, \textrm{onde}\,\,\,\,\,\,\,   \nabla(\nabla \cdot \ven{E})=0
            \end{equation}
            
            Substituindo [\ref{rot_rot_e}] em [\ref{menos_rot_e}] temos:
            
            \begin{equation}
             \nabla^2 \ven{E} - \mu \sigma \parct{E} - \mu \varepsilon \parctto{E} = 0
            \end{equation}
            
            De forma análoga podemos verificar que:
            
            \begin{equation}
             \nabla^2 \ven{H} - \mu \sigma \parct{H} - \mu \varepsilon \parctto{H} = 0
            \end{equation}

            Seguindo a dedução das equações como demostrado no trabalho de \citeauthor{didana2010} em \citeyearpar{didana2010}, podemos verificar que:
            
            \begin{equation}
            \label{Ex_AB}
             \textrm{E}_x = A e^{-\imath k z} + \textrm{B} e^{\imath k z}
            \end{equation}
            
            \begin{equation}
            \label{Hy_AB}
             \textrm{H}_y = \dfrac{k}{\omega \mu_0} (A e^{-\imath k z} + \textrm{B} e^{\imath k z})
            \end{equation}

            Onde $k^2 = \imath \omega \mu_0 \sigma$.

 
    
    
    
% MT
%===================================================================================================================
    \section{Resposta do Método Magnetotelúrico}
        \subsection{Impedância Eletromagnética}
        Outro conceito importante é o tensor Impedância, ele é descrito como uma
	    relação entre os campos elétricos e magnéticos, análogo a Lei de Ohm \cite{eletromag8hayt}
	    que apresenta resistência a passagem de corrente.
	    
	    \begin{equation}
		\left (\begin{array}{c}
		 \textrm{E}_x\\
		 \textrm{E}_y
		\end{array}\right)
		=
		\left (\begin{array}{cc}
		 \textrm{Z}_{xx} & \textrm{Z}_{xy}\\
		 \textrm{Z}_{yx} & \textrm{Z}_{yy}
		\end{array}\right) \left (\begin{array}{c}
		 \textrm{H}_x\\
		 \textrm{H}_y
		\end{array}\right)
	    \end{equation}
	    
	    \begin{equation}
	    \label{Ex_Z}
	     \textrm{E}_x (\omega)=\textrm{Z}_{xx}(\omega) \textrm{H}_{x}(\omega) + \textrm{Z}_{xy}(\omega) \textrm{H}_{y}(\omega)
	    \end{equation}
	    \begin{equation}
	    \label{Ey_Z}
	     \textrm{E}_y (\omega)=\textrm{Z}_{yx}(\omega) \textrm{H}_{x}(\omega) + \textrm{Z}_{yy}(\omega) \textrm{H}_{y}(\omega)
	    \end{equation}  
    
    
    \section{Modelo de Dimensões MT}
    
        
        \subsection{Terra 1D}
        
        Para o modelo de Terra 1D considera-se que a resistividade elétrica varia apenas em uma direção, ou seja, a resistividade elétrica varia com a profundidade.        
        A matriz impedância para esse modelo tem a sua diagonal principal igual a zero.
        \begin{equation}
         \textrm{Z}_{1 D} = \left ( \begin{array}{cc}
                                     0 & \textrm{Z}_{xy} \\
                                    -\textrm{Z}_{yx} & 0
                                     \end{array}  \right)
        \end{equation}
        
        Isso significa que a resistividade elétrica nas duas direções não iguais porem a fase entre elas são opostas.
        
        Substituindo [\ref{Ex_AB}] e [\ref{Hy_AB}] na equação \ref{Ex_Z}, obtemos:
        
        \begin{equation}
        \label{Zxy}
         \textrm{Z}_{xy}(\omega) = \dfrac{\textrm{E}_x(\omega)}{\textrm{H}_y(\omega)} = \dfrac{\omega \mu_0}{k}
        \end{equation}
        
        Elevando o módulo ao quadrado da equação \ref{Zxy}, temos:
        
        \begin{equation}
         \left | \dfrac{\textrm{E}_x(\omega)}{\textrm{H}_y(\omega)} \right | ^2 = \left | \dfrac{\omega \mu_0}{k} \right | ^2 = \dfrac{\omega \mu_0}{\sigma}
        \end{equation}
        
        Portanto:
        
        \begin{equation}
         \dfrac{1}{\sigma} = \dfrac{1}{\omega \mu_0} \left | \dfrac{\textrm{E}_x(\omega)}{\textrm{H}_y(\omega)} \right | ^2 = \rho
        \end{equation}
        
        A onda sofre influência de todas as camadas que percorre isso significa que a resistividade elétrica é classificada como aparente onde ela representa a resistividade elétrica de todo o pacote, assim:
        
        \begin{equation}
         \rho_a = \dfrac{1}{\omega \mu_0} \left | \textrm{Z} \right | ^2
        \end{equation}
        
        A fase do tensor impedância é definido como sendo o arco tangente da parte imaginária sobre a parte real na matriz complexa do tensor.
        
        \begin{equation}
         \phi = \textrm{arctg} \left ( \dfrac{\textrm{Im \, Z}}{\textrm{Re \, Z}} \right )
        \end{equation}
        
        A equação \ref{rela_prof_periodo} mostra a relação entre a profundidade
	($\delta_f[m])$, frequência ($f[Hz]$) e a resistividade aparente ($\rho_a[\Omega.m]$), essa 
	profundidade é chamada de \textit{skin-depth} \cite{eletromag8hayt}, e decai com o inverso de $e$.
	\begin{align*}
	 \delta_\omega & = \dfrac{1}{\textrm{Re}(k)} \\
	  \delta_\omega & = \dfrac{1}{\textrm{Re} \left (\sqrt{\imath \omega \mu_0 \sigma} \right)}  \\
	  \delta_\omega & = \sqrt{ \left (\dfrac{2}{\omega \mu_0 \sigma} \right )}
	\end{align*}

	
	\begin{equation}
	 \label{rela_prof_periodo}
	 \delta_\omega = \sqrt{\frac{2}{\omega \mu \sigma}} \longrightarrow \delta_f \approx 500  \sqrt{\frac{\rho_a}{f}}
	\end{equation}
	
	Essa relação mostra que para uma mesma profundidade variando à resistividade
	aparente a frequência é alterada.
        
   
        \subsection{Terra 2D}
        
        O modelo de Terra 2D é caracterizado pelo contato vertical entre dois meios de diferentes resistividades elétricas. Se o contato é
	    paralelo ao eixo $x$ então é definido a direção do \textit{strike} no eixo $x$, a direção deve ser paralela ao plano de contato,
	    ou seja, onde a condutividade elétrica é constante.
	    
	    \begin{figure}[H]
	        \caption{Modelo de Terra 2D para a resistividade elétrica variando na direção $y$}
	        \begin{center}
	        \includegraphics[width=12cm]{texto/fig/Tm_Te.png} 
	        \end{center}
		\fonte{Adaptado \cite{didana2010}}
		\label{fig_strike}
	    \end{figure}
	    Devido a essa diferença entre as resistividades elétricas polarizamos os campos em TE (Transversal Elétrico) e TM (Transversal Magnético).
	    Para esse modelo temos o tensor impedância como:
	    \begin{equation}
	     \textrm{Z}_{2D} = \left (\begin{array}{cc}
	                               0 & \textrm{Z}_{xy} \\
	                               \textrm{Z}_{yx} & 0
	                              \end{array} \right)
	    \end{equation}
	    Assim cada polarização pode ser escrita como:
	    \begin{equation}
	     \textrm{TE} = \left \{ \begin{array}{l}
	            \dfrac{\partial \textrm{E}_x}{\partial y} = \dfrac{\partial \textrm{B}_z}{\partial t} = -i\omega \textrm{B}_z \\[10pt]
	           \dfrac{\partial \textrm{E}_x}{\partial z} = \dfrac{\partial \textrm{B}_y}{\partial t} = i\omega \textrm{B}_y \\[10pt]
	           \dfrac{\partial \textrm{B}_z}{\partial y} - \dfrac{\partial \textrm{B}_y}{\partial z} = \mu \sigma \textrm{E}_x 
	           \end{array} \right.
	    \end{equation}
	    \begin{equation}
	     \textrm{TM} = \left \{ \begin{array}{l}
	            \dfrac{\partial \textrm{B}_x}{\partial y} = \mu \sigma \textrm{E}_z \\[10pt]
	           -\dfrac{\partial \textrm{B}_x}{\partial z} = \mu \sigma \textrm{E}_y \\[10pt]
	           \dfrac{\partial \textrm{E}_z}{\partial y} - \dfrac{\partial \textrm{E}_y}{\partial z} = i \omega \textrm{B}_x 
	           \end{array} \right.
	    \end{equation}
        
        
        \subsection{Terra 3D}
        Na maioria das condições geológicas o modelo se comporta como 3D, isso implica que a 
	    condutividade elétrica varia ao longo das três direções ($\sigma = \sigma_{x,y,z}$).
	    
	    A matriz do tensor impedância é então calculada com todos os termos, sem nenhum 0.    

%=======================================================================================================================================================	    
    \chapter{\en{Softwares} Utilizados em Dados \MT}
    
    \citar{Comentar sobre como vinha sendo tratado os dados até então}
    \subsection{Pacotes de Processamentos do grupo Geoma - INPE}
    
        O grupo Geoma do INPE (Institudo Nacional de Pesquisas Espaciais) oferece um treinamento para processamento do \MT para alunos e colaboradores.
        
        Os \en{scripts} oferecidos para o processamento MT foram desenvolvidos pelo Dr. Marcelo Banik de Pádua obtidos diretamente pelo autor.
        
        A natureza dos pacotes são \en{scripts} em \en{Shell} que utilizam o programa GMT \cite{gmt} para plotagem dos gráficos e o núcleo de processamento dos dados são as rotinas propostas por \citeauthor{egbert97} e nos trabalhos de Alan Jones, disponíveis no site \citeauthor{mtnet}
    

    
    
%========================================================================================================================================================    
    \chapter{Proposta de \en{Software} Livre Integrador}
    
        O desenvolvimento do \textit{software} foi baseado na filosofia de \textit{Software Livre} \cite{soft_free} onde o código fonte será liberado e distribuído para a comunidade geofísica. A linguagem base escolhida para o projeto foi o Python, visto as vastas bibliotecas para trabalhar com dados científicos e a simplicidade da implementação do código.  
        
        \subsection{Linguagem PYTHON}
            \label{lim_python}
            
            Criada nos anos 80 por Guido Van Rossum no CWI (\en{Centrum Wishunde \& Informatica}) em Amsterdã, Holanda a linguagem Python foi idealizada no grupo de desenvolvimento da linguagem ABC do CWI, onde rapidamente começou a se destacar.
            
            Na década de 90 foi criada a \en{Python Software Activity} que começou a cuidar dos interesses da linguagem, nesse período apenas o criador Guido Van tomava as decisões e cuidava do desenvolvedor. Finalmente em 2001 é fundada a \en{Python Software Foundation} que mantém a linguagem e todos os direitos sobre ela \cite{python36}.  
            
            Python é uma linguagem de alto nível\footnote{Esta relacionada a abstração da linguagem, alto nível significa mais longe da linguagem de código de máquina} onde seu código deve ser organizado favorecendo a interpretação e sendo ao mesmo tempo simples.
            
            Exemplos de código Python:

            Mostrar conteúdo na Tela:
            Como comentado, o código tem fácil leitura, para imprimir um conteúdo na tela, por exemplo, 
            podemos simplesmente usar o comando \verb|print|, aproximando muito da língua inglesa.  
            \begin{quote}
             \codbox{\ini   \cc{Comentários}                  \\
                     \ini   \f{print} ('Hello World')          \\
                     Hello World
                     }                                          \codnum{\ref{lim_python}.1}

            \end{quote}
            
            Operações Matemáticas:
            
            As variáveis no código não precisam ser declaradas para um 
            tipo específico (Ex.: \textit{float, int, string}), como em outras linguagens, o que deixa o código mais fluido. 
            \begin{quote}
            
             \codbox{\ini a = 2                               \\
                     \ini b = 5                               \\
                     \ini \f{print}(a + b)                    \\
                     7                                        \\
                     %\ini                                    \\
                     \ini \f{print}(b / a)                    \\
                     2.5                                      \\
                     %\ini \cl{class} Tela(App):              \\
                     %\init  \cl{return} Tela
                     }                                          \codnum{\ref{lim_python}.2}
            \end{quote}
            
            Importando Módulos:
            
            Módulos são estruturas que podemos importar objetos de um código a outro,
            no script \ref{lim_python}.3 importamos o valor de $\pi$ que esta contido na variável \verb|pi| dentro do pacote \verb|math|. 
            
            \begin{quote}
             \codbox{\ini \cl{import} math                    \\
                     \ini                                     \\
                     \ini pi = math.pi                        \\
                     \ini \f{print}(pi)                       \\
                     3.141592653589793
                    }                                           \codnum{\ref{lim_python}.3}
            \end{quote}
            
        \subsection{Módulos e Pacotes}
            
            A vasta quantidade de pacotes de terceiros para Python é o que faz a linguagem tão rica.
            De fato, os 
            pacotes facilitam a implementação do código, por exemplo, se for preciso calcular o espectro de 
            frequência de um conjunto de dados, não será necessário implementar todo o algoritmo para efetuar o cálculo, resolver as integrais e assim por diante, mas sim podemos utilizar o pacote \verb|scipy| e importarmos a função \verb|fftpack| que já foi implementada e executar em nosso código, esse processo economiza tempo em desenvolvimento.           
            
            \subsubsection{Kivy}
            
            
            \label{lim_kivy}
            Kivy é um \textit{framework} criado em 2010 pela KIVY ORGANIZATION \cite{kivy} e \textit{Open Source} para o desenvolvimento de interfaces gráficas, a escolha dessa interface foi a alta compatibilidade entre sistemas operacionais e todo o processamento nativo para desenhar a tela é feita no chip gráfico liberando então mais processamento pela CPU.
            
            Kivy também é uma linguagem de programação que permite a criação da interface de forma mais fácil, similar ao QT \cite{qt} ela usa uma linguagem de marcação e indentada onde as propriedades dos \textit{widgets} (Objetos interativos com o usuário) são adicionadas colocando-as a baixo e com espaçamento de 4 espaços do \textit{widget}. 
                        
            Exemplo do Kivy dentro do código Python:
            \begin{quote}
             \codbox{\ini \cl{from} kivy.app \cl{import} App                      \\
                     \ini \cl{from} kivy.uix.button \cl{import} Button            \\
                     \ini                                                         \\
                     \ini      \cl{class} Test(App):                              \\
                     \init          \cl{def} build(self):                         \\
                     \init \,\,\,\,\,\,     \cl{return} Button(\ob{text}=\st{'Hello World')} \\
                     \ini                                                         \\
                     \ini Test().run()                                            
             }                                                                    \codnum{\ref{lim_kivy}.1}
            \end{quote}
            
            \begin{figure}[H]
                \caption{Exemplo de janela com Kivy implementada somente com código Python}
                \begin{center}
                    \includegraphics[width=7cm]{texto/fig/hello_world_kivy.png} 
                \end{center}
                \fonte{O Autor, 2018}
                \label{janela_kivy} 
            \end{figure}


            
            \subsubsection{SciPy, MatPlotLib, NumPy}
            \label{lim_scipy}
            
            SciPy é um ecossistema de ferramentas para processamento de dados científicos contando com ferramentes de manipulação de matrizes, plotagem de gráficos, interpolação dentro outras ferramentas \cite{scipy}.
            
            O ecossistema é de código aberto e as principais ferramentas são: NumPy para trabalhos com vetores e matrizes, MatPlotLib são ferramentas para plotagem de dados e o próprio SciPy para interpolação, cálculo de espectro de frequência dentre outras.
            
            A tabela \ref{pontos_ex_inter} apresenta 9 pontos distribuídos numa matriz quadrada de ordem 3, onde, a posição (2,2) possui uma anomalia, o código \ref{lim_scipy}.1 mostra como fazer a interpolação dos pontos e como plotar o resultado (figura \ref{fig_ex_inter}).
            
            \begin{table}[H]
                \centering
                \caption{Distribuição de pontos com valor anômalo ao centro.}
                \label{pontos_ex_inter}
                \resizebox{.22\textwidth}{!}{%
                \begin{tabular}{@{}cccl@{}}
                    \toprule
                    \textbf{Pontos}  & \textbf{x}   & \textbf{y}   & \textbf{z}  \\
                    \midrule
                        1&1       &   1     &  1     \\
                        %\hline
                        2&2       &   1     &  1     \\
                        %\hline
                        3&3       &   1     &  1     \\
                        %\hline
                        4&1       &   2     &  1     \\
                        %\hline
                        5&2       &   2     &  3     \\
                        %\hline
                        6&3       &   2     &  1     \\
                        %\hline
                        7&1       &   3     &  1     \\
                        %\hline
                        8&2       &   3     &  1     \\
                        %\hline
                        9&3       &   3     &  1     \\
                    \bottomrule
                \end{tabular}%
                }
                \fonte{O Autor, 2018}
            \end{table}
            

            Exemplo Numpy:
            \begin{quote}
             \codbox{\ini \cl{import} Numpy \cl{as} np        \\
                     \ini                                     \\
                     \ini  x = np.array([1,2,3,1,2,3,1,2,3])  \\
                     \ini  y = np.array([1,1,1,2,2,2,3,3,3])  \\
                     \ini  z = np.array([1,1,1,1,3,1,1,1,1])                                                                  
             }                                            \codnum{\ref{lim_scipy}.1}
            \end{quote}
            
            Exemplo SciPy:
            
            \begin{quote}
             \codbox{\ini \cl{from} scipy \cl{import} interpolate              \\
                     \ini \cl{from} scipy.interpolate \cl{import} griddata     \\
                     \ini                                                      \\
                     \ini  xi = np.arange(x.min(), x.max(), .01)               \\
                     \ini  yi = np.arange(y.min(), y.max(), .01)               \\
                     \ini  xi,yi = meshgrid(xi,yi)                             \\
                     \ini                                                      \\
                     \ini  \cc{ Interpolate}                                   \\
                     \ini  zi = griddata((x,y),z,(xi,yi),\ob{method}=\st{'cubic'})       
             }                                                                    \codnum{cont. \ref{lim_scipy}.1}
            \end{quote}
            
            Exemplo Matplotlib:
            
            \begin{quote}
             \codbox{\ini \cl{import} matplotlib.pyplot \cl{as} plt            \\
                     \ini                                                      \\
                     \ini  plt.figure(1)                                       \\
                     \ini  plt.subplot(111)                                    \\
                     \ini                                                      \\
                     \ini  zn = np.arange(z.min(), z.max() + 0.01, .01)        \\
                     \ini                                                      \\
                     \ini  plt.plot(x, y, \st{'kx'})                           \\
                     \ini  plt.contourf(xi, yi, zi, zn)                        \\
                     \ini  plt.colorbar()                                      \\ 
                     \ini  plt.grid()                                          \\
                     \ini  plt.set\_cmap(\st{'jet'})                           \\
                     \ini  plt.show()                                          \\
             }                                                                   \codnum{cont. \ref{lim_scipy}.1}
            \end{quote}
            
            \begin{figure}[H]
                \caption{Exemplo dos pontos interpolados usando SciPy e plotados usando MatPlotLib}
                \begin{center}
                    \includegraphics[width=10cm]{texto/fig/plot_ex.png} 
                \end{center}
                \fonte{O Autor, 2018}
                \label{fig_ex_inter} 
            \end{figure}
    

	


% CAPITULO 3 ===================================================================================================================
	\chapter{Algoritmos e Processamentos}
\label{cap-areadeestudo}



	

% CAPITULO 4 ===================================================================================================================
	% RESULTADOR ESPERADOS
\chapter{Resultados Esperados}
    \label{cap-resultados}
    Espera-se ao final desse trabalho de conclusão de curso criar um programa para processamento do método \MT, escrito em Python e de fácil usabilidade.
    
    Também melhorar a compatibilidade com os diversos sistemas operacionais e distribuir sobre a licença de \en{software livre} para a comunidade geofísica. O que deve possibilitar a expansão no \MT na academia visto que qualquer pessoa terá acesso ao programa.
    
    Ao final comparar os resultados obtidos com o programa em relação a forma que vinha sendo trabalhada até então. As principais comparações serão: tempo de processamento, visualização, tempo de aprendizagem para uso da plataforma, coerência entre resultados e manipulação da forma de visualização. 

	

% CAPITULO 5 ===================================================================================================================
	% CRONOGRAMA

\chapter{Cronograma de Atividades}
    \label{cap-cronograma}
    %% CRONOGRAMA

\chapter{Cronograma de Atividades}
    \label{cap-cronograma}
    %% CRONOGRAMA

\chapter{Cronograma de Atividades}
    \label{cap-cronograma}
    %\input{cap/Cronograma}
    
    
    \section{1º Semestre}
    
    \begin{table}[h]
    \caption{Cronograma - 1º Semestre 2018}
    
    \begin{center}
    \centering
    \begin{tabular}{|l|c|c|c|c|c|c|}
    
    \hline
    {\bf Tarefa}                   		& {\bf Jan}& {\bf Fev}& {\bf Mar}& {\bf Abr}& {\bf Mai}& {\bf Jun}\\  
    \hline
    {\bf 1.} Revisão Bibliográfica 		& 	X  & 	      & 	 & 	    & 	       & 	  \\ 
    \hline
    {\bf 1.1} Magnetotelúrico      		& 	X  & 	X     & X        & 	    & 	       & 	  \\ 
    \hline
    {\bf 1.2} Python 3.5           		& 	   &          & X	 &X 	    & 	       & 	  \\ 
    \hline
    {\bf 1.2.1} Linguagem           		& 	   & 	      & X	 &X 	    & 	       & 	  \\ 
    \hline
    {\bf 1.2.2} Kivy 1.10.0           		& 	   & 	      & X	 & X	    & 	       & 	  \\ 
    \hline
    {\bf 1.2.3} Numpy, Scipy, MatplotLib        & 	   & 	      & X	 & X	    & 	       & 	  \\ 
    \hline
    {\bf 1.3} Pacote PROC-MT (INPE)           	& 	   & 	      & 	 & 	    & X	       & X	  \\ 
    \hline
    {\bf 1.3.1} Ats2asc           		& 	   & 	      & 	 & 	    &X 	       & 	  \\ 
    \hline
    {\bf 1.3.2} ProcessamentoZ           	& 	   & 	      & 	 & 	    & 	       & X	  \\ 
    \hline
    {\bf 1.3.3} Tojones           		& 	   & 	      & 	 & 	    & 	       & X	  \\ 
    \hline
   
    
    
    \end{tabular}
 
   
    \end{center}

    \fonte{O autor}  
    \end{table}

    \section{2º Semestre}
    
    \begin{table}[h]
    \caption{Cronograma - 2º Semestre 2018}
    
    \begin{center}
    \centering
    \begin{tabular}{|l|c|c|c|c|c|c|}
    
    \hline
    {\bf Tarefa}                    		 & {\bf Jul}& {\bf Ago}& {\bf Set}& {\bf Out}& {\bf Nov}& {\bf Dez}\\  
    \hline
    {\bf 1.} Construção da Interface Gráfica 	 &X 	    & X	       & 	  & 	     & 	        & 	   \\ 
    \hline
    {\bf 2.} Desenvolvimento dos Scripts         & 	    & 	       & X	  & 	     & 	        & 	   \\ 
    \hline
    {\bf 3.} Fase de testes com Dados Sintéticos & 	    & 	       & 	  & X	     & 	        & 	   \\ 
    \hline
    {\bf 4.} Fase de testes com Dados Reais      & 	    & 	       & 	  & 	     &X	        & 	   \\ 
    \hline
    {\bf 5.} Liberação do Código                 & 	    & 	       & 	  & 	     & 	        & X	   \\ 
    \hline
    
   
    
    
    \end{tabular}
 
   
    \end{center}

    \fonte{O autor}  
    \end{table}

    
    
    \section{1º Semestre}
    
    \begin{table}[h]
    \caption{Cronograma - 1º Semestre 2018}
    
    \begin{center}
    \centering
    \begin{tabular}{|l|c|c|c|c|c|c|}
    
    \hline
    {\bf Tarefa}                   		& {\bf Jan}& {\bf Fev}& {\bf Mar}& {\bf Abr}& {\bf Mai}& {\bf Jun}\\  
    \hline
    {\bf 1.} Revisão Bibliográfica 		& 	X  & 	      & 	 & 	    & 	       & 	  \\ 
    \hline
    {\bf 1.1} Magnetotelúrico      		& 	X  & 	X     & X        & 	    & 	       & 	  \\ 
    \hline
    {\bf 1.2} Python 3.5           		& 	   &          & X	 &X 	    & 	       & 	  \\ 
    \hline
    {\bf 1.2.1} Linguagem           		& 	   & 	      & X	 &X 	    & 	       & 	  \\ 
    \hline
    {\bf 1.2.2} Kivy 1.10.0           		& 	   & 	      & X	 & X	    & 	       & 	  \\ 
    \hline
    {\bf 1.2.3} Numpy, Scipy, MatplotLib        & 	   & 	      & X	 & X	    & 	       & 	  \\ 
    \hline
    {\bf 1.3} Pacote PROC-MT (INPE)           	& 	   & 	      & 	 & 	    & X	       & X	  \\ 
    \hline
    {\bf 1.3.1} Ats2asc           		& 	   & 	      & 	 & 	    &X 	       & 	  \\ 
    \hline
    {\bf 1.3.2} ProcessamentoZ           	& 	   & 	      & 	 & 	    & 	       & X	  \\ 
    \hline
    {\bf 1.3.3} Tojones           		& 	   & 	      & 	 & 	    & 	       & X	  \\ 
    \hline
   
    
    
    \end{tabular}
 
   
    \end{center}

    \fonte{O autor}  
    \end{table}

    \section{2º Semestre}
    
    \begin{table}[h]
    \caption{Cronograma - 2º Semestre 2018}
    
    \begin{center}
    \centering
    \begin{tabular}{|l|c|c|c|c|c|c|}
    
    \hline
    {\bf Tarefa}                    		 & {\bf Jul}& {\bf Ago}& {\bf Set}& {\bf Out}& {\bf Nov}& {\bf Dez}\\  
    \hline
    {\bf 1.} Construção da Interface Gráfica 	 &X 	    & X	       & 	  & 	     & 	        & 	   \\ 
    \hline
    {\bf 2.} Desenvolvimento dos Scripts         & 	    & 	       & X	  & 	     & 	        & 	   \\ 
    \hline
    {\bf 3.} Fase de testes com Dados Sintéticos & 	    & 	       & 	  & X	     & 	        & 	   \\ 
    \hline
    {\bf 4.} Fase de testes com Dados Reais      & 	    & 	       & 	  & 	     &X	        & 	   \\ 
    \hline
    {\bf 5.} Liberação do Código                 & 	    & 	       & 	  & 	     & 	        & X	   \\ 
    \hline
    
   
    
    
    \end{tabular}
 
   
    \end{center}

    \fonte{O autor}  
    \end{table}

    
    
    \section{1º Semestre}
    
    \begin{table}[h]
    \caption{Cronograma - 1º Semestre 2018}
    
    \begin{center}
    \centering
    \begin{tabular}{|l|c|c|c|c|c|c|}
    
    \hline
    {\bf Tarefa}                   		& {\bf Jan}& {\bf Fev}& {\bf Mar}& {\bf Abr}& {\bf Mai}& {\bf Jun}\\  
    \hline
    {\bf 1.} Revisão Bibliográfica 		& 	X  & 	      & 	 & 	    & 	       & 	  \\ 
    \hline
    {\bf 1.1} Magnetotelúrico      		& 	X  & 	X     & X        & 	    & 	       & 	  \\ 
    \hline
    {\bf 1.2} Python 3.5           		& 	   &          & X	 &X 	    & 	       & 	  \\ 
    \hline
    {\bf 1.2.1} Linguagem           		& 	   & 	      & X	 &X 	    & 	       & 	  \\ 
    \hline
    {\bf 1.2.2} Kivy 1.10.0           		& 	   & 	      & X	 & X	    & 	       & 	  \\ 
    \hline
    {\bf 1.2.3} Numpy, Scipy, MatplotLib        & 	   & 	      & X	 & X	    & 	       & 	  \\ 
    \hline
    {\bf 1.3} Pacote PROC-MT (INPE)           	& 	   & 	      & 	 & 	    & X	       & X	  \\ 
    \hline
    {\bf 1.3.1} Ats2asc           		& 	   & 	      & 	 & 	    &X 	       & 	  \\ 
    \hline
    {\bf 1.3.2} ProcessamentoZ           	& 	   & 	      & 	 & 	    & 	       & X	  \\ 
    \hline
    {\bf 1.3.3} Tojones           		& 	   & 	      & 	 & 	    & 	       & X	  \\ 
    \hline
   
    
    
    \end{tabular}
 
   
    \end{center}

    \fonte{O autor}  
    \end{table}

    \section{2º Semestre}
    
    \begin{table}[h]
    \caption{Cronograma - 2º Semestre 2018}
    
    \begin{center}
    \centering
    \begin{tabular}{|l|c|c|c|c|c|c|}
    
    \hline
    {\bf Tarefa}                    		 & {\bf Jul}& {\bf Ago}& {\bf Set}& {\bf Out}& {\bf Nov}& {\bf Dez}\\  
    \hline
    {\bf 1.} Construção da Interface Gráfica 	 &X 	    & X	       & 	  & 	     & 	        & 	   \\ 
    \hline
    {\bf 2.} Desenvolvimento dos Scripts         & 	    & 	       & X	  & 	     & 	        & 	   \\ 
    \hline
    {\bf 3.} Fase de testes com Dados Sintéticos & 	    & 	       & 	  & X	     & 	        & 	   \\ 
    \hline
    {\bf 4.} Fase de testes com Dados Reais      & 	    & 	       & 	  & 	     &X	        & 	   \\ 
    \hline
    {\bf 5.} Liberação do Código                 & 	    & 	       & 	  & 	     & 	        & X	   \\ 
    \hline
    
   
    
    
    \end{tabular}
 
   
    \end{center}

    \fonte{O autor}  
    \end{table}



% CAPITULO 6 ===================================================================================================================
	%\chapter{Materiais e métodos}
\label{cap-matemet}






% CAPITULO 7 ===================================================================================================================
	%\chapter{Planejamento}
\label{cap-planejamento}


\section{Fluxograma}
\label{cap-fluxograma}



\section{Cronograma}
\label{cap-cronograma}






% CAPITULO 8 ===================================================================================================================
	%\chapter{Resultados esperados}
\label{cap-resultadosesperados}




% CAPITULO 9 ===================================================================================================================
	%\chapter{Exemplo de Inserção de Figuras}
\label{cap-Figuras}

% % % % % % % % % % % % % % % % % % % % % % % % % % % % % % % % % % % % % % % % % % % % % %
% Incluir figuras no LaTeX não se dá por apenas copiar e colar, porém o processo é        %
% tão simples quanto. Use o ambiente figure demonstrado abaixo sempre que for necessário  %
% incluir uma imagem. Trocando apenas a localização/nome da imagem. O comando [h] na      %
% frente do ambiente é para que a imagem apareça o mais rápido possível no texto          %
% % % % % % % % % % % % % % % % % % % % % % % % % % % % % % % % % % % % % % % % % % % % % %

\begin{figure}[h]
\centering
\includegraphics[scale=0.5]{TEXTO/IMAGENS/nomedafigura.png}
\caption{}
\fonte{}
\end{figure}


% CAPITULO 10 ===================================================================================================================
	%\chapter{Exemplo de Tabela}
\label{cap-tabs}

% % % % % % % % % % % % % % % % % % % % % % % % % % % % % % % % % % % % % % % % % % % % 
% Para gerar tabelas mais facilmente utilize o site https://www.tablesgenerator.com/  %
% Porém para manter a configuração centralizada da tabela, copie do site apenas a     %
% partir de "\begin{tabular} até o final da tabela, NÃO SUBSTITUINDO o \end{tabular}%}%
% ou substitua porém lembre-se de incluir um %} ao final do comando.                  %
% % % % % % % % % % % % % % % % % % % % % % % % % % % % % % % % % % % % % % % % % % % % 

\begin{table}[h]
    \centering
    \caption{Exemplo de Tabela usando o Latex}
    \label{my-label}
    \resizebox{\textwidth}{!}{%
    \begin{tabular}{@{}ccccl@{}}
        \toprule
        \textbf{Um}  & \textbf{Exemplo}   & \textbf{de}   & \textbf{Tabela} &  \\
        \midrule
        \begin{tabular}[c]{@{}c@{}}
            coluna1     \\
            linha1
        \end{tabular} & 
        
        \begin{tabular}[c]{@{}c@{}}
            coluna2     \\
            linha1
        \end{tabular} & 
        
        \begin{tabular}[c]{@{}c@{}}
            coluna3     \\
            linha1
        \end{tabular} &         x               &  \\
        
        \begin{tabular}[c]{@{}c@{}}
        coluna1     \\
        linha2
        \end{tabular} & {\color[HTML]{000000} \begin{tabular}[c]{@{}c@{}}coluna2\\ linha2\end{tabular}} & \begin{tabular}[c]{@{}c@{}}coluna3\\ linha2\end{tabular}                         & x               &  \\
\begin{tabular}[c]{@{}c@{}}coluna1\\ linha3\end{tabular} & \begin{tabular}[c]{@{}c@{}}coluna2\\ linha3\end{tabular}                        & {\color[HTML]{000000} \begin{tabular}[c]{@{}c@{}}coluna3\\ linha3\end{tabular}}  & x               &  \\ 
\bottomrule
\end{tabular}%
}
\fonte{autor.}
\end{table}



% CAPITULO 11 ===================================================================================================================

% ===================================================================================================================








% BIBLIOGRAFIA ===================================================================================================================
\selectlanguage{brazilian}
\bibliographystyle{abntex2-alf}
\bibliography{biblio.bib}
% ===================================================================================================================







% GLOSSARIO ===================================================================================================================

	%\chapter*{Glossário}
% =============================================================================================================================







% APENDICE ===================================================================================================================

%\appendix

	%\chapter{Nome do Apêndice}
		%Depois do termo ``appendix'', qualquer capítulo aparecerá na forma correta, com o termo ``Apêndice''. Use apêndices quando houver material produzido pelo autor que ajuda no entendimento do trabalho mas que não faz parte do texto principal. Modelos de questionários utilizados, código fonte de programas, partituras completas, provas de teoremas acessórias, etc.
% ===================================================================================================================








% ANEXO ======================================================================================================================
%\annex

	%\chapter{Nome do Anexo}
		%Depois do termo ``annex'', qualquer capítulo aparecerá na forma correta, com o termo ``Anexo'' no título. Use anexos quando se tratar de material não produzido pelo autor, mas necessário no entendimento do trabalho. Por exemplo, definições matemáticas, sintaxe formal de linguagens de programação, trechos de manuais, etc.
% ===============================================================================================================================






\end{document}
% =============================================================================================================================
