\chapter{Justificativa}
\label{cap-justificativa}

    Esse trabalho foi pensado para auxiliar a expansão do \MT, hoje existem vários programas proprietários para processamento destes tipos de dados, mas, são programas de alto custo e fechados, isto é, não possuem transparência dos fluxos de processamentos dentro do programa. O processamento hoje usado na maioria dos trabalhos usam como base os algoritmos propostos por \citeauthor{egbert97} para a primeira fase de processamento e a rotina To Jones \cite{egbert97} para compilação dos dados em arquivos manipuláveis.
    
    Esses processos atualmente são feitos através de linhas de comando unindo diversos programas separados, o que torna o procedimento muito instável provocando diversos tipos de erros além de não prever outros. %Também força a instalação de vários pacotes separados que muitas vezes são compilados e instalados manualmente, afastando uma pessoa leiga em computação a utilizar.
    
    O programa proposto aqui pretende ser escrito em apenas uma linguagem e utilizar pacotes nativamente compatíveis, o que prevê erros de compatibilidades e torna a manutenção do código mais fácil, uma vez que estes não serão compilados.
    
    Outra forma de beneficiar a utilização do programa se dará pela construção de uma interface gráfica para todos os processos, tornando assim mais fácil utilizar e prevenir erros provocados pelo usuário.    
    
    
