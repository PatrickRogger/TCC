    Este trabalho se trata da construção de um \en{Software} para o processamento de dados \MT, esse projeto foi pensado visto a carência de programas para o trabalho com este método geofísico, o programa aqui apresentado tratará dos dados desde a coleta, através de equipamentos do tipo Metronix ADU06 ou ADU07, até as etapas de visualização dos dados passando por processamentos estatísticos, conversão de formatos de arquivos e processamento robusto.
    Todo o programa será desenvolvido utilizando a linguagem de programação Python sob a licença de \en{software} Livre, o programa unirá inúmeros \en{scripts} e rotinas consagradas no processamento de dados magnetotelúrico. Estes processamentos serão feitos através de uma interface gráfica, visando então facilitar as etapas de processamentos para os usuários, pois isso vinham sendo feitas através apenas de linhas de comando.%126
