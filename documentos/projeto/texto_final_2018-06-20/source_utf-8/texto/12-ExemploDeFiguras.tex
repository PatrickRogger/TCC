\chapter{Exemplo de Inserção de Figuras}
\label{cap-Figuras}

% % % % % % % % % % % % % % % % % % % % % % % % % % % % % % % % % % % % % % % % % % % % % %
% Incluir figuras no LaTeX não se dá por apenas copiar e colar, porém o processo é        %
% tão simples quanto. Use o ambiente figure demonstrado abaixo sempre que for necessário  %
% incluir uma imagem. Trocando apenas a localização/nome da imagem. O comando [h] na      %
% frente do ambiente é para que a imagem apareça o mais rápido possível no texto          %
% % % % % % % % % % % % % % % % % % % % % % % % % % % % % % % % % % % % % % % % % % % % % %

\begin{figure}[h]
\centering
\includegraphics[scale=0.5]{TEXTO/IMAGENS/nomedafigura.png}
\caption{}
\fonte{}
\end{figure}