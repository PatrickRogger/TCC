\chapter{Exemplo de Tabela}
\label{cap-tabs}

% % % % % % % % % % % % % % % % % % % % % % % % % % % % % % % % % % % % % % % % % % % % 
% Para gerar tabelas mais facilmente utilize o site https://www.tablesgenerator.com/  %
% Porém para manter a configuração centralizada da tabela, copie do site apenas a     %
% partir de "\begin{tabular} até o final da tabela, NÃO SUBSTITUINDO o \end{tabular}%}%
% ou substitua porém lembre-se de incluir um %} ao final do comando.                  %
% % % % % % % % % % % % % % % % % % % % % % % % % % % % % % % % % % % % % % % % % % % % 

\begin{table}[h]
    \centering
    \caption{Exemplo de Tabela usando o Latex}
    \label{my-label}
    \resizebox{\textwidth}{!}{%
    \begin{tabular}{@{}ccccl@{}}
        \toprule
        \textbf{Um}  & \textbf{Exemplo}   & \textbf{de}   & \textbf{Tabela} &  \\
        \midrule
        \begin{tabular}[c]{@{}c@{}}
            coluna1     \\
            linha1
        \end{tabular} & 
        
        \begin{tabular}[c]{@{}c@{}}
            coluna2     \\
            linha1
        \end{tabular} & 
        
        \begin{tabular}[c]{@{}c@{}}
            coluna3     \\
            linha1
        \end{tabular} &         x               &  \\
        
        \begin{tabular}[c]{@{}c@{}}
        coluna1     \\
        linha2
        \end{tabular} & {\color[HTML]{000000} \begin{tabular}[c]{@{}c@{}}coluna2\\ linha2\end{tabular}} & \begin{tabular}[c]{@{}c@{}}coluna3\\ linha2\end{tabular}                         & x               &  \\
\begin{tabular}[c]{@{}c@{}}coluna1\\ linha3\end{tabular} & \begin{tabular}[c]{@{}c@{}}coluna2\\ linha3\end{tabular}                        & {\color[HTML]{000000} \begin{tabular}[c]{@{}c@{}}coluna3\\ linha3\end{tabular}}  & x               &  \\ 
\bottomrule
\end{tabular}%
}
\fonte{autor.}
\end{table}
