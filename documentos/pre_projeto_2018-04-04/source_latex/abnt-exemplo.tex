\documentclass[a4paper,12pt]{article}
\usepackage{verbatim}
\usepackage[portuges]{babel}

\newcommand{\ii}{\'{\i}}
\newcommand{\ca}{\c{c}\~ao}
\newcommand{\co}{\c{c}\~oes}
\newcommand{\enf}{\em}
%\newcommand{\identacao}{\\[1mm] \hspace*{\parindent}}
\newcommand{\identacao}{\\[1mm] \hspace*{3cm}}
\hyphenation{pri-mei-ros}


\begin{document}

\title{Estilo Bib\TeX\ com as Normas da ABNT}
\author{Fernando Gon\c{c}alves Pilotto} 
\date{11 de Abril de 2003}
\maketitle

O estilo \verb+abnt+ \'e uma modifica{\ca} do estilo \verb+unsrt+ que
incorpora as normas da Associa{\ca} Brasileira de Normas T\'ecnicas (ABNT).

Neste texto explicamos como usar o Bib\TeX\ e o estilo \verb+abnt+.
Se houver algum problema, meu e-mail \'e
\verb+pilotto@if.ufrgs.br+.


\tableofcontents

\newpage



\section{Estilos de bibliografia}


Um estilo de bibliografia \'e um conjunto de regras que determinam
a ordena{\ca} e o modo como devem ser escritos os elementos 
de uma refer\^encia. Por exemplo, uma cita{\ca} de artigo no estilo
da {\it Physical Review} \'e escrita como
\begin{center}
R. P. Feynman, Physical Review 76 (1949) 749.
\end{center}
A mesma refer\^encia no estilo da ABNT \'e escrita na forma
\begin{center}
\parbox{0.8\textwidth}{
FEYNMAN, R. P. The theory of positrons. {\it Physical Review}, \\
New York, v. 76, n. 1, p. 749-759, May 1949.}
\end{center}
O estilo de bibliografia varia bastante com o tipo de documento que
estamos produzindo.
Cada peri\'odico tem o seu estilo, que em geral \'e bem 
parecido com o da {\it Physical Review}. Editores de livros
tamb\'em gostam de criar estilos pr\'oprios. Para documentos
publicados no Brasil, o correto \'e usar as normas da ABNT.




\section{O \LaTeX\ e a bibliografia}


No \LaTeX\ as refer\^encias bibliogr\'aficas s\~ao formatadas 
atrav\'es do ambiente \verb+thebibliography+.
\'E bom enfatizar que o \LaTeX\ faz somente
a {\it formata{\ca}} do texto: 
isto \'e, determina as quebras de linha, o espa\c{c}o
entre as palavras, o ajuste do texto com as margens, o espa\c{c}o 
entre as linhas, etc. Para produzir a refer\^encia ao artigo
de Feynman segundo o estilo da {\it Physical Review},
devemos escrever
\begin{verbatim}
\bibitem{Feynman} R. P. Feynman, Physical Review 76 (1949) 749.
\end{verbatim}
Para produzir a mesma refer\^encia segundo o estilo da ABNT
devemos escre\-ver
\begin{verbatim}
\bibitem{Feynman} FEYNMAN, R. P. The theory of positrons.
                  {\it Physical Review}, \\ 
                  New York, v. 76, n. 1, p. 749-759, May 1949.
\end{verbatim}
Vemos que \'e muito trabalhoso modificar o estilo bibliogr\'afico
se estivermos trabalhando com o \LaTeX . Isto por que a mudan\c{c}a
de estilo requer n\~ao somente uma formata{\ca} diferente do texto
como tamb\'em a {\it manipula{\ca}} do texto. Por exemplo, o nome
do autor no estilo da ABNT deve aparecer com o sobrenome em letras
mai\'usculas e os primeiros nomes devem ser abreviados.



\section{O que \'e o Bib\TeX\ }



O Bib\TeX\ \'e uma complementa{\ca} do \LaTeX\ que permite um melhor
gerenciamento de refer\^encias bibliogr\'aficas, possibilitando
a manipula{\ca} do texto.
Para us\'a-lo,
os dados bibliogr\'aficos devem ser armazenados de forma padro\-ni\-zada
em um arquivo separado com a termina{\ca} ``\verb+.bib+''.
O Bib\TeX\ processar\'a este arquivo, produzindo as refer\^encias
de acordo com o estilo biblio\-gr\'afico que escolhermos.

A refer\^encia ao artigo de Feynman, por exemplo, pode ser escrita como
\begin{verbatim}
@ARTICLE{Feynman,
         author="Richard P. Feynman", 
         title="The theory of positrons",
         journal="Physical Review",
         address="New York",
         year=1949,
         month=may,
         volume=76,
         number=1,
         pages="749-759"}
\end{verbatim}
Aqui est\'a toda a informa{\ca} que precisamos para esta
refer\^encia. Depois de escolhermos um estilo bibliogr\'afico,
o Bib\TeX\ pode determinar quais as informa{\co} que ele
vai utilizar (o t{\ii}tulo do artigo, por exemplo,  n\~ao \'e
relevante no estilo da {\it Physical Review}) e como vai tratar
cada uma delas. A forma final da refer\^encia ao artigo de Feynman
segundo o estilo da ABNT pode ser vista na refer\^encia 
\cite{Feynman}.

O Bib\TeX\ pode fazer ainda mais. No estilo ABNT n\~ao \'e
necess\'ario informar a cidade onde o peri\'odico foi
publicado, pois isto \'e feito automaticamente
(ver se{\ca} \ref{sec_abnt}).




\section{Como usar o Bib\TeX\ }


O usu\'ario do Bib\TeX\ tem apenas dois comandos \`a disposi{\ca}:
o comando \verb+\bibliographystyle{+ {\it filename}\verb+}+, 
que define o estilo bibliogr\'afico,
e o comando \verb+\bibliography{+{\it filename1, filename2,}\verb+...}+, que
indica em quais arquivos est\~ao os dados bibliogr\'aficos. Voc\^e pode
ter mais de um arquivo: por exemplo, um arquivo com trabalhos publicados
por voc\^e ou pelo seu grupo, e outro arquivo com trabalhos publicados
por outras pessoas.

\newpage

Para usar o Bib\TeX{,} voc\^e deve incluir no final do arquivo 
``\verb+.tex+'' as linhas
\begin{verbatim}
\bibliographystyle{abnt}
\bibliography{mybibliography}
\end{verbatim}
sendo \verb+mybibliography.bib+
o arquivo em formato Bib\TeX\ que cont\'em os dados de bibliografia.
Lembre de colocar o arquivo \verb+abnt.bst+ em um caminho onde
o \LaTeX\ encontre-o (por exemplo, o mesmo diret\'orio onde est\'a
o arquivo ``\verb+.tex+''). A seguir, d\^e os comandos
\begin{center}
\begin{tabular}{l}
\verb+latex +{\it filename}\verb+.tex+ \\
\verb+bibtex +{\it filename} \\
\verb+latex +{\it filename}\verb+.tex+ \\
\verb+latex +{\it filename}\verb+.tex+
\end{tabular}
\end{center}
Observe que somente as refer\^encias citadas no
arquivo {\it filename}\verb+.tex+ aparecer\~ao no documento final
({\it filename}\verb+.dvi+). Por exemplo, se o arquivo 
\verb+mybibli+ \verb+ography.bib+ cont\'em 500 refer\^encias e somente
12 foram citadas em {\it filename} \verb+.tex+, ent\~ao somente estas
12 aparecer\~ao em {\it filename}\verb+.dvi+.



\section{Como o Bib\TeX\ comunica-se com o \LaTeX }


O Bib\TeX\ e o \LaTeX\ s\~ao dois programas separados.
O funcionamento em conjunto acontece da seguinte maneira:
\begin{enumerate}
\item 
Ao processar o arquivo {\it filename}\verb+.tex+,
o \LaTeX\ ignora o comando \verb+\bibli+ \verb+ographystyle{...}+,
que para ele n\~ao tem sentido. Pela presen\c{c}a do comando
\verb+\bibliography{...}+, ele recebe a instru{\ca} de processar a
bibliografia no arquivo {\it filename}\verb+.bbl+. 
\item
Ao ser rodado, o Bib\TeX\, procura no arquivo
{\it filename}\verb+.tex+ pelos comandos \verb+\bibliographystyle{...}+
e \verb+\bibliography{...}+. Um define o estilo bibliogr\'afico, o outro
indica em qual arquivo est\~ao os dados bibliogr\'aficos.
A seguir, o Bib\TeX\ busca no arquivo
{\it filename}\verb+.aux+ as refer\^encias que foram citadas,
e procura estas refer\^encias no(s) arquivo(s) ``\verb+.bib+'' que
aparecem no comando \verb+\bibliography{...}+.
Por fim, estas refer\^encias s\~ao formatadas de acordo com o estilo que
voc\^e escolheu e escritas no arquivo {\it filename}\verb+.bbl+.
\end{enumerate}

Se voc\^e olhar o arquivo {\it filename}\verb+.bbl+, 
vai ver que ele come\c{c}a com o comando \verb+\begin{thebibliography}+.
Este \'e o ambiente de bibliografia usual do \LaTeX , e \'e
o que voc\^e digitaria se n\~ao usasse o Bib\TeX .

Agora podemos entender o porqu\^e da lista de comandos na se{\ca}
anterior. O primeiro comando, \verb+latex +{\it filename}\verb+.tex+,
gera o arquivo {\it filename}\verb+.aux+, que cont\'em as refer\^encias
citadas. O segundo comando, \verb+bibtex +{\it filename},
gera o arquivo {\it filename}\verb+.bbl+, que cont\'em o ambiente
\verb+\begin{thebibliography}+. O terceiro e o quarto comando,
\verb+latex +{\it filename}\verb+.tex+, relacionam as cita{\co}
com as refer\^encias e colocam elas em ordem.




\section{Algumas vantagens do estilo ABNT}
\label{sec_abnt}


Uma das principais vantagens do estilo \verb+abnt+ \'e que n\~ao
precisamos informar a cidade onde os peri\'odicos foram publicados,
isto \'e feito automaticamente. Por exemplo, as revistas 
{\em Physical Review} foram publicadas em Nova York at\'e 1983, e depois de
1984 foram publicadas em Woodbury. Voc\^e n\~ao precisa
saber disso, pois o \verb+abnt+ sabe.

Voc\^e tamb\'em n\~ao precisa digitar \verb+p.+, \verb+v.+, \verb+n.+,
etc. antes do n\'umero de p\'aginas, do volume e do n\'umero
do peri\'odico. Isto faz parte do estilo \verb+abnt+, e
\'e feito automaticamente. Na se{\ca} \ref{entradas} temos as
descri{\co} de todas as entradas e do que \'e feito
automaticamente e do que deve ser feito manualmente.
 
Outra vantagem \'e que voc\^e pode usar o mesmo arquivo
de bibliografias
\verb+mybibliography.bib+ para os seus artigos e para a sua 
disserta{\ca} ou tese. Para os artigos, basta utilisar um outro
estilo de bibliografia, e as refer\^encias aparecer\~ao na maneira
``usual''.




\section{Instru{\co} para o arquivo .bib}
\label{entradas}


O arquivo ``\verb+.bib+'' tem um formato padronizado.
Isto \'e necess\'ario para que o Bib\TeX\ possa manipular
v\'arios tipos de refer\^encias.

O tipo de refer\^encia define o que se chama de {\it entrada}:
para artigos em peri\'odicos, usa-se a entrada \verb+@Article+, para
livros, usa-se a entrada \verb+@Book+, para teses de doutorado,
usa-se a entrada \verb+@Phdthesis+, etc. 

Cada entrada possui {\it campos},
como por exemplo \verb+author+, \verb+journal+, etc., onde podemos
colocar os dados espec{\ii}ficos de cada refer\^encia.
Alguns campos s\~ao obrigat\'orios, outros s\~ao opcionais.

Leia com aten{\ca} a descri{\ca} de cada entrada. 
Olhe as refer\^encias para ver como cada entrada \'e impressa.


\subsection{Entradas}


\subsection*{Article}

A entrada {\enf Article} \'e usada para referenciar um artigo publicado
em um peri\'odico.
Um exemplo \'e
\begin{verbatim}
@ARTICLE{Nogami83,
         author={Y. Nogami and Akira Suzuky}, 
         title={Divergence Disease of the Pion-Baryon
               Interaction in Quark-Based Models},
         journal={Progress of Theoretical Physics},
         address={Kyoto},
         month=apr,
         year=1983,
         volume=69,
         number=4,
         pages={1184-1194}}
\end{verbatim}
Esta \'e a refer\^encia \cite{Nogami83}.
Os campos obrigat\'orios s\~ao \verb+author+, \verb+title+, \verb+journal+, 
\verb+address+, \verb+month+, \verb+year+, \verb+volume+, \verb+number+ e
\verb+pages+. Se o peri\'odico citado est\'a inclu{\ii}do na lista
da se{\ca} \ref{revistas}, o campo \verb+address+ n\~ao \'e
necess\'ario, pois o preenchimento ser\'a autom\'atico.
O campo opcional \'e \verb+note+.


\subsection*{Book}

A entrada {\enf Book} \'e usada para referenciar um livro.
Um exemplo \'e
\begin{verbatim}
@BOOK{Muta87,
   author={Taizo Muta},
   title={Foundations of Quantum Chromodynamics},
   subtitle={An Introduction to Perturbative Methods
            in Gauge Theories},
   publisher={World Scientific},
   address={Singapore},
   year=1987}
\end{verbatim}
Esta \'e a refer\^encia \cite{Muta87}. Se o livro for
uma colet\^anea de v\'arios textos, cita-se o nome dos editores, 
como por exemplo em
\begin{verbatim}
@BOOK{Abramowitz,
   editor={M. ABRAMOWITZ and I. A. STEGUN},
   title={Handbook of mathematical functions},
   publisher={National Bureau of Standards},
   address={Washington},
   year=1964}
\end{verbatim}
Esta \'e a refer\^encia \cite{Abramowitz}.
Os campos obrigat\'orios s\~ao \verb+author+ ou \verb+editor+,
\verb+title+, \verb+publisher+, \verb+address+ e \verb+year+.
Os campos opcionais s\~ao \verb+subtitle+, \verb+volume+, \verb+number+,
\verb+series+, \verb+edition+ e \verb+note+.


\begin{comment}
\subsection*{Booklet}

A entrada {\enf booklet} \'e usada para referenciar livros publicados
sem aux{\ii}lio de editora.
Sua forma geral \'e \cite{}
\begin{verbatim}
\end{verbatim}



\subsection*{Conference}

\'E o mesmo que a entrada {\enf Inproceedings}.
\end{comment}



\subsection*{Inbook}

A entrada {\enf Inbook} \'e usada para referenciar uma parte de um livro,
que pode ser um cap{\ii}tulo ou algumas p\'aginas. Ela \'e quase
id\^entica \`a entrada {\enf Book}.
\begin{verbatim}
@BOOK{Muta87,
   author={Taizo Muta},
   title={Foundations of Quantum Chromodynamics},
   subtitle={An Introduction to Perturbative Methods
            in Gauge Theories},
   publisher={World Scientific},
   address={Singapore},
   year=1987,
   chapter=2}
\end{verbatim}
Esta \'e a refer\^encia \cite{Muta87in}.
Os campos obrigat\'orios e opcionais s\~ao os mesmos da entrada
{\enf Book}, exceto pela obrigatoriedade de um campo
\verb+chapter+ ou \verb+pages+.



\subsection*{Incollection}

A entrada {\enf Incollection} \'e usada para referenciar 
uma parte (em geral um cap{\ii}tulo) de uma colet\^anea.
\begin{verbatim}
@INCOLLECTION{Jaffe79,
   author={R. L. JAFFE},
   title={The bag},
   editor={A. Zichichi},
   booktitle={Pointlike structures inside and outside hadrons},
   publisher={Plenum Press},
   address={New York},
   year=1982,
   pages={99-146}}
\end{verbatim}
Esta \'e a refer\^encia \cite{Jaffe79}.
Os campos obrigat\'orios s\~ao \verb+author+, \verb+title+, \verb+editor+,
\verb+booktitle+, \verb+publisher+, \verb+address+, \verb+year+ e
\verb+chapter+ ou \verb+pages+.
Os campos opcionais s\~ao \verb+subtitle+, \verb+volume+, \verb+number+,
\verb+series+, \verb+edition+ e \verb+note+.



\subsection*{Inproceedings}

A entrada {\enf Inproceedings} \'e usada para referenciar um artigo em um 
proceedings de uma confer\^encia.
\begin{verbatim}
@INPROCEEDINGS{Inproceedings,
   author={H. Leutwyler},
   title={Principles of Chiral Perturbation Theory},
   conference={Workshop on Hadron Physics},
   cnumber=4,
   cyear=1994,
   caddress={Gramado},
   ctitle={Topics on the Structure and Interaction
          of Hadronic Systems},
   editor={Victoria E. Herscovitz and Cesar A. Z. Vasconcellos
            and Erasmo Ferreira},
   address={Singapore},
   publisher={World Scientific},
   year=1995,
   pages={1-46}}
\end{verbatim}
Esta \'e a refer\^encia \cite{Inproceedings}.


\subsection*{Internet}

A entrada {\enf Internet} \'e usada para referenciar artigos ou
documentos dispon{\ii}\-veis na internet.
\begin{verbatim}
@INTERNET{Steane97,
   author={Andrew Steane},
   title={Quantum Computing},
   internetaddress={quant-ph/9708022 v2},
   day=24,
   month=sep,
   year=1997}
\end{verbatim}
Esta \'e a refer\^encia \cite{Steane97}.
Os campos obrigat\'orios s\~ao \verb+author+,
\verb+title+, \verb+interne+\-\verb+taddress+, 
\verb+day+, \verb+month+ e \verb+year+.
O campo opcional \'e \verb+note+.




\begin{comment}
\subsection*{Manual}

A entrada {\enf Manual} \'e usada para referenciar documenta{\ca}
t\'ecnica.
Sua forma geral \'e \cite{}
\begin{verbatim}
\end{verbatim}
\end{comment}



\subsection*{Mastersthesis}

A entrada {\enf Mastersthesis} \'e usada para referenciar uma
disserta{\ca} de mes\-tra\-do.
\begin{verbatim}
@MASTERSTHESIS{Master,
   author={Jo{\~a}o Ningu\'em},
   title={Como Passar o Tempo Durante Dois Anos},
   address={Porto Alegre},
   school={Universidade Federal do Rio Grande do Sul,
           Instituto de F\'{\i}sica,
           Curso de P\'os-Gradua\c{c}\~ao em F\'{\i}sica},
   year=1995}
\end{verbatim}
Esta \'e a refer\^encia \cite{Master}.
Os campos obrigat\'orios s\~ao \verb+author+,
\verb+title+, \verb+address+, \verb+school+ e \verb+year+.
O campo opcional \'e \verb+note+.


\begin{comment}
\subsection*{Misc}

A entrada {\enf Misc} \'e usada para referenciar algo que n\~ao se 
enquadre nos moldes das outras entradas.
Sua forma geral \'e \cite{Misc}
\begin{verbatim}
\end{verbatim}
\end{comment}



\subsection*{Phdthesis}

A entrada {\enf Phdthesis} \'e usada para referenciar teses de doutorado.
\begin{verbatim}
@PHDTHESIS{Phd,
   author={Ningu{\'e}m~da~Silva, Jo{\~a}o},
   title={Como Passar o Tempo Durante Quatro Anos},
   address={Porto Alegre},
   school={Universidade Federal do Rio Grande do Sul,
           Instituto de F\'{\i}sica,
           Curso de P\'os-Gradua\c{c}\~ao em F\'{\i}sica},
   year=1999}
\end{verbatim}
Esta \'e a refer\^encia \cite{Phd}.
Os campos obrigat\'orios s\~ao \verb+author+,
\verb+title+, \verb+address+, \verb+school+ e \verb+year+.
O campo opcional \'e \verb+note+.




\subsection*{Proceedings}

A entrada {\enf Proceedings} \'e usada para referenciar os proceedings de uma 
confer\^encia.
Sua forma geral \'e \cite{Proceedings}
\begin{verbatim}
@PROCEEDINGS{Proceedings,
   conference={Workshop on Hadron Physics},
   cnumber=4,
   cyear=1994,
   caddress={Gramado},
   title={Topics on the Structure and Interaction
         of Hadronic Systems},
   editor={Victoria E. Herscovitz and Cesar A. Z. Vasconcellos
            and Erasmo Ferreira},
   address={Singapore},
   publisher={World Scientific},
   year=1995}
\end{verbatim}




\begin{comment}
\subsection*{Techreport}

A entrada {\enf Techreport} \'e usada para referenciar um artigo
publicado por uma institui{\ca}.
Sua forma geral \'e \cite{Techreport}
\begin{verbatim}
\end{verbatim}



\subsection*{Unpublished}

A entrada {\enf Unpublished} \'e usada para referenciar um artigo
ainda n\~ao publicado.
Sua forma geral \'e \cite{Unpublished}
\begin{verbatim}
\end{verbatim}
\end{comment}



\subsection{Campos}


O campo ``\verb+author+'' \'e
o mais complicado. Os autores devem ser separados por
``\verb+and+''. O primeiro nome de cada autor pode ser escrito por extenso
ou abreviado. O sobrenome pode ser escrito na forma usual, com
a primeira letra em mai\'usculo e as outras em min\'usculo.
Nomes que incluam ``Jr.'' no final devem ser escritos entre chaves, como
``\verb+C. G. {CALLAN Jr.}+''. 
Colocando ``\verb+and OTHERS+'' produzir\'a
a substitui{\ca} dos nomes seguintes por ``et. al''.
Olhe os exemplos de entradas para mais possibilidades.

O campo \verb+title+ \'e formatado automaticamente com a primeira 
letra da primeira palavra em mai\'usculo, todas as outras em min\'usculo. 
Para que nomes pr\'oprios sejam formatados
corretamente, deve-se proteger a letra mai\'uscula usando
chaves, como por exemplo em
\begin{center}
\verb+title={Monopolos de {D}irac}+
\end{center}

O campo \verb+journal+ aceita as abrevia{\co} listadas na
se{\ca} \ref{revistas}.

O campo \verb+month+ aceita as abrevia{\co} listadas na
se{\ca} \ref{sec_mes}.




\section{Usando outros estilos}


O arquivo \verb+mybibliography.bib+, o qual cont\'em os dados 
bibliogr\'aficos, pode ser usado com outros estilos de bibliografia
sem gerar incompatibilidades. A \'unica inconveni\^encia \'e a
refer\^encia \`as p\'aginas de um artigo. Usualmente coloca-se somente
a p\'agina inicial, mas de acordo com as regras da ABNT, a p\'agina
final tamb\'em deve ser referenciada.



\section{Abrevia{\co} de revistas}
\label{revistas}


O campo \verb+journal+ aceita as abrevia{\co}
\identacao
{\verb+actapola+} = {Acta Physica Polonica A}
\identacao
{\verb+actapolb+} = {Acta Physica Polonica B}
\identacao
{\verb+advnuclp+} = {Advances in Nuclear Physics}
\identacao
{\verb+amjp+} = {American Journal of Physics}
\identacao
{\verb+annphys+} = {Annals of Physics}
\identacao
{\verb+arnpc+} = {Annual Review of Nuclear and Particle Science}
\identacao
{\verb+canjphys+} = {Canadian Journal of Physics}
\identacao
{\verb+commnuclp+} = {Comments on Nuclear and Particle Physics}
\identacao
{\verb+npa+} = {Nuclear Physics A}
\identacao
{\verb+npb+} = {Nuclear Physics B}
\identacao
{\verb+npc+} = {Nuclear Physics C}
\identacao
{\verb+npd+} = {Nuclear Physics D}
\identacao
{\verb+pla+} = {Physics Letters A}
\identacao
{\verb+plb+} = {Physics Letters B}
\identacao
{\verb+physrep+} = {Physics Reports}
\identacao
{\verb+pr+} = {Physical Review}
\identacao
{\verb+pra+} = {Physical Review A}
\identacao
{\verb+prb+} = {Physical Review B}
\identacao
{\verb+prc+} = {Physical Review C}
\identacao
{\verb+prd+} = {Physical Review D}
\identacao
{\verb+pre+} = {Physical Review E}
\identacao
{\verb+progtheophys+} = {Progress of Theoretical Physics}
\begin{comment}
\identacao
{\verb+zfpa+} = {Zeitschrift f\"ur Physik A}
\identacao
{\verb+zfpb+} = {Zeitschrift f\"ur Physik B}
\identacao
{\verb+zfpc+} = {Zeitschrift f\"ur Physik C}
\end{comment}




\section{Abrevia{\co} dos meses}
\label{sec_mes}


O campo \verb+month+ aceita as abrevia{\co} para os meses em ingl\^es:
\begin{center}
\begin{tabular}{l@{\hspace{1.2cm}}l}
\verb+jan = "Jan."+  & \verb+jul = "July"+ \\
\verb+feb = "Feb."+  & \verb+aug = "Aug."+ \\
\verb+mar = "Mar."+  & \verb+sep = "Sept."+ \\
\verb+apr = "Apr."+  & \verb+oct = "Oct."+ \\
\verb+may = "May"+   & \verb+nov = "Nov."+ \\
\verb+jun = "June"+  & \verb+dec = "Dec."+
\end{tabular}
\end{center}
Para revistas em outras l{\ii}nguas, deve-se usar a abrevia{\ca}
correspondente. 


%\appendix
%\addcontentsline{toc}{chapter}{REFER\^ENCIAS}

%\bibliographystyle{unsrt}
\bibliographystyle{abnt}
\bibliography{abntdoc}
%\bibliography{periodicos_extenso,abntdoc}
%\bibliography{macros,abntdoc}



\end{document}
